% Ben Crowder <crowderb@blankslate.net>
% http://www.blankslate.net/lang/etexts.php

% Image files scanned in by Sean Crist <kurisuto@unagi.cis.upenn.edu>
% http://www.ling.upenn.edu/~kurisuto/germanic/oi_sweet_about.html

% OCRed and reformatted for Project Gutenberg by Ben Crowder
% <crowderb@blankslate.net>

\documentclass[12pt,letterpaper]{book}

\usepackage{indentfirst}
\usepackage{float}
\usepackage{exaccent}
\usepackage{fancyhdr}
\usepackage[modulo]{lineno}

%%%

% Tell LaTeX that we want Omega fonts
\def\rmdefault{omlgc}
\def\ttdefault{uctt}

% Load the OCP for UTF-8 encoded text
\ocp\OCPutf=inutf8
% Load the OCP for reencoding Unicode to omlgc encoding
\ocp\OCPunitolat=uni2lat
% Load the OCP for composing characters we need
\ocp\OCPoiaccents=oiaccents
% Load the OCP for composing typographic symbols
\ocp\OCPfixthings=fixthings
% Set up an OCP list: 
%     Conversion UTF-8 -> Unicode
%     Typographic symbols
%     Compose accented characters
%     Unicode -> OMLGC encoding
\ocplist\OCPlistutf=
    \addbeforeocplist 1 \OCPutf
    \addbeforeocplist 1 \OCPfixthings
    \addbeforeocplist 1 \OCPoiaccents
    \addbeforeocplist 1 \OCPunitolat
    \nullocplist
% Activate the OCP list we just built
\pushocplist\OCPlistutf

%%%

\pagestyle{fancy}
\renewcommand{\sectionmark}[1]{\markright{\thesection\ #1}}
\lhead[\thepage]{\rightmark}
\rhead[\it{An Icelandic Primer}]{\thepage}
\chead[]{}
\lfoot[]{}
\cfoot[]{}
\rfoot[]{}

%%%

\newcommand{\gap}[1][.25in]{\hspace{#1}}
\newcommand{\bgap}[1][.5in]{\hspace{#1}}

%%%

\newcommand\newcaption{\small\refstepcounter{table}%
	\centering Table~\thetable}

%%%

\newcommand\emptypage{\clearpage{\pagestyle{empty}\cleardoublepage}}

%%%

\title{\Huge{An Icelandic Primer}\\
\vspace{\baselineskip}
\Large{\textit{With Grammar, Notes, and Glossary}}}
\date{}
\author{By Henry Sweet, M.A.}

%%% End preamble %%%

\begin{document}

\frontmatter

\maketitle

%%% Verso %%%

\noindent
This book has been modified slightly from the original to make the paradigm
references clear.  The original was the second edition, published by Oxford
at the Clarendon Press, 1895.

\vspace{\baselineskip}
\noindent
The original page
images\footnote{http://www.ling.upenn.edu/~kurisuto/germanic/oi\_sweet\_about.html}
were scanned in by Sean Crist\footnote{kurisuto@unagi.cis.upenn.edu}.  They
were then OCRed and reformatted for Project Gutenberg into text, HTML, and
\TeX\ by Ben Crowder\footnote{crowderb@blankslate.net}.  Other versions of the text
may be found at the current homepage for the
primer\footnote{http://www.blankslate.net/lang/etexts.php} or through Project
Gutenberg\footnote{http://www.gutenberg.net/}.  Special thanks to Jonas
Öster\footnote{d97ost@dtek.chalmers.se} for providing the Omega ΩTP code to fix
nondisplaying characters.

\thispagestyle{empty}

%%% End verso %%%

\chapter*{Preface}

The want of a short and easy introduction to the study of
Icelandic has been felt for a long time---in fact, from the very
beginning of that study in England.  The \textit{Icelandic Reader},
edited by Messrs.\ Vigfusson and Powell, in the Clarendon Press
Series, is a most valuable book, which ought to be in the hands
of every student; but it still leaves room for an elementary
primer.  As the engagements of the editors of the Reader would
have made it impossible for them to undertake such a work for
some years to come, they raised no objections to my proposal to
undertake it myself.  Meanwhile, I found the task was a more
formidable one than I had anticipated, and accordingly, before
definitely committing myself to it, I made one final attempt to
induce Messrs.\ Vigfusson and Powell to take it off my hands; but
they very kindly encouraged me to proceed with it; and as I
myself thought that an Icelandic primer, on the lines of my
Anglo-Saxon one, might perhaps be the means of inducing some
students of Old English to take up Icelandic as well, I
determined to go on.

In the spelling I have not thought it necessary to adhere
strictly to that adopted in the Reader, for the editors have
themselves deviated from it in their \textit{Corpus Poeticum
Boreale}, in the way of separating \textit{ǫ} from \textit{ö}, etc.
My own principle has been to deviate as little as possible from the
traditional spelling followed in normalized texts.  There is, indeed, no
practical gain for the beginner in writing \textit{tīme} for \textit{tīmi},
discarding \textit{ð}, etc., although these changes certainly bring us
nearer the oldest MSS., and cannot be dispensed with in
scientific works.  The essential thing for the beginner is to
have \textit{regular} forms presented to him, to the exclusion, as far
as possible, of isolated archaisms, and to have the defective
distinctions of the MSS.\ supplemented by diacritics.  I have not
hesitated to substitute (¯) for (´) as the mark of length; the
latter ought in my opinion to be used exclusively---in Icelandic
as well as in Old English and Old Irish---to represent the actual
accents of the MSS.

In the grammar I have to acknowledge my great obligations to
Noreen's \textit{Altisländische Grammatik}, which is by far the best
Icelandic grammar that has yet appeared---at least from that
narrow point of view which ignores syntax, and concentrates
itself on phonology and inflections.

The texts are intended to be as easy, interesting, and
representative as possible.  With such a language, and such a
master of it as Snorri to choose from, this combination is not
difficult to realise.  The beginner is indeed to be envied who
makes his first acquaintance with the splendid mythological tales
of the North, told in an absolutely perfect style.  As the death
of Olaf Tryggvason is given in the Reader only from the longer
recension of the Heimskringla, I have been able to give the
shorter text, which is admirably suited for the purposes of this
book.  The story of Auðun is not only a beautiful one in itself,
but, together with the preceding piece, gives a vivid idea of the
Norse ideal of the kingly character, which was the foundation of
their whole political system.  As the Reader does not include
poetry (except incidentally), I have added one of the finest of
the Eddaic poems, which is at the same time freest from obscurity
and corruption---the song of Thor's quest of his hammer.

In the glossary I have ventured to deviate from the very
inconvenient Scandinavian arrangement, which puts \textit{þ},
\textit{æ}, \textit{œ}, right at the end of the alphabet.

I have to acknowledge the great help I have had in preparing the
texts and the glossary from Wimmer's \textit{Oldnordisk Læsebog}, which
I consider to be, on the whole, the best reading-book that exists
in any language.  So excellent is Wimmer's selection of texts,
that it was impossible for me to do otherwise than follow him in
nearly every case.

In conclusion, it is almost superfluous to say that this book
makes no pretension to originality of any kind.  If it
contributes towards restoring to Englishmen that precious
heritage---the old language and literature of Iceland---which our
miserably narrow scheme of education has hitherto defrauded them
of, it will have fulfilled its purpose.

\begin{flushright}
HENRY SWEET.
\end{flushright}

\textsc{London,}\\
\textit{February}, 1886

\tableofcontents{}

\emptypage

\mainmatter

\part{Grammar}
\thispagestyle{empty}

\emptypage

\chapter{Pronunciation}

\textbf{1.} This book deals with Old Icelandic in its classical period,
between 1200 and 1350.

\textbf{2.} The Icelandic alphabet was founded on the Latin, with the
addition of \textit{þ} and \textit{ð}, and of the modified letters \textit{ę},
\textit{ǫ}, \textit{ø}, which last is in this book written \textit{ö},
\textit{ǫ̈}.

\section{Vowels}

\textbf{3.} The vowel-letters had nearly the same values as in Old
English.  Long vowels were often marked by (´).  In this book
long vowels are regularly marked by (¯)\footnote{Note that the longs
of \textit{ę}, \textit{ö} are written \textit{æ}, \textit{œ}, respectively.}.
See Table~\ref{tab:3} for the elementary vowels and diphthongs, with
examples, and key-words from English, French (F.), and German (G.).

\begin{table}[htbp]
\begin{center}
\begin{minipage}{3in}
\begin{tabular}{lcll}
	a & \textit{as in} & mann (G.) & halda (\textit{hold})\\
	ā & '' & father & rāð (\textit{advice})\\
	e & '' & été (F.) & gekk (\textit{went})\\
	ē\protect\footnote{Where no keyword is given for a long vowel, its sound is
		that of the corresponding short vowel lengthened.} & & . . . & lēt (let
		\textit{pret.})\\
	ę & '' & men & męnn (\textit{men})\\
	æ & '' & there & sær (\textit{sea})\\
	i & '' & fini (F.) & mikill (\textit{great})\\
	ī & & . . . & lītill (\textit{little}) \\
	o & '' & beau (F.) & orð (\textit{word})\\
	ō & & . . . & tōk (\textit{look})\\
	ǫ & '' & not & hǫnd (\textit{hand})\\
	ö & '' & peu (F.) & kömr (\textit{comes})\\
	œ & & . . . & fœra (\textit{bring})\\
	ǫ̈ & '' & peur (F.) & gǫ̈ra (\textit{make})\\
	u & '' & sou (F.) & upp (\textit{up})\\
	ū & & . . . & hūs (\textit{house})\\
	y & '' & tu (F.) & systir (\textit{sister})\\
	ȳ & & . . . & lȳsa (\textit{shine})\\
	au & '' & haus (G.) & lauss (\textit{loose})\\
	ei & = & ę + i & bein (\textit{bone})\\
	ey & = & ę + y & leysa (\textit{loosen})\\
\end{tabular}
\end{minipage}
\end{center}
\newcaption
\label{tab:3}
\end{table}

\textbf{4.} The unaccented \textit{i} in \textit{systir}, etc.\ (which is generally
written \textit{e} in the MSS.) probably had the sound of \textit{y} in
\textit{pity}, which is really between \textit{i} and \textit{e}.  The
unacc.\ \textit{u} in \textit{fōru} (they went), etc.\ (which is generally written
\textit{o} in the MSS.) probably had the sound of \textit{oo} in \textit{good}.

Note that several of the vowels go in pairs of \textit{close} and
\textit{open}, as shown in Table~\ref{tab:4}.

\begin{table}[htbp]
\begin{center}
\begin{tabular}{lllllll}
	close: & e & ē & o & ō & ö & œ\\
	open: & ę & æ & ǫ & -- & ǫ̈ & --\\
\end{tabular}
\end{center}
\newcaption
\label{tab:4}
\end{table}

\section{Consonants}

\textbf{5.} Double consonants followed by a vowel must be pronounced
really double, as in Italian.  Thus the \textit{kk} in \textit{drekka} (to
drink) must be pronounced like the \textit{kc} in \textit{bookcase}, while the
\textit{k} in \textit{dręki} (dragon) is single, as in \textit{booking}.  When final
(or followed by another cons.) double conss.\ are pronounced long,
as in \textit{munn} (mouth \textit{acc.}), \textit{hamarr} (hammer \textit{nom.}),
\textit{steinn} (stone \textit{nom.}), distinguished from \textit{mun} (will
\textit{vb.}), and the accusatives \textit{hamar}, \textit{stein}.

\textbf{6.} \textit{k} and \textit{g} had a more front (palatal) sound before the front
vowels \textit{e}, \textit{ę}, \textit{i}, \textit{ö}, \textit{ǫ̈}, \textit{y}, and their
longs, as also before \textit{j}, as in \textit{kęnna} (known), \textit{keyra}
(drive), \textit{gǫ̈ra} (make), \textit{liggja} (lie).

\textbf{7.} \textit{kkj}, \textit{ggj} were probably pronounced simply as double front
\textit{kk}, \textit{gg}, the \textit{j} not being pronounced separately.

\textbf{8.} {\bf f} had initially the sound of our \textit{f}, medially and finally
that of \textit{v}, as in \textit{gefa} (give), \textit{gaf} (gave), except of
course in such combinations as \textit{ft}, where it had the sound of
\textit{f}.

\textbf{9.} {\bf g} was a stopped (back or front---guttural or palatal)
cons.\ initially and in the combination \textit{ng}, the two \textit{g}'s in
\textit{ganga} (go) being pronounced as in \textit{go}.  It had the open sound of
G.\ \textit{g} in \textit{sagen} medially before the back vowels \textit{a}, \textit{o},
\textit{ǫ}, \textit{u}, and all conss.\ except \textit{j}, and finally:---\textit{saga}
(tale), \textit{dǫgum} (with days); \textit{sagði} (he said); \textit{lag} (he lay).
Before the front vowels and \textit{j} it had the sound of G.\ \textit{g} in
\textit{liegen}, or nearly that of \textit{j} (our \textit{y}), as in \textit{sęgir}
(says), \textit{sęgja} (to say).

\textbf{10.} Before voiceless conss.\ (\textit{t}, \textit{s}) \textit{g} seems to have been
pronounced \textit{k}, as in \textit{sagt} (said), \textit{dags} (day's).

\textbf{11.} The \textit{g} was always sounded in the combination \textit{ng},
as in \textit{single}, not as in \textit{singer}.

\textbf{12.} {\bf h} was sounded before \textit{j} in such words as \textit{hjarta}
(heart) much as in E.\ \textit{hue} (= hjū).  \textit{hl}, \textit{hn}, \textit{hr},
\textit{hv} probably represented voiceless \textit{l}, \textit{n}, \textit{r},
\textit{w} respectively, \textit{hv} being identical with E.\ \textit{wh}:
\textit{hlaupa} (leap), \textit{hnīga} (bend), \textit{hringr} (ring), \textit{hvat}
(what).

\textbf{13.} {\bf j} is not distinguished from \textit{i} in the MSS\@.  It had the
sound of E.\ \textit{y} in \textit{young}: \textit{jǫrð} (earth), \textit{sętja}
(to set).

\textbf{14.} {\bf p} in \textit{pt} probably had the sound of \textit{f}: \textit{lopt} (air).

\textbf{15.} {\bf r} was always a strong point trill, as in Scotch.

\textbf{16.} {\bf s} was always sharp.

\textbf{17.} {\bf v} (which was sometimes written \textit{u} and \textit{w}) had the sound
of E.\ \textit{w}: \textit{vel} (well), \textit{hǫggva} (hew).

\textbf{18.} {\bf z} had the sound of \textit{ts}: \textit{bęztr} (best).

\textbf{19.} {\bf þ} and {\bf ð} were used promiscuously in the older MSS., the
very oldest using \textit{þ} almost exclusively.  In Modern Icelandic
\textit{þ} is written initially to express the sound of E.\ hard \textit{th},
\textit{ð} medially and finally to express that of soft \textit{th}; as there
can be no doubt that this usage corresponds with the old
pronunciation, it is retained in this book: \textit{þing} (parliament),
\textit{faðir} (father), \textit{við} (against).  In such combinations as
\textit{pð} the \textit{ð} must of course be pronounced \textit{þ}.

\section{Stress}

\textbf{20.} The stress (accent) is always on the first syllable.

\emptypage

\chapter{Phonology}

\section{Vowels}

\textbf{21.} The vowels are related to one another in different ways, the
most important of which are \textit{mutation} (umlaut), \textit{fracture}
(brechung), and \textit{gradation} (ablaut).

\subsection{Mutation}

\textbf{22.} The following changes are {\bf i}-mutations (caused by an older
\textit{i} or \textit{j} following, which has generally been
dropped)\footnote{Many of the \textit{i}'s which appear in derivative and
inflectional syllables are late weakenings of \textit{a} and other
vowels, as in \textit{bani} (death) = Old English \textit{bana}; these
do not cause mutation.}:

\gap{\bf a} ({\bf ǫ})\ldots  {\bf ę} :--- \textit{mann} (man \textit{acc.}),
\textit{męnn} (men); \textit{hǫnd} (hand), \textit{hęndr} (hands).

\gap{\bf ā}\ldots {\bf æ} :--- \textit{māl} (speech), \textit{mæla} (speak).

\gap{\bf e} ({\bf ja}, {\bf jǫ})\ldots {\bf i} :--- \textit{verðr}
(worth), \textit{virða} (estimate).

\gap{\bf u} ({\bf o})\ldots {\bf y} :--- \textit{fullr} (full), \textit{fylla}
(to fill); \textit{lopt} (air), \textit{lypta} (lift).

\gap{\bf ū}\ldots {\bf ȳ} :--- \textit{brūn} (eyebrow), pl.\ \textit{brȳnn}.

\gap{\bf o}\ldots {\bf ö} :--- \textit{koma} (to come), \textit{kömr} (comes).

\gap{\bf ō}\ldots {\bf œ} :--- \textit{fōr} (went), \textit{fœra} (bring).

\gap{\bf au}\ldots {\bf ey} :--- \textit{lauss} (loose), \textit{leysa}
(loosen).

\gap{\bf jū} ({\bf jō})\ldots {\bf ȳ} :--- \textit{sjūkr} (sick), \textit{sȳki}
(sickness); \textit{ljōsta} (strike), \textit{lȳstr} (strikes).

\textbf{23.} The change of \textit{a} into \textit{ę} is sometimes the result of a
following \textit{k}, \textit{g}, or \textit{ng}, as in \textit{dęgi} dat.\
sg.\ of \textit{dagr} (day), \textit{tękinn} (taken), \textit{gęnginn} (gone),
inf.\ \textit{taka}, \textit{ganga}.  \textit{i} appears instead of \textit{e}, and
\textit{u} instead of \textit{o} before a nasal followed by another cons.:
cp.\ \textit{binda} (to bind), \textit{bundinn} (bound) with \textit{bresta} (burst)
ptc.\ prt.\ \textit{brostinn}.

\textbf{24.} There is also a {\bf u}-mutation, caused by a following \textit{u},
which has often been dropped:

\gap{\bf a}\ldots {\bf ǫ} :--- \textit{dagr} (day) dat.\ pl.\ \textit{dǫgum};
\textit{land} (land) pl.\ \textit{lǫnd}.

\textbf{25.} Unaccented \textit{ǫ} becomes \textit{u}, as in \textit{sumur} pl.\ of
\textit{sumar} (summer), \textit{kǫlluðu} (they called), infin.\ \textit{kalla}.

\subsection{Fracture}

\textbf{26.} The only vowel that is affected by fracture is \textit{e}: when
followed by original \textit{a} it becomes \textit{ja}, when followed by
original \textit{u} it becomes \textit{jǫ}, as in \textit{jarðar} gen.\ of
\textit{jǫrð} (earth)\footnote{Cp.\ German \textit{erde}.}.  When followed
by original \textit{i}, the \textit{e} is, of course, mutated to \textit{i},
as in \textit{skildir} plur.\ nom.\ of \textit{skjǫldr} (shield), gen.\ {\it
skjaldar}.

\subsection{Gradation}

\textbf{27.} By gradation the vowels are related as follows:---

\gap{\bf a}\ldots {\bf ō} :--- \textit{fara} (go) pret.\ \textit{fōr}, whence by
mut.\ \textit{fœra} (bring).

\gap{\bf e} (i, ja)\ldots {\bf a}\ldots {\bf u} (o) :--- \textit{bresta}
(burst), prt.\ \textit{brast}, prt.\ pl.\ \textit{brustu}, ptc.\ prt.\ \textit{brostinn};
\textit{finna} (find), \textit{fundinn} (found \textit{ptc.}), \textit{fundr}
(meeting).

\gap{\bf e}\ldots {\bf a}\ldots {\bf ā}\ldots {\bf o} :--- \textit{stela} (steal),
prt.\ \textit{stal}, prt.\ pl.\ \textit{stālu}, ptc.\ prt.\ \textit{stolinn}.

\gap{\bf e}\ldots {\bf a}\ldots {\bf ā}\ldots {\bf e} :--- \textit{gefa} (give),
\textit{gaf} (he gave), \textit{gāfu} (they gave), \textit{gefinn} (given),
\textit{gjǫf} (gift), \textit{u}-fracture of \textit{gef-}, \textit{gæfa} (luck)
mut.\ of \textit{gāf-}.

\gap{\bf ī}\ldots {\bf ei}\ldots {\bf i} :--- \textit{skīna} (shine), \textit{skein}
(he shone), \textit{skinu} (they shone).  \textit{sōl-skin} (sunshine).

\gap{\bf jū} (jō)\ldots {\bf au}\ldots {\bf u}\ldots {\bf o} :--- \textit{ljūga}
(tell a lie), prt.\ \textit{laug}, prt.\ pl.\ \textit{lugu},
ptc.\ prt.\ \textit{loginn}.  \textit{lygi} (lie \textit{sbst.}) mut.\ of \textit{lug-}.
\textit{skjōta} (shoot), \textit{skjōtr} (swift), \textit{skotinn} (shot
\textit{ptc.}), \textit{skot} (shot \textit{subst.}).

\subsection{Other changes}

\textbf{28.} All final vowels are long in accented syllables: \textit{þā} (then),
\textit{nū} (now).

\textbf{29.} Inflectional and derivative vowels are often dropt after long
accented vowels: cp.\ \textit{ganga} (to go) with \textit{fā} (to get), the
dat.\ plurals \textit{knjām} (knees) with \textit{hūsum} (houses).

\textbf{30.} Vowels are often lengthened before \textit{l} + cons.: \textit{hālfr}
(half \textit{adj.}), \textit{fōlk} (people); cp.\ \textit{fōlginn} (hidden)
with \textit{brostinn} (burst \textit{ptc.}).

\section{Consonants}

\textbf{31.} \textit{v} is dropped before \textit{o} and \textit{u}: \textit{vaxa}
(to grow), prt.\ \textit{ōx}, \textit{vinna} (to win), \textit{unninn}
(won \textit{ptc.}), \textit{svelta} (to starve), \textit{soltinn}
(starved, hungry).

Final \textit{r} is often assimilated to a preceding cons.

\textbf{32.} \textit{*-lr}, \textit{*-nr}, \textit{*-sr} always become \textit{-ll},
\textit{-nn}, \textit{-ss} after a long vowel or diphthong, as in \textit{stōll}
(chair \textit{nom.}), acc.\ \textit{stōl}, \textit{steinn} (stone \textit{nom.}),
acc.\ \textit{stein}, \textit{vīss} (wise \textit{masc.\ nom.\ sg.}), \textit{vīs}
fem.\ nom.\ sg., and in unacc.\ syllables, as in the masc.\ sg.\ nominatives
\textit{mikill} (great), fem.\ \textit{mikil}, \textit{borinn} (carried), fem.
\textit{borin}, \textit{ȳmiss} (various) fem.\ \textit{ȳmis}.

\textbf{33.} Words in which \textit{l}, \textit{n}, \textit{r}, \textit{s} are preceded
by a cons.\ drop the \textit{r} entirely, as in the masc.\ nominatives
\textit{jarl} (earl), \textit{hrafn} (raven), \textit{vitr} (wise), \textit{þurs}
(giant), \textit{lax} (salmon).

\textbf{34.} If \textit{l} and \textit{n} are preceded by a short accented vowel,
the \textit{r} is generally kept, as in \textit{stelr} (steals),
\textit{vinr}, (friend), \textit{sr} becoming \textit{ss}, as elsewhere.

\textbf{35.} \textit{r} is kept after \textit{ll}, and generally after \textit{nn},
as in the masc.\ nom.\ \textit{allr} (all), and in \textit{bręnnr} (burns).

\textbf{36.} \textit{z} often stands for \textit{ðs} as well as \textit{ts}, as in
\textit{þēr þykkizk} (ye seem) = \textit{*þykkið-sk}, \textit{Vest-firzkr}
(belonging to the West Firths) = \textit{-*firðskr} (\textit{fǫrðr},
firth).

\textbf{37.} Inflectional \textit{t} is generally doubled after a long accented
vowel: \textit{fār} (few) neut.\ \textit{fātt} (cp.\ \textit{allr} `all,'
neut.\ \textit{allt}), \textit{sā} (I saw), \textit{sātt} `thou sawest.'

\emptypage

\chapter{Inflections}

\section{Nouns}

\textbf{38.} \textbf{Gender.}  There are three genders in Icelandic---masculine,
feminine, and neuter.  The gender is partly natural, partly
grammatical, generally agreeing with the gender in Old English.
Compound words follow the gender of their last element.

\textbf{39.} \textbf{Strong and Weak.}  All weak nouns end in a vowel in the
nom.\ sg.\ and in most of the other cases as well.  Most strong nouns
end in a cons.\ in the nom.\ sg.

\textbf{40.} \textbf{Cases.}  There are four cases---nominative, accusative,
dative, genitive.  All nouns (except a few contractions) have the
gen.\ pl.\ in \textit{-a} (\textit{fiska}, of fishes), and the dat.\ pl.\ in
\textit{-um} (\textit{fiskum}).  All strong masculines (\textit{fiskr}) and
some strong feminines (\textit{brūðr}, bride) take \textit{r}\footnote{Subject,
of course, to the assimilations described above.} in the nom.\ sg.  Most
strong feminines show the bare root in the nom.\ sg.\ with \textit{u}-mutation,
if possible (\textit{āst}, favour, \textit{fǫr}, journey).  The nom.\ pl.\ of all
strong masc.\ and fem.\ nouns ends in \textit{r} (\textit{fiskar}, \textit{āstir}).
The acc.\ pl.\ of fem.\ nouns is the same as the nom.\ pl.\ (\textit{āstir}).
The acc.\ pl.\ of masc.\ strong nouns always ends in a vowel (\textit{fiska}).
The plur.\ nom.\ and acc.\ of neuters is the same as the sing.\ nom.\ and
acc., except that in the plur.\ nom.\ and acc.\ they take \textit{u}-mutation,
if possible (\textit{hūs}, houses, \textit{lǫnd}, lands).

\textbf{41.} The declensions are most conveniently distinguished by the
acc.\ plur.

\subsection{Strong Masculines}

\subsubsection{(1) \textit{a}-plurals}

\begin{table}[htbp]
\begin{center}
\begin{tabular}{rll}
	& \textbf{Singular} & \textbf{Plural} \\
	\textit{Nom.} & fisk-r (\textit{fish}) & fisk-ar \\
	\textit{Acc.} & fisk & fisk-a \\
	\textit{Dat.} & fisk-i & fisk-um \\
	\textit{Gen.} & fisk-s & fisk-a \\
\end{tabular}
\end{center}
\newcaption
\label{tab:42}
\end{table}

\textbf{42.}  See Table~\ref{tab:42}.  So also \textit{heimr} (home, world);
\textit{konungr} (king); \textit{Þōrr} (Thor), acc.\ \textit{Þōr},
gen.\ \textit{Þōrs}; \textit{steinn} (stone), acc.\ \textit{stein},
gen.\ \textit{steins}, pl.\ nom.\ \textit{steinar}; \textit{hrafn} (raven),
acc.\ \textit{hrafn}, pl.\ nom.\ \textit{hrafnar}; \textit{þurs} (giant),
acc.\ gen.\ \textit{þurs}, pl.\ nom.\ \textit{þursar}.

\textbf{43.} Dissyllables in \textit{-r}, \textit{-l}, \textit{-n} generally throw out the
preceding vowel before a vowel-inflection: \textit{hamarr} (hammer),
dat.\ \textit{hamri}; \textit{jǫtunn} (giant), pl.\ nom.\ \textit{jǫtnar}.
\textit{kętill} (kettle) and \textit{lykill} (key) show unmutated vowels in the
contracted forms, as in the acc.\ plur.\ \textit{katla}, \textit{lukla}.

\textbf{44.} Some nouns of this decl.\ take \textit{-ar} in the gen.\ sing.,
especially proper names, such as \textit{Hākon}, gen.\ \textit{Hākonar}.

\textbf{45.} Some nouns add \textit{v} before vowels: \textit{sær} (sea),
gen.\ \textit{sævar}.

\textbf{46.} The dat.\ sometimes drops the \textit{i}: \textit{sæ} (sea), \textit{Þōr}.
\textit{dagr} (day) mutates its vowel in the dat.\ \textit{dęgi}.

\textbf{47.} Nouns in \textit{-ir} keep the \textit{i} in the sing., and drop it in the
plur., as shown in Table~\ref{tab:47}.

\begin{table}[htbp]
\begin{center}
\begin{tabular}{rll}
	& \textbf{Singular} & \textbf{Plural} \\
	\textit{Nom.} & hęlli-r (\textit{cave}) & hęll-ar \\
	\textit{Acc.} & hęlli & hęll-a \\
	\textit{Dat.} & hęlli & hęll-um \\
	\textit{Gen.} & hęlli-s & hęll-a \\
\end{tabular}
\end{center}
\newcaption
\label{tab:47}
\end{table}

\textbf{48.} So also a number of proper names, such as \textit{Skrȳmir},
\textit{Þōrir}.  

\subsubsection{(2) \textit{i}-plurals}

\begin{table}[htbp]
\begin{center}
\begin{tabular}{rll}
	& \textbf{Singular} & \textbf{Plural} \\
	\textit{Nom.} & stað-r (\textit{place}) & stað-ir \\
	\textit{Acc.} & stað & stað-i \\
	\textit{Dat.} & stað & stǫð-um \\
	\textit{Gen.} & stað-ar & stað-a \\
\end{tabular}
\end{center}
\newcaption
\label{tab:49}
\end{table}

\textbf{49.} See Table~\ref{tab:49}.  So also \textit{gripr} (precious
thing), \textit{salr} (hall).

\textbf{50.} \textit{gęstr} (guest) takes \textit{-i} in the dat.\ sg., and \textit{-s}
in the gen.\ sg.

\textbf{51.} Those ending in \textit{g} or \textit{k} (together with some others) insert
\textit{j} before \textit{a} and \textit{u}: \textit{bękkr} (bench), \textit{bękk},
\textit{bękk}, \textit{bękkjar}; \textit{bękkir}, \textit{bękki}, \textit{bękkjum},
\textit{bękkja}.  So also \textit{męrgr} (marrow), \textit{stręngr} (string).

\subsubsection{(3) \textit{u}-plurals}

\begin{table}[htbp]
\begin{center}
\begin{tabular}{rll}
	& \textbf{Singular} & \textbf{Plural} \\
	\textit{Nom.} & skjǫld-r (\textit{shield}) & skild-ir \\
	\textit{Acc.} & skjǫld & skjǫld-u \\
	\textit{Dat.} & skild-i & skjǫld-um \\
	\textit{Gen.} & skjald-ar & skjald-a \\
\end{tabular}
\end{center}
\newcaption
\label{tab:52}
\end{table}

\textbf{52.}  See Table~\ref{tab:52}.  So also \textit{vǫndr} (twig),
\textit{vǫllr} (plain), \textit{viðr} (wood).  \textit{āss} (god) has
plur.\ nom.\ \textit{æsir}, acc.\ \textit{āsu}.  \textit{sonr} (son)
has dat.\ sg.\ \textit{syni}, plur.\ nom.\ \textit{synir}.  It regularly
drops its \textit{r} of the nom.\ in such compounds as
\textit{Tryggva-son} (son of Tryggvi).

\subsubsection{(4) \textit{r}-plurals}

\begin{table}[htbp]
\begin{center}
\begin{tabular}{rll}
	& \textbf{Singular} & \textbf{Plural} \\
	\textit{Nom.} & fōt-r (\textit{foot}) & fœt-r \\
	\textit{Acc.} & fōt & fœt-r \\
	\textit{Dat.} & fœt-i & fōt-um \\
	\textit{Gen.} & fōt-ar & fōt-a \\
\end{tabular}
\end{center}
\newcaption
\label{tab:54}
\end{table}

\textbf{54.} See Table~\ref{tab:54}.  So also \textit{brōðir} (brother),
pl.\ \textit{brœðr}.

\textbf{55.} Pres.\ participles used as nouns follow this decl.\ in the pl.,
following the weak class in the sg., as shown in Table~\ref{tab:55}.

\begin{table}[htbp]
\begin{center}
\begin{tabular}{rll}
	& \textbf{Singular} & \textbf{Plural} \\
	\textit{Nom.} & bōndi (\textit{yeoman}) & bœndr \\
	\textit{Acc.} & bōnda & bœndr \\
	\textit{Dat.} & bōnda & bōndum \\
	\textit{Gen.} & bōnda & bōnda \\
\end{tabular}
\end{center}
\newcaption
\label{tab:55}
\end{table}

\textbf{56.} So also \textit{frœndi} (kinsman), pl.\ \textit{frœndr}.

\subsection{Strong Neuters}

\begin{table}[htbp]
\begin{center}
\begin{tabular}{rll}
	& \textbf{Singular} & \textbf{Plural} \\
	\textit{Nom.} & skip (\textit{ship}) & skip \\
	\textit{Acc.} & skip & skip \\
	\textit{Dat.} & skip-i & skip-um \\
	\textit{Gen.} & skip-s & skip-a \\
\end{tabular}
\end{center}
\newcaption
\label{tab:57}
\end{table}

\textbf{57.} See Table~\ref{tab:57}.  So also \textit{orð} (word),
\textit{land} (land) pl.\ \textit{lǫnd}, \textit{sumar} (summer)
pl.\ \textit{sumur} (§ 25).

\textbf{58.} \textit{męn} (necklace), \textit{kyn} (race), \textit{grey} (dog)
insert \textit{j} before \textit{a} and \textit{u}: \textit{greyjum}.
\textit{hǫgg} (stroke) inserts \textit{v} before a vowel: \textit{hǫggvi}.
\textit{knē} (knee), \textit{knē}, \textit{knē}, \textit{knēs};
\textit{knē}, \textit{knē}, \textit{kjām}, \textit{knjā}.  So also \textit{trē}
(tree).

\textbf{59.} \textit{fē} (money) is contracted: gen.\ \textit{fjār}, dat.\ \textit{fē}.

\begin{table}[htbp]
\begin{center}
\begin{tabular}{rll}
	& \textbf{Singular} & \textbf{Plural} \\
	\textit{Nom.} & kvæði (\textit{poem}) & kvæði \\
	\textit{Acc.} & kvæði & kvæði \\
	\textit{Dat.} & kvæði & kvæðum \\
	\textit{Gen.} & kvæði-s & kvæða \\
\end{tabular}
\end{center}
\newcaption
\label{tab:60}
\end{table}

\textbf{60.} See Table~\ref{tab:60}.  So also \textit{klæði} (cloth).  Those in
\textit{k} insert \textit{j} before \textit{a} and \textit{u}: \textit{męrki}
(mark), \textit{męrkjum}, \textit{męrkja}.  So also \textit{rīki} (sovereignty).

\subsection{Strong Feminines}

\subsubsection{(1) \textit{ar}-plurals}

\begin{table}[htbp]
\begin{center}
\begin{tabular}{rll}
	& \textbf{Singular} & \textbf{Plural} \\
	\textit{Nom.} & gjǫf (\textit{gift}) & gjaf-ar \\
	\textit{Acc.} & gjǫf & gjaf-ar \\
	\textit{Dat.} & gjǫf & gjǫf-um \\
	\textit{Gen.} & gjaf-ar & gjaf-a \\
\end{tabular}
\end{center}
\newcaption
\label{tab:61}
\end{table}

\textbf{61.} See Table~\ref{tab:61}.  So also \textit{mǫn} (mane),
\textit{gjǫrð} (girdle), \textit{ār} (oar).

\textbf{62.} \textit{ā} (river) contracts: \textit{ā}, \textit{ā}, \textit{ā},
\textit{ār}; \textit{ār}, \textit{ār}, \textit{ām}, \textit{ā}.

\textbf{63.} Many take \textit{-u} in the dat.\ sg.: \textit{kęrling} (old woman),
\textit{kęrling}, \textit{kęrlingu}, \textit{kęrlingar}; \textit{kęrlingar},
\textit{kęrlingar}, \textit{kęrlingum}, \textit{kęrlinga}.  So also \textit{laug}
(bath).

\textbf{64.} Those with a mutated root-vowel (or \textit{i}) insert \textit{j} in
inflection: \textit{ey} (island), \textit{ey}, \textit{eyju}, \textit{eyjar};
\textit{eyjar}, \textit{eyjar}, \textit{eyjum}, \textit{eyja}.  So also
\textit{Frigg}, \textit{Hęl}.  \textit{mær} (maid), \textit{mey}, \textit{meyju},
\textit{meyjar}; \textit{meyjar}, \textit{meyjar}, \textit{meyjum}, \textit{meyja}.

\textbf{65.} See Table~\ref{tab:65}.

\begin{table}[htbp]
\begin{center}
\begin{tabular}{rll}
	& \textbf{Singular} & \textbf{Plural} \\
	\textit{Nom.} & heið-r (\textit{heath}) & heið-ar \\
	\textit{Acc.} & heið-i & heið-ar \\
	\textit{Dat.} & heið-i & heið-um \\
	\textit{Gen.} & heið-ar & heið-a \\
\end{tabular}
\end{center}
\newcaption
\label{tab:65}
\end{table}

\subsubsection{(2) \textit{ir}-plurals}

\begin{table}[htbp]
\begin{center}
\begin{tabular}{rll}
	& \textbf{Singular} & \textbf{Plural} \\
	\textit{Nom.} & tīð & tīð-ir \\
	\textit{Acc.} & tīð & tīð-ir \\
	\textit{Dat.} & tīð & tīð-um \\
	\textit{Gen.} & tīð-ar & tīð-a \\
\end{tabular}
\end{center}
\newcaption
\label{tab:66}
\end{table}

\textbf{66.} See Table~\ref{tab:66}.  So also \textit{sorg} (sorrow),
\textit{skipun} (arrangement), \textit{hǫfn} (harbour) pl.\ \textit{hafnir},
and the majority of strong feminines.

\textbf{67.} Many have \textit{-u} in the dat.\ sg.: \textit{sōl} (sun), \textit{sōl},
\textit{sōlu}, \textit{sōlar}; \textit{sōlir}, \textit{sōlir}, \textit{sōlum},
\textit{sōla}.  So also \textit{jǫrð} (earth), \textit{stund} (period of time).

\textbf{68.} One noun has \textit{r} in the nom.\ sg., following \textit{heiðr} in the
sg.: \textit{brūðr} (bride), \textit{brūði}, \textit{brūði}, \textit{brūðar};
\textit{brūðir}, \textit{brūðir}, \textit{brūðum}, \textit{brūða}.

\subsubsection{(3) \textit{r}-plurals}

\begin{table}[htbp]
\begin{center}
\begin{tabular}{rll}
	& \textbf{Singular} & \textbf{Plural} \\
	\textit{Nom.} & bōk (\textit{book}) & bœk-r \\
	\textit{Acc.} & bōk & bœk-r \\
	\textit{Dat.} & bōk & bōk-um \\
	\textit{Gen.} & bōk-ar & bōk-a \\
\end{tabular}
\end{center}
\newcaption
\label{tab:69}
\end{table}

\textbf{69.} See Table~\ref{tab:69}.  So also \textit{nātt} (night)
pl.\ \textit{nætr}, \textit{bōt} (compensation) pl.\ \textit{bœtr},
\textit{tǫnn} (tooth) gen.\ \textit{tannar} pl.\ \textit{tęnnr}.

\textbf{70.} \textit{hǫnd} (hand) pl.\ \textit{hęndr} has dat.\ sg.\ \textit{hęndi}.

\textbf{71.} \textit{kȳr} (cow) has acc.\ \textit{kū}, pl.\ \textit{kȳr}.

\textbf{72.} \textit{brūn} (eyebrow) assimilates the \textit{r} of the pl.:
\textit{brȳnn}.

\begin{table}[htbp]
\begin{center}
\begin{tabular}{rll}
	& \textbf{Singular} & \textbf{Plural} \\
	\textit{Nom.} & mōðir (\textit{mother}) & mœðr \\
	\textit{Acc.} & mōður & mœðr \\
	\textit{Dat.} & mōður & mœðrum \\
	\textit{Gen.} & mōður & mœðra \\
\end{tabular}
\end{center}
\newcaption
\label{tab:73}
\end{table}

\textbf{73.}  See Table~\ref{tab:73}.  So also \textit{dōttir} (daughter)
pl.\ \textit{dœtr}; \textit{systir} (sister) pl.\ \textit{systr}.

\subsection{Weak Masculines}

\begin{table}[htbp]
\begin{center}
\begin{tabular}{rll}
	& \textbf{Singular} & \textbf{Plural} \\
	\textit{Nom.} & bog-i (\textit{bow}) & bog-ar \\
	\textit{Acc.} & bog-a & bog-a \\
	\textit{Dat.} & bog-a & bog-um \\
	\textit{Gen.} & bog-a & bog-a \\
\end{tabular}
\end{center}
\newcaption
\label{tab:74}
\end{table}

\textbf{74.} See Table~\ref{tab:74}.  So also \textit{māni} (moon),
\textit{fēlagi} (companion).

\textbf{75.} \textit{hǫfðingi} (chief) and some others insert \textit{j} in inflection:
\textit{hǫfðingja}, \textit{hǫfðingjar}, \textit{hǫfðingjum}.

\textbf{76.} \textit{lē} (scythe) is contracted; its gen.\ sg.\ is \textit{ljā}.

\textbf{77.} \textit{oxi} (ox) has pl.\ \textit{öxn}.

\textbf{78.} \textit{herra} (lord) is indeclinable in the sg.

\subsection{Weak Neuters}

\begin{table}[htbp]
\begin{center}
\begin{tabular}{rll}
	& \textbf{Singular} & \textbf{Plural} \\
	\textit{Nom.} & hjart-a (\textit{heart}) & hjǫrt-u \\
	\textit{Acc.} & hjart-a & hjǫrt-u \\
	\textit{Dat.} & hjart-a & hjǫrt-um \\
	\textit{Gen.} & hjart-a & hjart-na \\
\end{tabular}
\end{center}
\newcaption
\label{tab:79}
\end{table}

\textbf{79.} See Table~\ref{tab:79}.  So also \textit{auga} (eye).

\subsection{Weak Feminines}

\begin{table}[htbp]
\begin{center}
\begin{tabular}{rll}
	& \textbf{Singular} & \textbf{Plural} \\
	\textit{Nom.} & tung-a (\textit{tongue}) & tung-ur \\
	\textit{Acc.} & tung-u & tung-ur \\
	\textit{Dat.} & tung-u & tung-um \\
	\textit{Gen.} & tung-u & tung-na \\
\end{tabular}
\end{center}
\newcaption
\label{tab:80}
\end{table}

\textbf{80.} See Table~\ref{tab:80}.  So also \textit{stjarna} (star)
pl.\ \textit{stjǫrnur}, \textit{kirkja} (church), gen.\ plurals
\textit{stjarna}, \textit{kirkna}.

\begin{table}[htbp]
\begin{center}
\begin{tabular}{rl}
	\textit{Sg.\ Nom.} & ęlli (\textit{old age}) \\
	\textit{Acc.} & ęlli \\
	\textit{Dat.} & ęlli \\
	\textit{Gen.} & ęlli \\
\end{tabular}
\end{center}
\newcaption
\label{tab:81}
\end{table}

\textbf{81.} See Table~\ref{tab:81}.  So also \textit{glęði} (joy) and
many abstract nouns.

\textbf{82.} \textit{lygi} (falsehood) has pl.\ \textit{lygar}; so also \textit{gǫ̈rsimi}
(precious thing).

\section{Adjectives}

\textbf{83.} Adjectives have three genders, and the same cases as nouns,
though with partly different endings, together with strong and
weak forms.

\subsection{Strong Adjectives}

\begin{table}[htbp]
\begin{center}
\begin{tabular}{rlll}
	& \textbf{Masc.} & \textbf{Neut.} & \textbf{Fem.} \\
	\textit{Sg.\ Nom.} & ung-r (\textit{young}) & ung-t & ung \\
	\textit{Acc.} & ung-an & ung-t & ung-a \\
	\textit{Dat.} & ung-um & ung-u & ung-ri \\
	\textit{Gen.} & ung-s & ung-s & ung-rar \\
	\\
	\textit{Pl.\ Nom.} & ung-ir & ung & ung-ar \\
	\textit{Acc.} & ung-a & ung & ung-ar \\
	\textit{Dat.} & ung-um & ung-um & ung-um \\
	\textit{Gen.} & ung-ra & ung-ra & ung-ra \\
\end{tabular}
\end{center}
\newcaption
\label{tab:84}
\end{table}

\textbf{84.} See Table~\ref{tab:84}.  So also \textit{fagr} (fair),
fem.\ \textit{fǫgr}, neut.\ \textit{fagrt}.

\textbf{85.} Some insert \textit{j} before \textit{a} and \textit{u}: \textit{nȳr} (new),
\textit{nȳjum}, \textit{nȳjan}.

\textbf{86.} Some insert \textit{v} before a vowel: \textit{hār} (high), \textit{hāvan},
\textit{dökkr} (dark), \textit{dökkvir}, \textit{kykr} (alive), \textit{kykvir}.

\textbf{87.} The \textit{t} of the neut.\ is doubled after a long vowel: \textit{nȳtt},
\textit{hātt}.  Monosyllables in \textit{ð}, \textit{dd}, \textit{tt} form their neut.\ in
\textit{-tt}: \textit{breiðr} (broad), \textit{breitt}; \textit{leiddr} (led), \textit{leitt}.
\textit{gōðr} (good) has neut.\ \textit{gott}.  \textit{sannr} (true) has neut.\ \textit{satt}.
In unaccented syllables or if a cons.\ precedes, \textit{tt} is shortened
to \textit{t}: \textit{kallaðr} (called), \textit{kallat}; \textit{blindr} (blind),
\textit{blint}, \textit{harðr} (hard), \textit{hart}, \textit{fastr} (firm),
\textit{fast}.

\textbf{88.} \textit{l} and \textit{n} assimilate a following \textit{r}: \textit{gamall}
(old), fem.\ \textit{gǫmul}, fem.\ acc.\ \textit{gamla}, dat.\ \textit{gamalli}.
\textit{vǣnn} (beautiful), gen.\ pl.\ \textit{vænna}.

\begin{table}[htbp]
\begin{center}
\begin{tabular}{rlll}
	& \textbf{Masc.} & \textbf{Neut.} & \textbf{Fem.} \\
	\textit{Sg.\ Nom.} & mikill (\textit{great}) & mikit & mikil \\
	\textit{Acc.} & mikinn & mikit & mikla \\
	\textit{Dat.} & miklum & miklu & mikilli \\
	\textit{Gen.} & mikils & mikils & mikillar \\
	\\
	\textit{Pl.\ Nom.} & miklir & mikil & miklar \\
	\textit{Acc.} & mikla & mikil & miklar \\
	\textit{Dat.} & miklum & miklum & miklum \\
	\textit{Gen.} & mikilla & mikilla & mikilla \\
\end{tabular}
\end{center}
\newcaption
\label{tab:89}
\end{table}

\textbf{89.} See Table~\ref{tab:89}.  So also \textit{lītill} (little).

\textbf{90.} Dissyllables in \textit{-inn} have \textit{-it} in the neut., and
\textit{-inn} in the masc.\ sg.\ acc.: \textit{tīginn} (distinguished),
\textit{tīgit}, \textit{tīginn}, pl.\ \textit{tīgnīr}.  So also \textit{kominn}
(come).

\textbf{91.}  See Table~\ref{tab:91}.

\begin{table}[htbp]
\begin{center}
\begin{tabular}{rlll}
	& \textbf{Masc.} & \textbf{Neut.} & \textbf{Fem.} \\
	\textit{Sg.\ Nom.} & annarr (\textit{other}) & annat & ǫnnur \\
	\textit{Acc.} & annan & annat & aðra \\
	\textit{Dat.} & ǫðrum & ǫðru & annarri \\
	\textit{Gen.} & annars & annars & annarrar \\
	\\
	\textit{Pl.\ Nom.} & aðrir & ǫnnur & aðrar \\
	\textit{Acc.} & aðra & ǫnnur & aðrar \\
	\textit{Dat.} & ǫðrum & ǫðrum & ǫðrum \\
	\textit{Gen.} & annarra & annarra & annarra \\
\end{tabular}
\end{center}
\newcaption
\label{tab:91}
\end{table}

\subsection{Weak Adjectives}

\begin{table}[htbp]
\begin{center}
\begin{tabular}{rlll}
	& \textbf{Masc.} & \textbf{Neut.} & \textbf{Fem.} \\
	\textit{Sg.\ Nom.} & ung-i & ung-a & ung-a \\
	\textit{Acc.} & ung-a & ung-a & ung-u \\
	\textit{Dat.} & ung-a & ung-a & ung-u \\
	\textit{Gen.} & ung-a & ung-a & ung-u \\
	\\
	\textit{Pl.\ Nom.} & ung-u & ung-u & ung-u \\
	\textit{Acc.} & ung-u & ung-u & ung-u \\
	\textit{Dat.} & ung-u & ung-u & ung-u \\
	\textit{Gen.} & ung-u & ung-u & ung-u \\
\end{tabular}
\end{center}
\newcaption
\label{tab:92}
\end{table}

\textbf{92.}  See Table~\ref{tab:92}.  So also \textit{fagri},
\textit{hāvi}, \textit{mikli}, etc.

\begin{table}[htbp]
\begin{center}
\begin{tabular}{rlll}
	& \textbf{Masc.} & \textbf{Neut.} & \textbf{Fem.} \\
	\textit{Sg.\ Nom.} & yngri (\textit{younger}) & yngra & yngri \\
	\textit{Acc.} & yngra & yngra & yngri \\
	\textit{Dat.} & yngra & yngra & yngri \\
	\textit{Gen.} & yngra & yngra & yngri \\
	\\
	\textit{Pl.\ Nom.} & yngri & yngri & yngri \\
	\textit{Acc.} & yngri & yngri & yngri \\
	\textit{Dat.} & yngrum & yngrum & yngrum \\
	\textit{Gen.} & yngri & yngri & yngri \\
\end{tabular}
\end{center}
\newcaption
\label{tab:93}
\end{table}

\textbf{93.}  See Table~\ref{tab:93}.  So also all comparatives, such as
\textit{meiri} (greater), and pres.\ partic.\ when used as adjectives,
such as \textit{gefandi} (giving), dat.\ pl.\ \textit{gefǫndum}.

\section{Comparison}

\textbf{94.} (1) with \textit{-ari}, \textit{-astr}: \textit{rīkr} (powerful),
\textit{rīkari}, \textit{rīkastr}; \textit{gǫfugr} (distinguished),
\textit{gǫfgari}, \textit{gǫfgastr}.

\textbf{95.} (2) with \textit{-ri}, \textit{-str} and mutation: \textit{langr}
(long), \textit{lęngri}, \textit{lęngstr}; \textit{stōrr} (big), \textit{stœrri},
\textit{stœrstr}; \textit{ungr} (young), \textit{yngri}, \textit{yngstr}.

\textbf{96.} The adjectives in Table~\ref{tab:96} are irregular.

\begin{table}[htbp]
\begin{center}
\begin{tabular}{lll}
	gamall (\textit{old}) & ęllri & ęlztr \\
	gōðr (\textit{good}) & bętri & bęztr \\
	illr (\textit{bad}) & vęrri & vęrstr \\
	lītill (\textit{little}) & minni & minstr \\
	margr (\textit{many}) & fleiri & flestr \\
	mikill (\textit{great}) & meiri & mestr \\
\end{tabular}
\end{center}
\newcaption
\label{tab:96}
\end{table}

\section{Numerals}

\textbf{97.}  See Table~\ref{tab:97_1} and Table~\ref{tab:97_2}.

\begin{table}[htbp]
\begin{center}
\begin{tabular}{rll}
	& \textbf{Cardinal} & \textbf{Ordinal} \\
	1. & einn (\textit{one}) & fyrstr (\textit{first}) \\
	2. & tveir & annarr \\
	3. & þrīr & þriði \\
	4. & fjōrir & fjōrði \\
	5. & fimm & fimmti \\
	6. & sex & sētti \\
	7. & sjau & sjaundi \\
	8. & ātta & ātti \\
	9. & nīu & nīundi \\
	10. & tīu & tīundi \\
	11. & ellifu & ellifti \\
	12. & tōlf & tōlfti \\
	13. & þrettān & þrettāndi \\
\end{tabular}
\end{center}
\newcaption
\label{tab:97_1}
\end{table}

\begin{table}[htbp]
\begin{center}
\begin{tabular}{rl}
	& \textbf{Cardinal} \\
	14. & fjōrtān \\
	15. & fimmtān \\
	16. & sextān \\
	17. & sjautān \\
	18. & ātjān \\
	19. & nītjān \\
	20. & tuttugu \\
	21. & einn ok tuttugu, etc. \\
	30. & þrīr tigir, etc. \\
	100. & tīu tigir \\
	110. & ellifu tigir \\
	120. & hundrað \\
	1200. & þūsund \\
\end{tabular}
\end{center}
\newcaption
\label{tab:97_2}
\end{table}

\textbf{98.}  \textit{einn} is declined like other adjectives.  See
Table~\ref{tab:98}.

\begin{table}[htbp]
\begin{center}
\begin{tabular}{rlll}
    & \textbf{Masc.} & \textbf{Neut.} & \textbf{Fem.} \\
    \textit{Nom.} & einn & eitt & ein \\
    \textit{Acc.} & einn & eitt & eina \\
    \textit{Dat.} & einum & einu & einni \\
    \textit{Gen.} & eins & eins & einnar \\
\end{tabular}
\end{center}
\newcaption
\label{tab:98}
\end{table}

It also has a pl.\ \textit{einir}, \textit{einar}, \textit{ein}; gen.
\textit{einna}, etc.\ in the sense of `some.'

\textbf{99.} The next three show various irregularities.  See
Table~\ref{tab:99}.

\begin{table}[htbp]
\begin{center}
\begin{tabular}{rlll}
    & \textbf{Masc.} & \textbf{Neut.} & \textbf{Fem.} \\
    \textit{Nom.} & tveir & tvau & tvær \\
    \textit{Acc.} & tvā & tvau & tvær \\
    \textit{Dat.} & tveim & tveim & tveim \\
    \textit{Gen.} & tvęggja & tvęggja & tvęggja \\
\end{tabular}
\end{center}
\newcaption
\label{tab:99}
\end{table}

\textbf{100.}  Similarly \textit{bāðir} (both).  See Table~\ref{tab:100}.

\begin{table}[htbp]
\begin{center}
\begin{tabular}{rlll}
    & \textbf{Masc.} & \textbf{Neut.} & \textbf{Fem.} \\
    \textit{Nom.} & bāðir & bæði & bāðar \\
    \textit{Acc.} & bāða & bæði & bāðar \\
    \textit{Dat.} & bāðum & bāðum & bāðum \\
    \textit{Gen.} & bęggja & bęggja & bęggja \\
\end{tabular}
\end{center}
\newcaption
\label{tab:100}
\end{table}

\textbf{101.}  See Table~\ref{tab:101}.

\begin{table}[htbp]
\begin{center}
\begin{tabular}{rlll}
    & \textbf{Masc.} & \textbf{Neut.} & \textbf{Fem.} \\
    \textit{Nom.} & þrīr & þrjū & þrjār \\
    \textit{Acc.} & þrjā & þrjū & þrjār \\
    \textit{Dat.} & þrim & þrim & þrim \\
    \textit{Gen.} & þriggja & þriggja & þriggja \\
\end{tabular}
\end{center}
\newcaption
\label{tab:101}
\end{table}

\textbf{102.} See Table~\ref{tab:102}.

\begin{table}[htbp]
\begin{center}
\begin{tabular}{rlll}
    & \textbf{Masc.} & \textbf{Neut.} & \textbf{Fem.} \\
    \textit{Nom.} & fjōrir & fjogur & fjōrar \\
    \textit{Acc.} & fjōra & fjogur & fjōrar \\
    \textit{Dat.} & fjōrum & fjōrum & fjōrum \\
    \textit{Gen.} & fjogurra & fjogurra & fjogurra \\
\end{tabular}
\end{center}
\newcaption
\label{tab:102}
\end{table}

\textbf{103.} The others are indeclinable up to \textit{þrīr tigir}, etc.; the
\textit{tigir} being declined regularly as a plural strong \textit{u}-masculine
\textit{tigir}, \textit{tigu}, \textit{tigum}, \textit{tiga}.

\textbf{104.} \textit{hundrað} is a strong neut.: \textit{tvau hundruð} (240), \textit{tveim
hundruðum}, etc.  It governs the gen.\ (as also does \textit{þūsund}):
\textit{fimm hundruð gōlfa}, `five (six) hundred chambers.'

\textbf{105.} \textit{þūsund} is a strong \textit{ir}-feminine: \textit{tvær þūsundir}
(2400).

\textbf{106.} \textit{hundrað} and \textit{þūsund} are rarely = 100 and 1000.

\textbf{107.} Of the ordinals \textit{fyrstr} and \textit{annarr} (§ 91) are strong, the
others weak adjectives.  \textit{þriði} inserts a \textit{j}: \textit{þriðja}, etc.

\section{Pronouns}

\subsection{Personal}

\textbf{108.}  See Table~\ref{tab:108_1} and Table~\ref{tab:108_2}.

\begin{table}[htbp]
\begin{center}
\begin{tabular}{rlll}
    \textit{Sg.\ Nom.} & ek (\textit{I}) & þū (\textit{thou}) & -- \\
    \textit{Acc.} & mik & þik & sik (\textit{oneself}) \\
    \textit{Dat.} & mēr & þēr & sēr \\
    \textit{Gen.} & mīn & þīn & sīn \\
    \\
    \textit{Dual Nom.} & vit & it & -- \\
    \textit{Acc.} & okkr & ykkr & sik \\
    \textit{Dat.} & okkr & ykkr & sēr \\
    \textit{Gen.} & okkar & ykkar & sīn \\
    \\
    \textit{Pl.\ Nom.} & vēr (\textit{we}) & þēr (\textit{ye}) & -- \\
    \textit{Acc.} & oss & yðr & sik (\textit{oneselves}) \\
    \textit{Dat.} & oss & yðr & sēr \\
    \textit{Gen.} & vār & yðar & sīn \\
\end{tabular}
\end{center}
\newcaption
\label{tab:108_1}
\end{table}

\begin{table}[htbp]
\begin{center}
\begin{tabular}{rlll}
	& \textbf{Masc.} & \textbf{Neut.} & \textbf{Fem.} \\
	\textit{Sg.\ Nom.} & hann (\textit{he}) & þat (\textit{it}) & hon (\textit{she}) \\
	\textit{Acc.} & hann & þat & hana \\
	\textit{Dat.} & honum & þvī & hęnni \\
	\textit{Gen.} & hans & þess & hęnnar \\
	\\
	\textit{Pl.\ Nom.} & þeir (\textit{they}) & þau & þær \\
	\textit{Acc.} & þā & þau & þær \\
	\textit{Dat.} & þeim & þeim & þeim \\
	\textit{Gen.} & þeira & þeira & þeira \\
\end{tabular}
\end{center}
\newcaption
\label{tab:108_2}
\end{table}

\textbf{109.} \textit{ek} was often suffixed to its verb, especially in poetry,
being sometimes added twice over: \textit{mætta-k} (I might), \textit{sā-k-a-k}
(I saw not; \textit{a}=`not').  So also \textit{þū}: \textit{er-tu} (art thou),
\textit{skalt-u} (shalt thou) = \textit{*skalt-tu}.

\subsection{Possessive}

\begin{table}[htbp]
\begin{center}
\begin{tabular}{rlll}
	& \textbf{Masc.} & \textbf{Neut.} & \textbf{Fem.} \\
	\textit{Sg.\ Nom.} & minn (\textit{my}) & mitt & mīn \\
	\textit{Acc.} & minn & mitt & mīna \\
	\textit{Dat.} & mīnum & mīnu & minni \\
	\textit{Gen.} & mīns & mīns & minnar \\
	\\
	\textit{Pl.\ Nom.} & mīnir & mīn & mīnar \\
	\textit{Acc.} & mīna & mīn & mīnar \\
	\textit{Dat.} & mīnum & mīnum & mīnum \\
	\textit{Gen.} & minna & minna & minna \\
\end{tabular}
\end{center}
\newcaption
\label{tab:110}
\end{table}

\textbf{110.} See Table~\ref{tab:110}.  So also \textit{þinn} (thy),
\textit{sinn} (his, etc., reflexive).

\textbf{111.} \textit{vārr}, \textit{vārt}, \textit{vār} (our) is regular: acc.\ masc.\
\textit{vārn}, masc.\ plur.\ \textit{vārir}, \textit{vāra}, \textit{vārum},
\textit{vārra}, etc.

\begin{table}[htbp]
\begin{center}
\begin{tabular}{rlll}
	& \textbf{Masc.} & \textbf{Neut.} & \textbf{Fem.} \\
	\textit{Sg.\ Nom.} & yðarr (\textit{your}) & yðart & yður \\
	\textit{Acc.} & yðarn & yðart & yðra \\
	\textit{Dat.} & yðrum & yðru & yðarri \\
	\textit{Gen.} & yðars & yðars & yðarrar \\
	\\
	\textit{Pl.\ Nom.} & yðrir & yður & yðrar \\
	\textit{Acc.} & yðra & yður & yðrar \\
	\textit{Dat.} & yðrum & yðrum & yðrum \\
	\textit{Gen.} & yðarra & yðarra & yðarra \\
\end{tabular}
\end{center}
\newcaption
\label{tab:112}
\end{table}

\textbf{112.} See Table~\ref{tab:112}.  So also \textit{okkarr} (our two)
and \textit{ykkarr} (your two).

\textbf{113.} \textit{hans} (his), \textit{þess} (its), \textit{hęnnar} (her), and
\textit{þeira} (their) are indeclinable.

\subsection{Demonstrative}

\textbf{114.} See Table~\ref{tab:114}.

\begin{table}[htbp]
\begin{center}
\begin{tabular}{rlll}
	& \textbf{Masc.} & \textbf{Neut.} & \textbf{Fem.} \\
	\textit{Sg.\ Nom.} & sā (\textit{that}) & þat & sū \\
	\textit{Acc.} & þann & þat & þā \\
	\textit{Dat.} & þeim & þvī & þeiri \\
	\textit{Gen.} & þess & þess & þeirar \\
	\\
	\textit{Pl.\ Nom.} & þeir & þau & þær \\
	\textit{Acc.} & þā & þau & þær \\
	\textit{Dat.} & þeim & þeim & þeim \\
	\textit{Gen.} & þeira & þeira & þeira \\
\end{tabular}
\end{center}
\newcaption
\label{tab:114}
\end{table}

\textbf{115.} \textit{hinn}, \textit{hitt}, \textit{hin} (that) is inflected like
\textit{minn} (except that its vowel is short throughout): acc.\ masc.\ 
\textit{hinn}, plur.\ masc.  \textit{hinir}, \textit{hina}, \textit{hinum},
\textit{hinna}.

\textbf{116.} See Table~\ref{tab:116}.

\begin{table}[htbp]
\begin{center}
\begin{tabular}{rlll}
	& \textbf{Masc.} & \textbf{Neut.} & \textbf{Fem.} \\
	\textit{Sg.\ Nom.} & þessi (\textit{this}) & þetta & þessi \\
	\textit{Acc.} & þenna & þetta & þessa \\
	\textit{Dat.} & þessum & þessu & þessi \\
	\textit{Gen.} & þessa & þessa & þessar \\
	\\
	\textit{Pl.\ Nom.} & þessir & þessi & þessar \\
	\textit{Acc.} & þessa & þessi & þessar \\
	\textit{Dat.} & þessum & þessum & þessum \\
	\textit{Gen.} & þessa & þessa & þessa \\
\end{tabular}
\end{center}
\newcaption
\label{tab:116}
\end{table}

\subsection{Definite}

\textbf{117.}  The prefixed definite article is declined as shown in
Table~\ref{tab:117}.

\begin{table}[htbp]
\begin{center}
\begin{tabular}{rlll}
	& \textbf{Masc.} & \textbf{Neut.} & \textbf{Fem.} \\
	\textit{Sg.\ Nom.} & inn & it & in \\
	\textit{Acc.} & inn & it & ina \\
	\textit{Dat.} & inum & inu & inni \\
	\textit{Gen.} & ins & ins & innar \\
	\\
	\textit{Pl.\ Nom.} & inir & in & inar \\
	\textit{Acc.} & ina & in & inar \\
	\textit{Dat.} & inum & inum & inum \\
	\textit{Gen.} & inna & inna & inna \\
\end{tabular}
\end{center}
\newcaption
\label{tab:117}
\end{table}

\textbf{118.} When suffixed to its noun it undergoes various changes.  In
its monosyllabic forms it drops its vowel after a short
(un-accented) vowel, as in \textit{auga-t} (the eye), but keeps it after
a long vowel, as in \textit{ā-in} (the river), \textit{trē-it} (the tree).
The dissyllabic forms drop their initial vowel almost everywhere;
not, however, after the \textit{-ar}, \textit{-r}, of the gen.\ sg., nor in
\textit{męnninir} (men, \textit{nom.}), \textit{męnn-ina} (men, \textit{acc.}).
The \textit{-m} of the dat.\ pl.\ is dropped before the suffixed \textit{-num}.
See Table~\ref{tab:118_1} and Table~\ref{tab:118_2}.

\begin{table}[htbp]
\begin{center}
\begin{tabular}{rlll}
	& \textbf{Masc.} & \textbf{Neut.} & \textbf{Fem.} \\
	\textit{Sg.\ Nom.} & fiskr-inn & skip-it & gjǫf-in \\
	\textit{Acc.} & fisk-inn & skip-it & gjǫf-ina \\
	\textit{Dat.} & fiski-num & skipi-nu & gjǫf-inni \\
	\textit{Gen.} & fisks-ins & skips-ins & gjafar-innar \\
	\\
	\textit{Pl.\ Nom.} & fiskar-nir & skip-in & gjafar-nar\\
	\textit{Acc.} & fiska-na & skip-in & gjafar-nar \\
	\textit{Dat.} & fisku-num & skipu-num & gjǫfu-num \\
	\textit{Gen.} & fiska-nna & skipa-nna & gjafa-nna \\
\end{tabular}
\end{center}
\newcaption
\label{tab:118_1}
\end{table}
		
\begin{table}[htbp]
\begin{center}
\begin{tabular}{rlll}
	& \textbf{Masc.} & \textbf{Neut.} & \textbf{Fem.} \\
	\textit{Sg.\ Nom.} & bogi-nn & auga-t & tunga-n \\
	\textit{Acc.} & boga-nn & auga-t & tungu-na \\
	\textit{Dat.} & boga-num & auga-nu & tungu-nni \\
	\textit{Gen.} & boga-ns & auga-ns & tungu-nnar \\
	\\
	\textit{Pl.\ Nom.} & bogar-nir & augu-n & tungur-nar \\
	\textit{Acc.} & boga-na & augu-n & tungur-nar \\
	\textit{Dat.} & bogu-num & augu-num & tungnu-num \\
	\textit{Gen.} & boga-nna & augna-nna & tungna-nna \\
\end{tabular}
\end{center}
\newcaption
\label{tab:118_2}
\end{table}

\subsection{Relative}

\textbf{119.} The ordinary relative pron.\ is the indeclinable \textit{er}, often
preceded by \textit{sā}: \textit{sā er} = he who, who, \textit{sū er} who fem.

\subsection{Interrogative}

\textbf{120.} The neut.\ \textit{hvat} has gen.\ \textit{hvess}, dat.\ \textit{hvī}, which
last is chiefly used as an adverb = `why.'

\textbf{121.} See Table~\ref{tab:121}.

\begin{table}[htbp]
\begin{center}
\begin{tabular}{rlll}
	& \textbf{Masc.} & \textbf{Neut.} & \textbf{Fem.} \\
	\textit{Sg.\ Nom.} & hvārr (\textit{which of two}) & hvārt & hvār \\
	\textit{Acc.} & hvārn & hvārt & hvāra \\
	\textit{Dat.} & hvārum & hvāru & hvārri \\
	\textit{Gen.} & hvārs & hvārs & hvārrar \\
	\\
	\textit{Pl.\ Nom.} & hvārir & hvār & hvārar \\
	\textit{Acc.} & hvāra & hvār & hvārar \\
	\textit{Dat.} & hvārum & hvārum & hvārum \\
	\textit{Gen.} & hvārra & hvārra & hvārra \\
\end{tabular}
\end{center}
\newcaption
\label{tab:121}
\end{table}

\textbf{122.} See Table~\ref{tab:122}.

\begin{table}[htbp]
\begin{center}
\begin{tabular}{rlll}
	& \textbf{Masc.} & \textbf{Neut.} & \textbf{Fem.} \\
	\textit{Sg.\ Nom.} & hvęrr (\textit{which, who}) & hvęrt & hvęr \\
	\textit{Acc.} & hvęrn & hvęrt & hvęrja \\
	\textit{Dat.} & hvęrjum & hvęrju & hvęrri \\
	\textit{Gen.} & hvęrs & hvęrs & hvęrrar \\
	\\
	\textit{Pl.\ Nom.} & hvęrir & hvęr & hvęrjar \\
	\textit{Acc.} & hvęrja & hvęr & hvęrjar \\
	\textit{Dat.} & hvęrjum & hvęrjum & hvęrjum \\
	\textit{Gen.} & hvęrra & hvęrra & hvęrra \\
\end{tabular}
\end{center}
\newcaption
\label{tab:122}
\end{table}

\subsection{Indefinite}

\textbf{123.} \textit{einn-hvęrr}, \textit{eitthvęrt}, \textit{einhvęr} (some one)
keeps an invariable \textit{ein-} in the other cases, the second element
being inflected as above.

\textbf{124.} \textit{sumr} (some) is declined like an ordinary adjective.

\textbf{125.} See Table~\ref{tab:125}.

\begin{table}[htbp]
\begin{center}
\begin{tabular}{rlll}
	& \textbf{Masc.} & \textbf{Neut.} & \textbf{Fem.} \\
	\textit{Sg.\ Nom.} & nakkvarr (\textit{some}) & nakkvat & nǫkkur \\
	\textit{Acc.} & nakkvarn & nakkvat & nakkvara \\
	\textit{Dat.} & nǫkkurum & nǫkkuru & nakkvarri \\
	\textit{Gen.} & nakkvars & nakkvars & nakkvarrar \\
	\\
	\textit{Pl.\ Nom.} & nakkvarir & nǫkkur & nakkvarar \\
	\textit{Acc.} & nakkvara & nǫkkur & nakkvarar \\
	\textit{Dat.} & nǫkkurum & nǫkkurum & nǫkkurum \\
	\textit{Gen.} & nakkvarra & nakkvarra & nakkvarra \\
\end{tabular}
\end{center}
\newcaption
\label{tab:125}
\end{table}

\textbf{126.} See Table~\ref{tab:126}.

\begin{table}[htbp]
\begin{center}
\begin{tabular}{rlll}
	& \textbf{Masc.} & \textbf{Neut.} & \textbf{Fem.} \\
	\textit{Sg.\ Nom.} & engi (\textit{none, no}) & ekki & engi \\
	\textit{Acc.} & engan & ekki & enga \\
	\textit{Dat.} & engum & engu & engri \\
	\textit{Gen.} & engis & engis & engrar \\
	\\
	\textit{Pl.\ Nom.} & engir & engi & engar \\
	\textit{Acc.} & enga & engi & engar \\
	\textit{Dat.} & engum & engum & engum \\
	\textit{Gen.} & engra & engra & engra \\
\end{tabular}
\end{center}
\newcaption
\label{tab:126}
\end{table}

\textbf{127.} In \textit{hvār-tvęggja} (each of the two, both) the first element
is declined as above, the second is left unchanged.

\section{Verbs}

\textbf{128.} There are two classes of verbs, \textit{strong} and \textit{weak}.
Strong verbs are conjugated partly by means of gradation, weak verbs
by adding \textit{ð} (\textit{d}, \textit{t}).

\textbf{129.} The \textit{ð} of the 2 pl.\ is dropt before \textit{þit} (ye two)
and \textit{þēr} (ye): \textit{gefi þēr}, \textit{gāfu þit}.

\textbf{130.} There is a middle voice, which ends in \textit{-mk} in the 1
pers.\ sg.\ and pl., the rest of the verb being formed by adding \textit{sk} to
the active endings, \textit{r} being dropt, the resulting \textit{ts},
\textit{ðs} being written \textit{z} (§ 36): \textit{kvezk} (active \textit{kveðr}
`says'), \textit{þu fekkzk} (\textit{fekkt} `gottest').

\textbf{131.} The following is the conjugation of the strong verb \textit{gefa}
(give), which will show those endings which are common to all
verbs:

\subsection{Active}

See Table~\ref{tab:131_1}.

\begin{table}[htbp]
\begin{center}
\begin{tabular}{rlll}
	& \textbf{Indicative} & \textbf{Subjunctive} \\
	\textit{Present sg.} 1. & gef & gef-a \\
	2. & gef-r & gef-ir \\
	3. & gef-r & gef-i \\
	\\
	\textit{pl.} 1. & gef-um & gef-im \\
	2. & gef-ið & gef-ið \\
	3. & gef-a & gef-i \\
	\\
	\textit{Preterite sg.} 1. & gaf & gæf-a \\
	2. & gaf-t & gæf-ir \\
	3. & gaf & gæf-i \\
	\\
	\textit{pl.} 1. & gāf-um & gæf-im \\
	2. & gāf-uð & gæf-ið \\
	3. & gāf-u & gæf-i \\
\end{tabular}
\end{center}

\begin{center}
\begin{minipage}{3in}
\textit{Imperative sg.} 2 gef; \textit{pl.} 1 gef-um, 2 gef-ið.

\textit{Participle pres.} gef-andi; \textit{pret.} gef-inn.

\textit{Infin.} gefa.
\end{minipage}
\end{center}
\newcaption
\label{tab:131_1}
\end{table}

\subsection{Middle}

See Table~\ref{tab:131_2}.

\begin{table}[htbp]
\begin{center}
\begin{tabular}{rlll}
    & \textbf{Indicative} & \textbf{Subjunctive} \\
    \textit{Present sg.} 1. & gef-umk & gef-umk \\
    2. & gef-sk & gef-isk \\
    3. & gef-sk & gef-isk \\
    \\
    \textit{pl.} 1. & gef-umk & gef-imk \\
    2. & gef-izk & gef-izk \\
    3. & gef-ask & gef-isk \\
    \\
    \textit{Preterite sg.} 1. & gāf-umk & gæf-umk \\
    2. & gaf-zk & gæf-isk \\
    3. & gaf-sk & gæf-isk \\
    \\
    \textit{pl.} 1. & gāf-umk & gæf-imk \\
    2. & gāf-uzk & gæf-izk \\
    3. & gāf-usk & gæf-isk \\
\end{tabular}
\end{center}

\begin{center}
\begin{minipage}{3in}
\textit{Impers.\ sg.} 2 gef-sk; \textit{pl.} 1 gef-umk, 2 gef-izk.

\textit{Partic.\ pres.} gef-andisk; \textit{pret.} gef-izk \textit{neut.}

\textit{Infin.} gef-ask.
\end{minipage}
\end{center}
\newcaption
\label{tab:131_2}
\end{table}

\subsection{Strong Verbs}

\textbf{132.} In the strong verbs the plur.\ of the pret.\ indic.\ generally
has a different vowel from that of the sing.  The 1 sg.\ pret.\ of
the middle voice always has the vowel of the pl.\ pret.: \textit{gāfumk}.
The pret.\ subj.\ has the vowel of the pret.\ indic.\ plur.\ mutated:
\textit{skaut} (he shot), \textit{skutu} (they shot), \textit{skyti} (he
might shoot).  But there is no mutation in verbs of the first conj.:
\textit{hljōpi}, inf.\ \textit{hlaupa} (leap).

\textbf{133.} The pres.\ indic.\ sing.\ mutates the root-vowel in all three
persons: \textit{ek skȳt}, \textit{þū skȳtr}, \textit{hann skȳtr},
infin.\ \textit{skjōta} (shoot).  \textit{e} however is not mutated: \textit{ek gef},
\textit{þū gefr}.  The inflectional \textit{r} is liable to the same
modifications as the \textit{r} of nouns (§ 32): \textit{skīnn}, \textit{vęx},
infin.\ \textit{skīna} (shine), \textit{vaxa} (grow).

\textbf{134.} Verbs in \textit{ld} change the \textit{d} into \textit{t} in the 1, 3
sg.\ pret.\ indic.\ and in the imper.\ sg.: \textit{helt} (held), \textit{halt}
(hold!), infin.\ \textit{halda}.  \textit{nd} becomes \textit{tt}, and \textit{ng}
becomes \textit{kk} under the same conditions: \textit{binda} (bind),
\textit{ganga} (go), pret.\ \textit{batt}, \textit{gekk}, imper.\ \textit{bitt},
\textit{gakk}.

\textbf{135.} The \textit{t} of the 2 sg.\ pret.\ indic.\ is doubled after a long
accented vowel: \textit{þū sātt} (thou sawest).  If the 1 sg.\ pret.\ indic.\ ends
in \textit{t} or \textit{ð}, the 2 sg.\ ends in \textit{zt}:
\textit{lēt} (I let), \textit{þū lēzt}, \textit{bauð} (I offered) \textit{þū
bauzt}.

\textbf{136.} There are seven conjugations of strong verbs, distinguished
mainly by the characteristic vowels of their preterites.

\subsubsection{I.\ \textbf{`Fall'}-conjugation}

\textbf{137.} See Table~\ref{tab:137}.

\begin{table}[htbp]
\begin{center}
\begin{tabular}{lllll}
	\textbf{Infin.} & \textbf{Third Pres.} & \textbf{Prt.\ Sing.} & \textbf{Prt.\ Pl.} & \textbf{Ptc.\ Prt.} \\
	falla (\textit{fall}) & fęllr & fell & fellu & fallinn \\
	lāta (\textit{let}) & lætr & lēt & lētu & lātinn \\
	rāða (\textit{advise}) & ræðr & rēð & rēðu & rāðinn \\
	heita (\textit{call}) & heitr & hēt & hētu & heitinn \\
	halda (\textit{hold}) & hęldr & helt & heldu & haldinn \\
	ganga (\textit{go}) & gęngr & gekk & gengu & gęnginn \\
	fā (\textit{get}) & fær & fekk & fengu & fęnginn \\
	auka (\textit{increase}) & eykr & jōk & jōku & aukinn \\
	būa (\textit{dwell}) & bȳr & bjō & bjoggu & būinn \\
	hǫggva (\textit{hew}) & hǫggr & hjō & hjoggu & hǫggvinn \\
	hlaupa (\textit{leap}) & hleypr & hljōp & hljōpu & hlaupinn \\
\end{tabular}
\end{center}
\newcaption
\label{tab:137}
\end{table}

\textbf{138.} The verbs in Table~\ref{tab:138} have weak preterites in
\textit{r}.

\begin{table}[htbp]
\begin{center}
\begin{tabular}{lllll}
	\textbf{Infin.} & \textbf{Third Pres.} & \textbf{Prt.\ Sing.} & \textbf{Prt.\ Pl.} & \textbf{Ptc.\ Prt.} \\
	grōa (\textit{grow}) & grœr & gröri & gröru & grōinn \\
	rōa (\textit{row}) & rœr & röri & röru & rōinn \\
	snūa (\textit{twist}) & snȳr & snöri & snöru & snūinn \\
\end{tabular}
\end{center}
\newcaption
\label{tab:138}
\end{table}

\textbf{139.} \textit{heita} in the passive sense of `to be named, called' has a
weak present: \textit{ek heiti}, \textit{þū heitir}.

\subsubsection{II.\ \textbf{`Shake'}-conjugation}

\textbf{140.} See Table~\ref{tab:140}.

\begin{table}[htbp]
\begin{center}
\begin{tabular}{lllll}
	\textbf{Infin.} & \textbf{Third Pres.} & \textbf{Prt.\ Sing.} & \textbf{Prt.\ Pl.} & \textbf{Ptc.\ Prt.} \\
	fara (\textit{go}) & fęrr & fōr & fōru & farinn \\
    grafa (\textit{dig}) & gręfr & grōf & grōfu & grafinn \\
    hlaða (\textit{load}) & hlęðr & hlōð & hlōðu & hlaðinn \\
    vaxa (\textit{grow}) & vęx & ōx & ōxu & vaxinn \\
    standa (\textit{stand}) & stęndr & stōð & stōðu & staðinn \\
    aka (\textit{drive}) & ękr & ōk & ōku & ękinn \\
    taka (\textit{take}) & tękr & tōk & tōku & tękinn \\
    draga (\textit{draw}) & dręgr & drō & drōgu & dręginn \\
    flā (\textit{flay}) & flær & flō & flōgu & flęginn \\
    slā (\textit{strike}) & slær & slō & slōgu & slęginn \\
\end{tabular}
\end{center}
\newcaption
\label{tab:140}
\end{table}

\textbf{141.} The verbs in Table~\ref{tab:141} have weak presents.

\begin{table}[htbp]
\begin{center}
\begin{tabular}{lllll}
	\textbf{Infin.} & \textbf{Third Pres.} & \textbf{Prt.\ Sing.} & \textbf{Prt.\ Pl.} & \textbf{Ptc.\ Prt.} \\
	hęfja (\textit{lift}) & hęfr & hōf & hōfu & hafinn \\
    deyja (\textit{die}) & deyr & dō & dō & dāinn \\
    hlæja (\textit{laugh}) & hlær & hlō & hlōgu & hlęginn \\
\end{tabular}
\end{center}
\newcaption
\label{tab:141}
\end{table}

\subsubsection{III.\ \textbf{`Bind'}-conjugation}

\textbf{142.} See Table~\ref{tab:142}.

\begin{table}[htbp]
\begin{center}
\begin{tabular}{lllll}
	\textbf{Infin.} & \textbf{Third Pres.} & \textbf{Prt.\ Sing.} & \textbf{Prt.\ Pl.} & \textbf{Ptc.\ Prt.} \\
	bresta (\textit{burst}) & brestr & brast & brustu & brostinn \\
    hverfa (\textit{turn}) & hverfr & hvarf & hurfu & horfinn \\
    svelga (\textit{swallow}) & svelgr & svalg & sulgu & sōlginn \\
    verða (\textit{become}) & verðr & varð & urðu & orðinn \\
    skjālfa (\textit{shake}) & skelfr & skalf & skulfu & skolfinn \\
    drekka (\textit{drink}) & drekkr & drakk & drukku & drukkinn \\
    finna (\textit{find}) & finnr & fann & fundu & fundinn \\
    vinna (\textit{win}) & vinnr & vann & unnu & unninn \\
    binda (\textit{bind}) & bindr & batt & bundu & bundinn \\
    springa (\textit{spring}) & springr & sprakk & sprungu & sprunginn \\
    stinga (\textit{pierce}) & stingr & stakk & stungu & stunginn \\
    bregða (\textit{pull}) & bregðr & brā & brugðu & brugðinn \\
    sökkva (\textit{sink}) & sökkr & sǫkk & sukku & sokkinn \\
    stökkva (\textit{spring}) & stökkr & stǫkk & stukku & stokkinn \\
\end{tabular}
\end{center}
\newcaption
\label{tab:142}
\end{table}

\textbf{143.} The verbs in Table~\ref{tab:143} have weak presents (which
makes however no difference in their conjugation).

\begin{table}[htbp]
\begin{center}
\begin{tabular}{lllll}
	\textbf{Infin.} & \textbf{Third Pres.} & \textbf{Prt.\ Sing.} & \textbf{Prt.\ Pl.} & \textbf{Ptc.\ Prt.} \\
	bręnna (\textit{burn}) & bręnnr & brann & brunnu & brunninn \\
    ręnna (\textit{run}) & ręnnr & rann & runnu & runninn \\
\end{tabular}
\end{center}
\newcaption
\label{tab:143}
\end{table}

\subsubsection{IV.\ \textbf{`Bear'}-conjugation}

\textbf{144.} See Table~\ref{tab:144}.

\begin{table}[htbp]
\begin{center}
\begin{tabular}{lllll}
	\textbf{Infin.} & \textbf{Third Pres.} & \textbf{Prt.\ Sing.} & \textbf{Prt.\ Pl.} & \textbf{Ptc.\ Prt.} \\
	bera (\textit{carry}) & berr & bar & bāru & borinn \\
    nema (\textit{take}) & nemr & nam & nāmu & numinn \\
    fela (\textit{hide}) & felr & fal & fālu & fōlginn \\
    koma (\textit{come}) & kömr & kom & kvāmu & kominn \\
    sofa (\textit{sleep}) & söfr & svaf & svāfu & sofinn \\
\end{tabular}
\end{center}
\newcaption
\label{tab:144}
\end{table}

\subsubsection{V.\ \textbf{`Give'}-conjugation}

\textbf{145.} See Table~\ref{tab:145}.

\begin{table}[htbp]
\begin{center}
\begin{tabular}{lllll}
	\textbf{Infin.} & \textbf{Third Pres.} & \textbf{Prt.\ Sing.} & \textbf{Prt.\ Pl.} & \textbf{Ptc.\ Prt.} \\
	drepa (\textit{kill}) & drepr & drap & drāpu & drepinn \\
    gefa (\textit{give}) & gefr & gaf & gāfu & gefinn \\
    kveða (\textit{say}) & kveðr & kvað & kvāðu & kveðinn \\
    meta (\textit{estimate}) & metr & mat & mātu & metinn \\
    reka (\textit{drive}) & rekr & rak & rāku & rekinn \\
    eta (\textit{eat}) & etr & āt & ātu & etinn \\
	sjā (\textit{see}) & sēr\footnote{sē, sēr, sēr; sjām, sēð, sjā.  Subj.\ sē, sēr, sē; sēm, sēð, sē.} & sā & sā\footnote{sām, sāið, sā.} & sēnn \\
\end{tabular}
\end{center}
\newcaption
\label{tab:145}
\end{table}

\textbf{146.} The verbs in Table~\ref{tab:146} have weak presents.

\begin{table}[htbp]
\begin{center}
\begin{tabular}{lllll}
	\textbf{Infin.} & \textbf{Third Pres.} & \textbf{Prt.\ Sing.} & \textbf{Prt.\ Pl.} & \textbf{Ptc.\ Prt.} \\
	biðja (\textit{ash}) & biðr & bað & bāðu & beðinn \\
    sitja (\textit{sit}) & sitr & sat & sātu & setinn \\
    liggja (\textit{lie}) & liggr & lā & lāgum & leginn \\
    þiggja (\textit{receive}) & þiggr & þā & þāgu & þeginn \\
\end{tabular}
\end{center}
\newcaption
\label{tab:146}
\end{table}

\subsubsection{VI.\ \textbf{`Shine'}-conjugation}

\textbf{147.} See Table~\ref{tab:147}.

\begin{table}[htbp]
\begin{center}
\begin{tabular}{lllll}
	\textbf{Infin.} & \textbf{Third Pres.} & \textbf{Prt.\ Sing.} & \textbf{Prt.\ Pl.} & \textbf{Ptc.\ Prt.} \\
	bīta (\textit{bite}) & bītr & beit & bitu & bitinn \\
    drīfa (\textit{drive}) & drīfr & dreif & drifu & drifinn \\
    grīpa (\textit{grasp}) & grīpr & greip & gripu & gripinn \\
    līða (\textit{go}) & līðr & leið & liðu & liðinn \\
    līta (\textit{look}) & lītr & leit & litu & litinn \\
    rīða (\textit{ride}) & rīðr & reið & riðu & riðinn \\
    sīga (\textit{sink}) & sīgr & seig & sigu & siginn \\
    slīta (\textit{tear}) & slītr & sleit & slitu & slitinn \\
    stīga (\textit{advance}) & stīgr & steig & stigu & stiginn \\
    bīða (\textit{wait}) & bīðr & beið & biðu & beiðnn \\
\end{tabular}
\end{center}
\newcaption
\label{tab:147}
\end{table}

\textbf{148.} Vīkja (see Table~\ref{tab:148}) has a weak present.

\begin{table}[htbp]
\begin{center}
\begin{tabular}{lllll}
	\textbf{Infin.} & \textbf{Third Pres.} & \textbf{Prt.\ Sing.} & \textbf{Prt.\ Pl.} & \textbf{Ptc.\ Prt.} \\
	vīkja (\textit{move}) & vīkr & veik & viku & vikinn \\
\end{tabular}
\end{center}
\newcaption
\label{tab:148}
\end{table}

\subsubsection{VII.\ \textbf{`Choose'}-conjugation}

\textbf{149.} See Table~\ref{tab:149}.

\begin{table}[htbp]
\begin{center}
\begin{tabular}{lllll}
	\textbf{Infin.} & \textbf{Third Pres.} & \textbf{Prt.\ Sing.} & \textbf{Prt.\ Pl.} & \textbf{Ptc.\ Prt.} \\
	bjōða (\textit{offer}) & bȳðr & bauð & buðu & boðinn \\
    brjōta (\textit{break}) & brȳtr & braut & brutu & brotinn \\
    fljōta (\textit{float}) & flȳtr & flaut & flutu & flotinn \\
    hljōta (\textit{receive}) & hlȳtr & hlaut & hlutu & hlotinn \\
    kjōsa (\textit{choose}) & kȳss & kaus & kusum & kosinn \\
    njōta (\textit{enjoy}) & nȳtr & naut & nutu & notinn \\
    skjōta (\textit{shoot}) & skȳtr & skaut & skutu & skotinn \\
    drjūpa (\textit{drop}) & drȳpr & draup & drupu & dropinn \\
    ljūga (\textit{tell lies}) & lȳgr & laug & lugu & loginn \\
    lūka (\textit{close}) & lȳkr & lauk & luku & lokinn \\
    lūta (\textit{bend}) & lȳtr & laut & lutu & lotinn \\
    fljūga (\textit{fly}) & flȳgr & flō & flugu & floginn \\
\end{tabular}
\end{center}
\newcaption
\label{tab:149}
\end{table}

\subsection{Weak Verbs}

\textbf{150.} There are three conjugations of weak verbs.  All those of
the first conjugation have mutated vowels in the pres., and form
their pret.\ with \textit{ð} (\textit{d}, \textit{t}): \textit{heyra} (hear),
\textit{heyrða}.  Those of the second form their pret.\ in the same
way, but have unmutated vowels in the pres.: \textit{hafa} (have)
\textit{hafða}.  Those of the third form their pret.\ in \textit{-aða}:
\textit{kalla} (call), \textit{kallaða}.

\subsubsection{I.\ \textbf{`Hear'}-conjugation}

\textbf{151.}  See Table~\ref{tab:151_1} (active) and Table~\ref{tab:151_2}
(middle).

\begin{table}[htbp]
\begin{center}
\begin{tabular}{rlll}
    & \textbf{Indicative} & \textbf{Subjunctive} \\
    \textit{Present sg.} 1. & heyr-i & heyr-a \\
    2. & heyr-ir & heyr-ir \\
    3. & heyr-ir & heyr-i \\
    \\
    \textit{pl.} 1. & heyr-um & heyr-im \\
    2. & heyr-ið & heyr-ið \\
    3. & heyr-a & heyr-i \\
    \\
    \textit{Preterite sg.} 1. & heyr-ða & heyr-ða \\
    2. & heyr-ðir & heyr-ðir \\
    3. & heyr-ði & heyr-ði \\
    \\
    \textit{pl.} 1. & heyr-ðum & heyr-ðim \\
    2. & heyr-ðuð & heyr-ðið \\
    3. & heyr-ðu & heyr-ði \\
\end{tabular}
\end{center}

\begin{center}
\begin{minipage}{3in}
\textit{Imper.\ sg.} 1.\ heyr; \textit{pl.} 1.\ heyr-um, 2.\ heyr-ið.

\textit{Partic.\ pres.} heyr-andi; \textit{pret.} heyr-ðr.

\textit{Infin.} heyr-a.
\end{minipage}
\end{center}
\newcaption
\label{tab:151_1}
\end{table}

\begin{table}[htbp]
\begin{center}
\begin{tabular}{rlll}
    & \textbf{Indicative} & \textbf{Subjunctive} \\
    \textit{Present sg.} 1. & heyr-umk & heyr-umk \\
    2. & heyr-isk & heyr-isk \\
    3. & heyr-isk & heyr-isk \\
    \\
    \textit{pl.} 1. & heyr-umk & heyr-imk \\
    2. & heyr-izk & heyr-izk \\
    3. & heyr-ask & heyr-isk \\
    \\
    \textit{Preterite sg.} 1. & heyr-ðumk & heyr-ðumk \\
    2. & heyr-ðisk & heyr-ðisk \\
    3. & heyr-ðisk & heyr-ðisk \\
    \\
    \textit{pl.} 1. & heyr-ðumk & heyr-ðimk \\
    2. & heyr-ðuzk & heyr-ðizk \\
    3. & heyr-ðusk & heyr-ðisk \\
\end{tabular}
\end{center}

\begin{center}
\begin{minipage}{3.5in}
\textit{Imper.\ sg.} 2.\ heyr-sk; \textit{pl.} 1.\ heyr-umk, 2.\ heyr-izk.

\textit{Partic.\ pres.} heyr-andisk; \textit{pret.} heyr-zk \textit{neut.}

\textit{Infin.} heyr-ask.
\end{minipage}
\end{center}
\newcaption
\label{tab:151_2}
\end{table}

\begin{center}
\textit{A.\ Without vowel-change}
\end{center}

\textbf{152.} The inflectional \textit{ð} becomes \textit{d} after long syllables ending
in \textit{l} or \textit{n}: \textit{sigla} (sail), \textit{siglda}; \textit{nęfna}
(name), \textit{nęfnda}, \textit{nęfndr}.

\textbf{153.} \textit{-ðð} becomes \textit{dd}: \textit{leiða} (lead), \textit{leidda}.

\textbf{154.} \textit{ð} after \textit{s} and \textit{t} becomes \textit{t}: \textit{reisa}
(raise), \textit{reista}; \textit{mœta} (meet), \textit{mœtta}.  Also in a few
verbs in \textit{l}, \textit{n}: \textit{mæla} (speak), \textit{mælta};
\textit{spęnna} (buckle), \textit{spęnta}.

\textbf{155.} After \textit{nd} and \textit{pt} it is dropped: \textit{sęnda} (send),
\textit{sęnda}, \textit{sęndr}; \textit{lypta} (lift), \textit{lypta}.

\textbf{156.} It is preserved in such verbs as the following: \textit{dœma}
(judge), \textit{dœmða}; \textit{fœra} (lead), \textit{fœrða}; \textit{hęrða}
(harden), \textit{hęrða}; \textit{hleypa} (gallop), \textit{hleypða}.

\begin{center}
\textit{B.\ With vowel-change}
\end{center}

\textbf{157.} All these verbs have \textit{j} preceded by a short syllable
(tęlja), or a long vowel without any cons.\ after it (dȳja), or
\textit{gg} (lęggja); the \textit{j} being kept before \textit{a} and
\textit{u}, as in the pres.\ ind.\ of \textit{spyrja} (ask): \textit{spyr},
\textit{spyrr}, \textit{spyrr}; \textit{spyrjum}, \textit{spyrið},
\textit{spyrja}, pres.\ subj.\ 1 sg.\ \textit{ek spyrja}; they unmutate
their vowel in the pret.\ and ptc.\ pret.\ (spurða, spurðr), the
mutation being restored in the pret.\ subj.\ \textit{spyrða},
\textit{spyrðir}, etc.  The ptc.\ pret.\ often has an \textit{i} before the
\textit{ð}.  See Table~\ref{tab:157}.

\begin{table}[htbp]
\begin{center}
\begin{tabular}{lll}
	bęrja (\textit{strike}) & barða & barðr \\
	lęggja (\textit{lay}) & lagða & lag(i)ðr \\
	tęlja (\textit{tell}) & talða & tal(i)ðr \\
	vękja (\textit{wake}) & vakða & vakðr \\
	flytja (\textit{remove}) & flutta & fluttr \\
	dȳja (\textit{shake}) & dūða & dūðr \\
\end{tabular}
\end{center}
\newcaption
\label{tab:157}
\end{table}

\textbf{158.} The verbs in Table~\ref{tab:158} keep the mutated vowel
throughout.

\begin{table}[htbp]
\begin{center}
\begin{tabular}{lll}
	sęlja (\textit{sell}) & sęlda & sęldr \\
	sętja (\textit{set}) & sętta & sęttr \\
\end{tabular}
\end{center}
\newcaption
\label{tab:158}
\end{table}

\begin{center}
\textit{C.}
\end{center}

\textbf{159.} The verbs in Table~\ref{tab:159} are irregular.

\begin{table}[htbp]
\begin{center}
\begin{tabular}{lll}
	sœkja (\textit{seek}) & sōtta & sōttr \\
	þykkja (\textit{seem}) & þōtta & þōttr \\
\end{tabular}
\end{center}

\begin{center}
\begin{minipage}{3in}
Subj.\ pret.\ \textit{sœtta}, \textit{þœtta}.
\end{minipage}
\end{center}
\newcaption
\label{tab:159}
\end{table}


\textbf{160.} The verb in Table~\ref{tab:160} has an adj.\ for its
partic.\ pret.

\begin{table}[htbp]
\begin{center}
\begin{tabular}{lll}
	gǫ̈ra (\textit{make}) & gǫ̈rða & gǫ̈rr. \\
\end{tabular}
\end{center}
\newcaption
\label{tab:160}
\end{table}

\subsubsection{II.\ \textbf{`Have'}-conjugation}

\textbf{161.} The few verbs of this class are conjugated like those of
conj.\ I, except that some of them have imperatives in \textit{-i}:
\textit{vaki}, \textit{þęfi}; \textit{uni}.  \textit{lifa}, \textit{sęgja}
have imper.\ \textit{lif}, \textit{sęg}.  They mutate the vowel of the
pret.\ subj.\ (ynða).  Their partic.\ pret.\ generally occurs only in
the neut.; sometimes the \textit{a} is dropped.  See Table~\ref{tab:161}.

\begin{table}[htbp]
\begin{center}
\begin{tabular}{llll}
	lifa (\textit{live}) & lifi & lifða & lifat \\
	una (\textit{be contented}) & uni & unða & unat \\
	skorta (\textit{be wanting}) & skorti & skorta & skort \\
	þola (\textit{endure}) & þoli & þolða & þolat \\
	þora (\textit{dare}) & þori & þorða & þorat \\
	nā (\textit{attain}) & nāi & nāða & nāðr, nāit \\
\end{tabular}
\end{center}
\newcaption
\label{tab:161}
\end{table}

\textbf{162.} The verbs in Table~\ref{tab:162} show mutation.

\begin{table}[htbp]
\begin{center}
\begin{tabular}{llll}
	sęgja (\textit{say}) & sęgi & sagða & sagðr \\
	þęgja (\textit{be silent}) & þęgi & þagða & þagat \\
	hafa (\textit{have}) & hęfi & hafða & hafðr \\
	kaupa (\textit{buy}) & kaupi & keypta & keyptr \\
\end{tabular}
\end{center}
\newcaption
\label{tab:162}
\end{table}

\textbf{163.} The present indic.\ of the first three is shown in
Table~\ref{tab:163}.

\begin{table}[htbp]
\begin{center}
\begin{tabular}{rlll}
	\textit{Sing.} 1. & hęfi & sęgi & þęgi \\
	2, 3. & hęfir & sęgir & þęgir \\
	\\
	\textit{Plur.} 1. & hǫfum & sęgjum & þęgjum \\
	2. & hafið & sęgið & þęgið \\
	3. & hafa & sęgja & þęgja \\
\end{tabular}
\end{center}
\newcaption
\label{tab:163}
\end{table}

\textbf{164.} The rest of \textit{hafa} is regular.  Pres.\ subj.\ \textit{hafa},
\textit{hafir}, \textit{hafi}; \textit{hafim}, \textit{hafið}, \textit{hafi}.
Pret.\ indic.\ \textit{hafða}, \textit{hafðir}, \textit{hafði}; \textit{hǫfdum},
\textit{hǫfðuð}, \textit{hǫfðu}.  Pret.\ subj.\ \textit{hęfða}, \textit{hęfðir},
\textit{hęfði}; \textit{hęfðim}, \textit{hęfðið}, \textit{hęfði}.
Imper.\ \textit{haf}, \textit{hǫfum}, \textit{hafið}.  Ptc.\ \textit{hafandi},
\textit{hafðr}.

\subsubsection{III.\ \textbf{`Call'}-conjugation}

See Table~\ref{tab:164_1} (active) and Table~\ref{tab:164_2} (middle).

\begin{table}[htbp]
\begin{center}
\begin{tabular}{rlll}
    & \textbf{Indicative} & \textbf{Subjunctive} \\
    \textit{Present sg.} 1. & kall-a & kall-a \\
    2. & kall-ar & kall-ir \\
    3. & kall-ar & kall-i \\
    \\
    \textit{pl.} 1. & kǫll-um & kall-im \\
    2. & kall-ið & kall-ið \\
    3. & kall-a & kall-i \\
    \\
    \textit{Preterite sg.} 1. & kall-aða & kall-aða \\
    2. & kall-aðir & kall-aðir \\
    3. & kall-aði & kall-aði \\
    \\
    \textit{pl.} 1. & kǫll-uðum & kall-aðim \\
    2. & kǫll-uðuð & kall-aðið \\
    3. & kǫll-uðu & kall-aði \\
\end{tabular}
\end{center}

\begin{center}
\begin{minipage}{3.5in}
\textit{Imper.\ sing.} 2.\ kall-a; \textit{plur.} 1.\ kǫll-um, 2.\ kall-ið.

\textit{Partic.\ pres.} kall-andi; \textit{pret.} kallaðr (\textit{neut.} kallat).

\textit{Infin.} kalla.
\end{minipage}
\end{center}
\newcaption
\label{tab:164_1}
\end{table}

\begin{table}[htbp]
\begin{center}
\begin{tabular}{rlll}
    & \textbf{Indicative} & \textbf{Subjunctive} \\
    \textit{Present sg.} 1. & kǫll-umk & kǫll-umk \\
    2. & kall-ask & kall-isk \\
    3. & kall-ask & kall-isk \\
    \\
    \textit{pl.} 1. & kǫll-umk & kall-imk \\
    2. & kall-izk & kall-izk \\
    3. & kall-ask & kall-isk \\
    \\
    \textit{Preterite sg.} 1. & kǫll-uðumk & kǫll-uðumk \\
    2. & kall-aðisk & kall-aðisk \\
    3. & kall-aðisk & kall-aðisk \\
    \\
    \textit{pl.} 1. & kǫll-uðumk & kall-aðimk \\
    2. & kǫll-uðuzk & kall-aðizk \\
    3. & kǫll-uðusk & kall-aðisk \\
\end{tabular}
\end{center}

\begin{center}
\begin{minipage}{3.5in}
\textit{Imper.\ sing.} 2.\ kall-ask; \textit{pl.} 1.\ kǫll-umk, 2.\ kall-izk.

\textit{Partic.\ pres.} kall-andisk; \textit{pret.} kall-azk \textit{neut.}

\textit{Infin.} kall-ask.
\end{minipage}
\end{center}
\newcaption
\label{tab:164_2}
\end{table}


\textbf{165.} So also \textit{byrja} (begin), \textit{hęrja} (make war), \textit{vakna}
(awake).

\subsection{Strong-Weak Verbs}

\textbf{166.} The verbs in Table~\ref{tab:166} have old strong preterites
for their presents, from which new weak preterites are formed.

\begin{table}[htbp]
\begin{center}
\begin{tabular}{lllll}
	\textbf{Infin.} & \textbf{Third Pres.} & \textbf{Prt.\ Sing.} & \textbf{Prt.\ Pl.} & \textbf{Ptc.\ Prt.} \\
	eiga (\textit{possess}) & ā & eigu & ātta & āttr \\
    kunna (\textit{can}) & kann & kunnu & kunna & kunnat \textit{n.} \\
    mega (\textit{can}) & mā & megu & mātta & mātt \textit{n.} \\
    muna (\textit{remember}) & man & munu & munða & munat \textit{n.} \\
    munu (\textit{will}) & mun & munu & munða & --- \\
    skulu (\textit{shall}) & skal & skulu & skylda & skyldr \\
    þurfa (\textit{need}) & þarf & þurfu & þurfta & þurft \textit{n.} \\
    unna (\textit{love}) & ann & unnu & unna & unnt \textit{n.} \\
    vita (\textit{know}) & veit & vitu & vissa & vitaðr \\
\end{tabular}
\end{center}
\newcaption
\label{tab:166}
\end{table}

\textbf{167.} Of these verbs \textit{munu} and \textit{skulu} have preterite infinitives:
\textit{mundu}, \textit{skyldu}.

\subsection{Anomalous Verbs}

\textbf{168.} \textit{Vilja} (will) in the present is shown in
Table~\ref{tab:168}.

\begin{table}[htbp]
\begin{center}
\begin{tabular}{lllll}
	& \textbf{Sing.} & \textbf{Plur.} \\
	1. & vil & viljum \\
	2. & vill & vilið \\
	3. & vill & vilja \\
\end{tabular}
\end{center}

\begin{center}
\begin{minipage}{3.5in}
\noindent\textit{Subj.\ pres.} vili.  \textit{Pret.\ ind.} vilda.  \textit{Ptc.\ prt.} viljat.
\end{minipage}
\end{center}
\newcaption
\label{tab:168}
\end{table}


\textbf{169.} \textit{Vera} (be) is shown in Table~\ref{tab:169}.

\begin{table}[htbp]
\begin{center}
\begin{tabular}{rlll}
    & \textbf{Indicative} & \textbf{Subjunctive} \\
    \textit{Present sg.} 1. & em & sē \\
    2. & ert & sēr \\
    3. & er & sē \\
    \\
    \textit{pl.} 1. & erum & sēm \\
    2. & eruð & sēð \\
    3. & eru & sē \\
    \\
    \textit{Preterite sg.} 1. & var & væra \\
    2. & vart & værir \\
    3. & var & væri \\
    \\
    \textit{pl.} 1. & vārum & værim \\
    2. & vāruð & værið \\
    3. & vāru & væri \\
\end{tabular}
\end{center}

\begin{center}
\begin{minipage}{3in}
\noindent\textit{Imper.\ sg.} ver; \textit{pl.} verið.  \textit{Ptc.\ prt.} verit \textit{n.}
\end{minipage}
\end{center}
\newcaption
\label{tab:169}
\end{table}


\emptypage

\chapter{Composition}

\textbf{170.} Composition with the genitive is very frequent in Icelandic.
Thus by the side of \textit{skip-stjōrn} (ship-steering) we find
\textit{skips-brot} (ship's breaking, shipwreck), \textit{skipa-hęrr} (army of
ships, fleet).  Genitival composition often expresses possession,
as in \textit{konungs-skip} (king's ship).

\section{Derivation}

\subsection{Prefixes}

\textbf{171.} Prefixes are much less used in Icelandic than in Old
English.

\gap\textbf{al-} `quite,' `very': \textit{al-būinn} `quite ready,' \textit{al-snotr}
`very clever.'

\gap\textbf{all-} `all,' `very': \textit{all-valdr} `all-ruler, monarch,'
\textit{all-harðr} `very hard,' \textit{all-stōrum} `very greatly.'

\gap\textbf{and-} `against': \textit{and-lit} `countenance' (\textit{lita}, look),
\textit{and-svar} `answer.'

\gap\textbf{fjǫl-} `many': \textit{fjǫl-męnni} `multitude' (\textit{maðr}, man).

\gap\textbf{mis-} `mis-': \textit{mis-līka} `displease.'

\gap\textbf{ū-} `un-': \textit{ū-friðr} `war' (\textit{friðr}, peace), \textit{ū-happ}
`misfortune' (\textit{happ} luck).

\subsection{Endings}

\subsubsection{(a) Nouns}

\begin{center}
\textit{Personal}
\end{center}

\textbf{172.} \textbf{-ingr, -ingi, -ing}: \textit{vīkingr} `pirate,' \textit{hǫfðingi}
`chief,' \textit{kęrling} `old woman.'

\begin{center}
\textit{Abstract}
\end{center}

\textbf{173.} \textbf{-ð}, fem.\ with mutation: \textit{fęgrð} `beauty' (\textit{fagr},
fair), \textit{fęrð} `journey' (\textit{fara}, go), \textit{lęngð} `length'
(\textit{langr}, long).

\gap\textbf{-ing}, fem.: \textit{svipting} `pulling,' \textit{vīking} `piracy,'
\textit{virðing} `honour.'

\gap\textbf{-leikr}, masc.: \textit{kœr-leikr} `affection' (\textit{hœrr}, dear),
\textit{skjōt-leikr} `speed' (\textit{skjōtr}, swift).

\gap\textbf{-an, -un}, fem.: \textit{skipan} `arrangement,' \textit{skęmtun}
`amusement.'

\subsubsection{(b) Adjectives}

\textbf{174.} \textbf{-ugr}: \textit{rāðugr} `sagacious,' \textit{þrūðugr} `strong.'

\gap\textbf{-ōttr}: \textit{kollōttr} `bald,' \textit{ǫndōttr} `fierce.'

\gap\textbf{-lauss} `-less': \textit{fē-lauss} `moneyless,' \textit{ōtta-lauss}
`without fear.'

\gap\textbf{-ligr} `-ly': \textit{undr-ligr} `wonderful,' \textit{sann-ligr}
`probable' (\textit{sannr}, true).

\gap\textbf{-samr}: \textit{līkn-samr} `gracious,' \textit{skyn-samr} `intelligent.'

\gap\textbf{-verðr} `-ward': \textit{ofan-verðr} `upper.'

\subsubsection{(c) Verbs}

\textbf{175.} \textbf{-na}: \textit{brotna} `be broken' (\textit{brotinn}, broken),
\textit{hvītna} `become white,' \textit{vakna} `awake.'  Used to form
intransitive and inchoative verbs of the third conj.

\subsubsection{(d) Adverbs}

\textbf{176.} \textbf{-liga} `-ly': \textit{undar-liga} `wonderfully,' \textit{stęrk-liga}
`strongly' (\textit{stęrkr}, strong).

\gap\textbf{-um}, dat.\ pl.: \textit{stōrum} `greatly' (\textit{stōrr}, great).

\emptypage

\chapter{Syntax}

\textbf{177.} Icelandic syntax greatly resembles Old English, but has
several peculiarities of its own.

\section{Concord}

\textbf{178.} Concord is carried out very strictly in Icelandic: \textit{allir
męnn vāru būnir} `all the men were ready,' \textit{allir vāru drepnir}
`all were killed.'

\textbf{179.} A plural adj.\ or pronoun referring to two nouns of different
(natural or grammatical) gender is always put in the neuter: \textit{þā
gekk hann upp, ok með honum Loki} (masc.), \textit{ok Þjālfi} (masc.),
\textit{ok Rǫskva} (fem.).  \textit{þā er þau} (neut.) \textit{hǫfðu lītla hrið
gęngit\ldots} `he landed, and with him L., and Þ., and R\@.  When they
had walked for some time\ldots'

\section{Cases}

\textbf{180.} The extensive use of the instrumental dative is very
characteristic of Icelandic: whenever the direct object of a verb
can be considered as the instrument of the action expressed by
the verb, it is put in the dative, as in \textit{kasta spjōti} `throw a
spear' (lit.\ `throw \textit{with} a spear'), \textit{hann helt hamarskaptinu}
`he grasped the handle of the hammer,' \textit{heita þvī} `promise
that,' \textit{jāta þvī} `agree to that.'

\section{Adjectives}

\textbf{181.} The weak form of adjectives is used as in O.E.\ after the
definite article, \textit{þessi} and other demonstratives.  \textit{annarr}
(other) is always strong.

\textbf{182.}, An adj.\ is often set in apposition to a following noun to
denote part of it: \textit{eiga hālft dȳrit} `to have half of the
animal,' \textit{ǫnnur þau} `the rest of them,' \textit{of miðja nātt} `in the
middle of the night.'

\subsection{Pronouns}

\textbf{183.} \textit{sā} is often put pleonastically before the definite article
\textit{inn}, both before and after the subst.: \textit{sā inn ungi maðr} `that
young man,' \textit{hafit þat it djūpa} `the deep sea.'

\textbf{184.} The definite article is generally not expressed at all, or
else \textit{einn}, \textit{einnhvęrr} is used.

\textbf{185.} A noun (often a proper name) is often put in apposition to a
dual pron.\ of the first and second persons, or a plur.\ of the
third person: \textit{þit fēlagar}, `thou and thy companions,' \textit{með
þeim Āka} `with him and Āki.'  Similarly \textit{stęndr Þōrr upp ok þeir
fēlagar} `Thor and his companions get up.'

\textbf{186.} The plurals \textit{vēr}, \textit{þēr} are sometimes used instead of the
singulars \textit{ek}, \textit{þū}, especially when a king is speaking or being
spoken to.

\textbf{187.} \textit{sik} and \textit{sēr} are used in a strictly reflexive sense,
referring back to the subject of the sentence, like \textit{se} in
Latin: \textit{Þorr bauð honum til matar með sēr} `Thor asked him to
supper with him.'

\subsection{Verbs}

\textbf{188.} The tenses for which there is no inflection in the active,
and all those of the passive, are formed by the auxiliaries
\textit{skal} (shall), \textit{hafa} (have), \textit{vera} (be) with the
infin.\ and ptc.\ pret., much as in modern English.

\textbf{189.} The historical present is much used, often alternating
abruptly with the preterite.

\textbf{190.} The middle voice is used: (1) in a purely reflexive sense:
\textit{spara} `spare,' \textit{sparask} `spare oneself, reserve one's
strength.'  (2) intransitively: \textit{būa} `prepare,' \textit{būask} `become
ready, be ready'; \textit{sętja} `set,' \textit{sętjask} `sit down';
\textit{sȳna} `show,' \textit{sȳnask} `appear, seem.'  (3) reciprocally:
\textit{bęrja} `strike,' \textit{bęrjask} `fight'; \textit{hitta}, `find,'
\textit{hittask} `meet.'  In other cases it specializes the meaning of
the verb, often emphasizing the idea of energy or effort: \textit{koma}
`come,' \textit{komask} `make one's way.'

\textbf{191.} The impersonal form of expression is widely used in
Icelandic: \textit{rak ā storm} (acc.) \textit{fyrir þeim} `a storm was driven
in their face.'

\textbf{192.} The indef.\ `one' is expressed in the same way by the third
pers.\ sg., and this form of expression is often used when the
subject is perfectly definite: \textit{ok freista skal þessar īþrōttar}
`and this feat shall be tried (by you).'

\textbf{193.} The abrupt change from the indirect to the direct narration
is very common: \textit{Haraldi konungi var sagt at þar var komit
bjarndȳri, `ok ā Īslęnzkr maðr,'} `King Harold was told that a
bear had arrived, and that an Icelander owned it.'  The direct
narration is also used after \textit{at} (that): \textit{hann svarar at `ek
skal rīða til Hęljar'} `he answers that he will ride to Hel.'

\emptypage

\part{Texts}


\emptypage

\chapter{Thor}

\resetlinenumber
\begin{linenumbers}

Þōrr er āsanna framastr, sā er kallaðr er Āsa-þōrr eða\\
Ǫku-þōrr; hann er stęrkastr allra guðanna ok manna.\\
Hann ā þar rīki er Þrūð-vangar heita, en hǫll hans heitir\\
Bilskīrnir; ī þeim sal eru fimm hundruð gōlfa ok fjōrir tigir;\\
þat er hūs mest, svā at męnn hafa gǫ̈rt.

Þōrr ā hafra tvā, er svā heita, Tann-gnjōstr ok Tann-\\
grisnir, ok reið þā er hann ękr, en hafrarnir draga reiðina;\\
þvī er hann kallaðr Ǫkuþōrr.  Hann ā ok þrjā kost-gripi.\\
Einn þeira er hamarrinn Mjǫllnir, er hrīm-þursar ok berg-\\
risar kęnna, þā er hann kömr ā lopt, ok er þat eigi undarligt:\\
hann hęfir lamit margan haus ā fęðrum eða frændum þeira.\\
Annan grip ā hann bęztan, męgin-gjarðar; ok er hann\\
spęnnir þeim um sik, þā vęx honum ās-męgin hālfu.  En\\
þriðja hlut ā hann þann er mikill gripr er ī, þat eru jārn-\\
glōfar; þeira mā hann eigi missa við hamarskaptit.  En\\
engi er svā frōðr at tęlja kunni ǫll stōr-virki hans.\\

\end{linenumbers}

\emptypage

\chapter{Thor and Ūgarðaloki}

\resetlinenumber
\begin{linenumbers}

Þat er upp-haf þessa māls at Ǫkuþōrr fōr með hafra sīna\\
ok reið, ok með honum sā āss er Loki er kallaðr; koma\\
þeir at kveldi til eins bōnda ok fā þar nātt-stað.  En um\\
kveldit tōk Þōrr hafra sīna, ok skar bāða; ęptir þat vāru\\
þeir flęgnir ok bornir til kętils; en er soðit var, þā sęttisk\\
Þōrr til nātt-verðar ok þeir lags-męnn.  Þōrr bauð til matar\\
með sēr bōndanum, ok konu hans, ok bǫrnum þeira; sonr\\
bōnda hēt Þjālfi, en Rǫskva dōttir.  Þā lagði Þōrr hafr-\\
stǫkurnar utar frā eldinum, ok mælti at bōndi ok heima-męnn\\
hans skyldu kasta ā hafrstǫkurnar beinunum.  Þjālfi, sonr\\
bōnda, helt ā lær-lęgg hafrsins, ok sprętti ā knīfi sīnum, ok\\
braut til męrgjar.  Þōrr dvalðisk þar of nāttina; en ī ōttu\\
fyrir dag stōð hann upp, ok klæddi sik, tōk hamarinn\\
Mjǫllni ok brā upp, ok vīgði hafrstǫkurnar; stōðu þā upp\\
hafrarnir, ok var þā annarr haltr ęptra fœti.  Þat fann Þōrr,\\
ok talði at bōndinn eða hans hjōn mundi eigi skynsamliga\\
hafa farit með beinum hafrsins: kęnnir hann at brotinn var\\
lærlęggrinn.  Eigi þarf langt frā þvī at sęgja: vita męgu þat\\
allir hvęrsu hræddr bōndinn mundi vera, er hann sā at Þōrr\\
lēt sīga brȳnnar ofan fyrir augun; en þat er sā augnanna,\\
þā hugðisk hann falla mundu fyrir sjōninni einni samt; hann\\
hęrði hęndrnar at hamar-skaptinu svā at hvītnuðu knūarnir.\\
En bōndinn gǫ̈rði sem vān var, ok ǫll hjōnin: kǫlluðu āka-\\
fliga, bāðu sēr friðar, buðu at fyrir kvæmi alt þat er þau\\
āttu.  En er hann sā hræzlu þeira, þā gekk af honum\\
mōðrinn, ok sefaðisk hann; ok tōk af þeim ī sætt bǫrn\\
þeira, Þjālfa ok Rǫsku, ok gǫ̈rðusk þau þā skyldir þjōnustu-\\
męnn Þōrs, ok fylgja þau honum jafnan sīðan.

Lēt hann þar ęptir hafra, ok byrjaði fęrðina austr ī\\
Jǫtun-heima, ok alt til hafsins; ok þā fōr hann ūt yfir\\
hafit þat it djūpa; en er hann kom til lands, þā gekk hann\\
upp, ok með honum Loki, ok Þjālfi, ok Rǫskva.  Þā er þau\\
hǫfðu lītla hrīð gęngit, varð fyrir þeim mǫrk stōr; gengu\\
þau þann dag allan til myrkrs.  Þjālfi var allra manna\\
fōt-hvatastr; hann bar kȳl Þōrs, en til vista var eigi gott.\\
Þā er myrkt var orðit, leituðu þeir sēr til nāttstaðar, ok\\
fundu fyrir sēr skāla nakkvarn mjǫk mikinn, vāru dyrr ā\\
ęnda, ok jafn-breiðar skālanum; þar leituðu þeir sēr nātt-\\
bōls.  En of miðja nātt varð land-skjālfti mikill, gekk jǫrðin\\
undir þeim skykkjum, ok skalf hūsit.  Þā stōð Þōrr upp, ok\\
hēt ā lagsmęnn sīna; ok leituðusk fyrir, ok fundu af-hūs til\\
hœgri handar i miðjum skālanum, ok gengu þannig; sęttisk\\
Þōrr ī dyrrin, en ǫnnur þau vāru innar frā honum, ok vāru\\
þau hrædd, en Þōrr helt hamarskaptinu, ok hugði at vęrja\\
sik; þā heyrðu þau ym mikinn ok gnȳ.  En er kom at\\
dagan, þā gekk Þōrr ūt, ok sēr hvar lā maðr skamt frā\\
honum ī skōginum, ok var sā eigi lītill; hann svaf, ok hraut\\
stęrkliga.  Þā þōttisk Þōrr skiIja hvat lātum verit hafði of\\
nāttina; hann spęnnir sik męgingjǫrðum, ok ōx honum\\
āsmęgin; en ī þvī vaknar maðr sā, ok stōð skjōtt upp; en\\
þā er sagt at Þōr varð bilt einu sinni at slā hann með ham-\\
rinum; ok spurði hann at nafni, en sā nęfndisk Skrȳmir:\\
`en eigi þarf ek,' sagði hann, `at spyrja þik at nafni: kęnni\\
ek at þū ert Āsaþōrr; en hvārt hęfir þū dręgit ā braut hanzka\\
minn?'  Seildisk þā Skrȳmir til, ok tōk upp hanzka sinn;\\
sēr Þōrr þā at þat hafði hann haft of nāttina fyrir skāla, en\\
afhūsit þat var þumlungrinn hanzkans.  Skrȳmir spurði ef\\
Þōrr vildi hafa fǫru-neyti hans, en Þōrr jātti þvī.  Þā tōk\\
Skrȳmir ok leysti nestbagga sinn, ok bjōsk til at eta dǫgurð,\\
en Þōrr ī ǫðrum stað ok hans fēlagar.  Skrȳmir bauð þā at\\
þeir lęgði mǫtu-neyti sitt, en Þōrr jātti þvī; þā batt Skrȳmir\\
nest þeira alt ī einn bagga, ok lagði ā bak sēr; hann gekk\\
fyrir of daginn, ok steig hęldr stōrum, en sīðan at kveldi\\
leitaði Skrȳmir þeim nāttstaðar undir eik nakkvarri mikilli.\\
Þā mælti Skrȳmir til Þōrs at hann vill lęggjask niðr at\\
sofna; `en þēr takið nest-baggann, ok būið til nātt-verðar\\
yðr.'  Þvī næst sofnar Skrȳmir, ok hraut fast; en Þōrr tōk\\
nest-baggann ok skal leysa; en svā er at sęgja, sem ū-trūligt\\
mun þykkja, at engi knūt fekk hann leyst, ok engi ālar-\\
ęndann hreyft, svā at þā væri lausari en āðr.  Ok er hann\\
sēr at þetta verk mā eigi nȳtask, þā varð hann reiðr, greip\\
þā hamarinn Mjǫllni tveim hǫndum, ok steig fram ǫðrum\\
fœti at þar er Skrȳmir lā, ok lȳstr ī hǫfuð honum; en\\
Skrȳmir vaknar, ok spyrr hvārt laufs-blað nakkvat felli ī\\
hǫfuð honum, eða hvārt þeir hęfði þā matazk, ok sē būnir\\
til rękkna.  Þōrr sęgir at þeir munu þā sofa ganga.  Ganga\\
þau þā undir aðra eik.  Er þat þēr satt at sęgja, at ekki var\\
þā ōttalaust at sofa.  En at miðri nātt þā heyrir Þōrr at\\
Skrȳmir hrȳtr, ok söfr fast, svā at dunar ī skōginum.  Þā\\
stęndr hann upp, ok gęngr til hans, reiðir hamarinn tītt ok\\
hart, ok lȳstr ofan ī miðjan hvirfil honum; hann kęnnir at\\
hamars-muðrinn sökkr djūpt ī hǫfuðit.  En ī þvī bili vaknar\\
Skrȳmir, ok mælti: `hvat er nū? fell akarn nakkvat ī hǫfuð\\
mēr? eða hvat er tītt um þik, Þōrr?'  En Þōrr gekk aptr\\
skyndiliga, ok svarar at hann var þā nȳvaknaðr, sagði at þā\\
var mið nātt, ok enn væri māl at sofa.  Þā hugsaði Þōrr þat,\\
ef hann kvæmi svā ī fœri at slā hann it þriðja hǫgg, at aldri\\
skyldi hann sjā sik sīðan; liggr nū ok gætir ef Skrȳmir\\
sofnaði fast.  En lītlu fyrir dagan þā heyrir hann at Skrȳmir\\
mun sofnat hafa; stęndr þā upp, ok hleypr at honum, reiðir\\
þā hamarinn af ǫllu afli ok lȳstr ā þunn-vangann þann er\\
upp vissi; sökkr þā hamarrinn upp at skaptinu.  En Skrȳ-\\
mir sęttisk upp, ok strauk of vangann, ok mælti: `hvārt\\
munu fuglar nakkvarir sitja ī trēnu yfir mēr? mik grunaði,\\
er ek vaknaða, at tros nakkvat af kvistunum felli ī hǫfuð\\
mēr; hvārt vakir þū, Þōrr?  Māl mun vera upp at standa\\
ok klæðask, en ekki eigu þēr nū langa leið fram til borgar-\\
innar er kǫlluð er Ūt-garðr.  Heyrt hęfi ek at þēr hafið\\
kvisat ī milli yðvar at ek væra ekki lītill maðr vęxti, en sjā\\
skulu þēr þar stœrri męnn, er þēr komið ī Ūtgarð.  Nū mun\\
ek rāða yðr heil-ræði: lāti þēr eigi stōrliga yfir yðr, ekki\\
munu hirðmęnn Ūtgarða-loka vel þola þvīlīkum kǫgur-\\
sveinum kǫpur-yrði; en at ǫðrum kosti hverfið aptr, ok\\
þann ætla ek yðr bętra af at taka.  En ef þēr vilið fram\\
fara, þā stęfni þēr ī austr, en ek ā nū norðr leið til fjalla\\
þessa er nū munu þēr sjā mega.'  Tękr Skrȳmir nest-bag-\\
garm, ok kastar ā bak sēr, ok snȳr þvęrs ā braut ī skōginn\\
frā þeim, ok er þess eigi getit at æsirnir bæði þā heila\\
hittask.

Þōrr fōr fram ā leið ok þeir fēlagar, ok gekk fram til miðs\\
dags; þā sā þeir borg standa ā vǫllum nǫkkurum, ok sęttu\\
hnakkann ā bak sēr aptr, āðr þeir fengu sēt yfir upp; ganga\\
til borgarinnar, ok var grind fyri borg-hliðinu, ok lokin aptr.\\
Þōrr gekk ā grindina, ok fekk eigi upp lokit; en er þeir\\
þreyttu at komask ī borgina, þā smugu þeir milli spalanna\\
ok kōmu svā inn; sā þā hǫll mikla, ok gengu þannig; var\\
hurðin opin; þā gengu þeir inn, ok sā þar marga męnn\\
ā tvā bękki, ok flesta œrit stōra.  Þvī næst koma þeir fyrir\\
konunginn, Ūtgarðaloka, ok kvǫddu hann, en hann leit seint\\
til þeira, ok glotti um tǫnn, ok mælti: `seint er um langan\\
veg at spyrja tīðinda, eða er annan veg en ek hygg, at þessi\\
svein-stauli sē Ǫkuþōrr? en meiri muntu vera en mēr līzk\\
þū; eða hvat īþrōtta er þat er þēr fēlagar þykkizk vera við\\
būnir?  Engi skal hēr vera með oss sā er eigi kunni nakkvars\\
konar list eða kunnandi um fram flesta męnn.'  Þā sęgir sā\\
er sīðast gekk, er Loki heitir: `kann ek þā īþrōtt, er ek em\\
al-būinn at reyna, at engi er hēr sā inni er skjōtara skal eta\\
mat sinn en ek.'  Þā svarar Ūtgarðaloki: `īþrōtt er þat, ef\\
þū ęfnir, ok freista skal þā þessar īþrōttar;' kallaði utar ā\\
bękkinn at sā er Logi heitir skal ganga ā gōlf fram, ok freista\\
sīn ī mōti Loka.  Þā var tękit trog eitt, ok borit inn ā hallar-\\
gōlfit, ok fylt af slātri; sęttisk Loki at ǫðrum ęnda, en Logi\\
at ǫðrum, ok āt hvārr-tvęggja sem tīðast, ok mœttusk ī miðju\\
troginu; hafði þā Loki etit slātr alt af beinum, en Logi hafði\\
ok etit slātr alt ok beinin með, ok svā trogit; ok sȳndisk nū\\
ǫllum sem Loki hęfði lātit leikinn.  Þā spyrr Ūtgarðaloki\\
hvat sā inn ungi maðr kunni leika.  En Þjālfi sęgir at hann\\
mun freista at ręnna skeið nǫkkur við einn-hvęrn þann er\\
Ūtgarðaloki fær til.  Hann sęgir, Ūtgarðaloki, at þetta er\\
gōð īþrōtt, ok kallar þess meiri vān at hann sē vel at sēr\\
būinn of skjōtleikinn, ef hann skal þessa īþrōtt inna; en þō\\
lætr hann skjōtt þessa skulu freista.  Stęndr þā upp Ūtgarða-\\
loki, ok gęngr ūt, ok var þar gott skeið at ręnna ęptir slēttum\\
vęlli.  Þā kallar Ūtgarðaloki til sīn sveinstaula nakkvarn, er\\
nęfndr er Hugi, ok bað hann ręnna ī kǫpp við Þjālfa.  Þā\\
taka þeir it fyrsta skeið, ok er Hugi þvī framar at hann\\
snȳsk aptr ī mōti honum at skeiðs ęnda.  Þā mælti Ūtgarða-\\
loki: `þurfa muntu, Þjālfi, at lęggja þik meir fram, ef þū\\
skalt vinna leikinn; en þō er þat satt, at ekki hafa hēr komit\\
þeir męnn er mēr þykkja fōthvatari en svā.'  Þā taka þeir\\
aptr annat skeið, ok þā er Hugi er kominn til skeiðs ęnda,\\
ok hann snȳsk aptr, þā var langt kōlf-skot til Þjālfa.  Þā\\
mælti Ūtgarðaloki: `vel þykkir mēr Þjālfi ręnna; en eigi\\
trūi ek honum nū at hann vinni leikinn, en nū mun reyna,\\
er þeir ręnna it þriðja skeiðit.'  Þā taka þeir ęnn skeið; en\\
er Hugi er kominn til skeiðs ęnda ok snȳsk aptr, ok er\\
Þjālfi eigi þā kominn ā mitt skeiðit; þā sęgja allir at reynt\\
er um þenna leik.  Þā spyrr Ūtgarðaloki Þōr, hvat þeira\\
īþrōtta mun vera er hann muni vilja birta fyrir þeim, svā\\
miklar sǫgur sem męnn hafa gǫ̈rt um stōrvirki hans.  Þā\\
mælti Þōrr at hęlzt vill hann þat taka til, at þreyta drykkju\\
við einnhvęrn mann.  Ūtgarðaloki sęgir at þat mā vel vera;\\
ok gęngr inn ī hǫllina, ok kallar skutil-svein sinn, biðr at\\
hann taki vītis-horn þat, er hirðmęnn eru vanir at drekka af.\\
Þvī næst kömr fram skutilsveinn með horninu, ok fær Þōr ī\\
hǫnd.  Þā mælti Ūtgarðaloki: `af horni þessu þykkir þā\\
vel drukkit, ef ī einum drykk gęngr af, en sumir męnn\\
drekka af ī tveim drykkjum, en engi er svā lītill drykkju-\\
maðr, at eigi gangi af ī þrimr.'  Þōrr lītr ā hornit, ok sȳnisk\\
ekki mikit, ok er þō hęldr langt, en hann er mjǫk þyrstr;\\
tękr at drekka, ok svelgr allstōrum, ok hyggr at eigi skal\\
þurfa at lūta optar at sinni ī hornit.  En er hann þraut\\
örindit, ok hann laut ōr horninu, ok sēr hvat leið drykkinum,\\
ok līzk honum svā, sem all-lītill munr mun vera at nū sē\\
lægra ī horninu en āðr.  Þā mælti Ūtgarðaloki: `vel er\\
drukkit, ok eigi til mikit; eigi munda-k trūa, ef mēr væri\\
sagt frā, at Āsaþōrr mundi eigi meira drykk drekka; en\\
þō veit ek at þū munt vilja drekka af ī ǫðrum drykk.'\\
Þōrr svarar engu, sętr hornit ā munn sēr, ok hyggr nū at\\
hann skal drekka meira drykk, ok þreytir ā drykkjuna, sem\\
honum vannsk til örindi, ok sēr ęnn at stikillinn hornsins\\
vill ekki upp svā mjǫk sem honum līkar; ok er hann tōk\\
hornit af munni sēr ok sēr, līzk honum nū svā, sem minna\\
hafi þorrit en ī inu fyrra sinni; er nū gott beranda borð\\
ā horninu.  Þā mælti Ūtgarðaloki: `hvat er nū, Þōrr?\\
muntu nū eigi sparask til eins drykkjar meira en þēr mun\\
hagr ā vera?  Svā līzk mēr, ef þū skalt nū drekka af horninu\\
inn þriðja drykkinn, sem þessi mun mestr ætlaðr; en ekki\\
muntu mega hēr með oss heita svā mikill maðr sem æsir\\
kalla þik, ef þū gǫ̈rir eigi meira af þēr um aðra leika en mēr\\
līzk sem um þenna mun vera.'  Þā varð Þōrr reiðr, sętr\\
hornit ā munn sēr, ok drekkr sem ākafligast mā hann, ok\\
þreytir sem lęngst ā drykkinn; en er hann sā ī hornit, þā\\
hafði nū hęlzt nakkvat munr ā fęngizk, ok þā bȳðr hann upp\\
hornit, ok vill eigi drekka meira.  Þā mælti Ūtgarðaloki:\\
`auð-sēt er nū at māttr þinn er ekki svā mikill sem vēr\\
hugðum; en vill-tu freista um fleiri leika?  Sjā mā nū, at\\
ekki nȳtir þū hēr af.'  Þōrr svarar: `freista mā ek ęnn of\\
nakkvara leika, en undarliga mundi mēr þykkja, þā er ek var\\
heima með āsum, ef þvīlīkir drykkir væri svā lītlir kallaðir.  En\\
hvat leik vili þēr nū bjōða mēr?'  Þā mælti Ūtgarðaloki: `þat\\
gǫ̈ra hēr ungir sveinar er lītit mark mun at þykkja, at hęfja\\
upp af jǫrðu kǫtt minn; en eigi munda-k kunna at mæla\\
þvīlīkt við Āsaþōr, ef ek hęfða eigi sēt fyrr at þū ert miklu\\
minni fyrir þēr en ek hugða.'  Þvī næst hljōp fram kǫttr\\
einn grār ā hallargōlfit, ok hęldr mikill; en Þōrr gekk til, ok\\
tōk hęndi sinni niðr undir miðjan kviðinn, ok lypti upp, en\\
kǫttrinn beygði kęnginn, svā sem Þōrr rētti upp hǫndina; en\\
er Þōrr seildisk svā langt upp sem hann mātti lęngst, þā lētti\\
kǫttrinn einum fœti, ok fær Þōrr eigi framit þenna leik.  Þā\\
mælti Ūtgarðaloki: `svā fōr þessi leikr sem mik varði;\\
kǫttrinn er hęldr mikill, en Þōrr er lāgr ok lītill hjā stōr-\\
męnni þvī sem hēr er með oss.'  Þā mælti Þōrr: `svā lītinn\\
sem pēr kallið mik, þā gangi nū til einnhvęrr, ok fāisk við\\
mik; nū em ek reiðr.'  Þā svarar Ūtgarðaloki, ok litask um\\
ā bękkina, ok mælti: `eigi sē ek þann mann hēr inni, er\\
eigi mun lītil-ræði ī þykkja at fāsk við þik;' ok ęnn mælti\\
hann: `sjām fyrst, kalli mēr hingat kęrlinguna, fōstru mīna\\
Ęlli, ok fāisk Þōrr við hana, ef hann vill; fęlt hęfir hon þā\\
męnn er mēr hafa litizk eigi ū-stęrkligri en Þōrr er.'  Þvī\\
næst gekk ī hǫllina kęrling ein gǫmul.  Þā mælti Ūtgar-\\
ðaloki, at hon skal taka fang við Āsaþōr.  Ekki er langt um.\\
at gǫ̈ra: svā fōr fang þat at þvī harðara er Þōrr knūðisk at\\
fanginu, þvī fastara stōð hon; þā tōk kęrling at leita til\\
þragða, ok varð Þōrr þā lauss ā fōtum, ok vāru þær svip-\\
tingar all-harðar, ok eigi lęngi āðr en Þōrr fell ā knē ǫðrum\\
fœti.  Þā gekk til Ūtgarðaloki, bað þau hætta fanginu, ok\\
sagði svā, at Þōrr mundi eigi þurfa at bjōða fleirum mǫnnum\\
fang ī hans hǫll; var þā ok liðit ā nātt, vīsaði Ūtgarðaloki\\
Þōr ok þeim fēlǫgum til sætis, ok dvęljask þar nātt-langt ī\\
gōðum fagnaði.

En at morgni, þegar dagaði, stęndr Þōrr upp ok þeir\\
fēlagar, klæða sik, ok eru būnir braut at ganga.  Þā kom\\
þar Ūtgarðaloki, ok lēt sętja þeim borð; skorti þā eigi\\
gōðan fagnað, mat ok drykk.  En er þeir hafa matazk, þā\\
snūask þeir til fęrðar.  Ūtgarðaloki fylgir þeim ūt, gęngr\\
með þeim braut ōr borginni; en at skilnaði þā mælti Ūtgar-\\
ðaloki til Þōrs, ok spyrr hvęrnig honum þykkir fęrð sīn\\
orðin, eða hvārt hann hęfir hitt rīkara mann nakkvarn en\\
sik.  Þōrr svarar at eigi mun hann þat sęgja, at eigi hafi\\
hann mikla ū-sœmð farit ī þeira við-skiptum; `en þō veit\\
ek at þēr munuð kalla mik lītinn mann fyrir mēr, ok uni ek\\
þvī illa.'  Þā mælti Ūtgarðaloki: `nū skal sęgja þēr it sanna,\\
er þū ert ūt kominn ōr borginni---ok ef ek lifi ok mega-k\\
rāða, þā skaltu aldri optar ī hana koma; ok þat veit trūa\\
mīn, at aldri hęfðir þū ī hana komit, ef ek hęfða vitat āðr at\\
þū hęfðir svā mikinn krapt með þēr, ok þū hafðir svā nær\\
haft oss mikilli ū-fœru.  En sjōn-hvęrfingar hęfi ek gǫ̈rt þēr,\\
svā at fyrsta sinn, er ek fann þik ā skōginum, kom ek til\\
fundar við yðr; ok þā er þū skyldir leysa nestbaggann, þā\\
hafða-k bundit með gres-jārni, en þū fannt eigi hvar upp\\
skyldi lūka.  En þvī næst laust þū mik með hamrinum þrjū\\
hǫgg, ok var it fyrsta minst, ok var þō svā mikit, at mēr\\
mundi ęndask til bana, ef ā hęfði komit; en þar er þū sātt\\
hjā hǫll minni set-berg, ok þar sātt-u ofan ī þrjā dala fer-\\
skeytta ok einn djūpastan, þat vāru hamarspor þin; setber-\\
ginu brā ek fyrir hǫggin en eigi sātt þū þat.  Svā var ok of\\
leikana, er þēr þreyttuð við hirðmęnn mīna.  Þā var þat it\\
fyrsta, er Loki gǫ̈rði; hann var mjǫk soltinn, ok āt tītt; en\\
sā er Logi hēt, þat var villi-eldr, ok bręndi hann eigi seinna\\
slātrit en trogit.  En er Þjālfi þreytti rāsina við þann er\\
Hugi hēt, þat var hugi minn, ok var Þjālfa eigi vænt at\\
þreyta skjōt-fœri við hann.  En er þū drakkt af horninu, ok\\
þōtti þēr seint līða,---en þat veit trūa mīn, at þā varð þat\\
undr, er ek munda eigi trūa at vera mætti; annarr ęndir\\
hornsins var ūt ī hafi, en þat sāttu eigi; en nū, er þū kömr\\
til sævarins, þā mun-tu sjā mega, hvęrn þurð þū hęfir drukkit\\
ā sænum.'  Þat eru nū fjǫrur kallaðar.  Ok ęnn mælti hann:\\
`eigi þōtti mēr hitt minna vera vert, er þū lyptir upp kęt-\\
tinum, ok þēr satt at sęgja, þā hræddusk allir þeir er sā,\\
er þū lyptir af jǫrðu einum fœtinum; en sā kǫttr var eigi\\
sem þēr sȳndisk; þat var Miðgarðs-ormr, er liggr um lǫnd\\
ǫll, ok vannsk honum varliga lęngðin til, at jǫrðina tœki\\
sporðr ok hǫfuð; ok svā langt seildisk þū upp at skamt var\\
þā til himins.  En hitt var ok mikit undr um fangit, er þū\\
fekkzk við Ęlli; fyrir þvī at engi hęfir sā orðit, ok engi\\
mun verða, ef svā gamall er at ęlli bīðr, at eigi komi ęllin\\
ǫllum til falls.  Ok er nū þat sātt at sęgja, at vēr munum\\
skiljask, ok mun þā bętr hvārratvęggju handar at þēr komið\\
eigi optar mik at hitta; ek mun ęnn annat sinn vęrja borg\\
mīna með þvīlīkum vēlum eða ǫðrum, svā at ekki vald munu\\
þēr ā mēr fā.'  En er Þōrr heyrði þessa tǫlu, greip hann til\\
hamarsins, ok bregðr ā lopt; en er hann skal fram reiða, þā\\
sēr hann þar hvęrgi Ūtgarðaloka, ok þā snȳsk hann aptr til\\
borgarinnar, ok ætlask þā fyrir at brjōta borgina; þā sēr\\
hann þar vǫllu vīða ok fagra, en enga borg.  Snȳsk hann\\
þā aptr, ok fęrr leið sina, til þess er hann kom aptr ī Þrūð-\\
vanga.

\end{linenumbers}

\emptypage

\chapter{Balder}

\resetlinenumber
\begin{linenumbers}

Annarr sonr Ōðins er Baldr, ok er frā honum gott at\\
sęgja: hann er bęztr, ok hann lofa allir.  Hann er svā fagr\\
ā-litum ok bjartr svā at lȳsir af honum; ok eitt gras er svā\\
hvītt at jafnat er til Baldrs brār, þat er allra grasa hvītast;\\
ok þar ęptir māttu marka hans fęgrð, bæði ā hār ok ā līki;\\
hann er vitrastr āsanna, ok fęgrstr taliðr ok līknsamastr.  En\\
sū nāttūra fylgir honum at engi mā haldask dōmr hans.\\
Hann bȳr þar sem heitir Breiða-blik, þat er ā himni; ī þeim\\
stað mā ekki vera ū-hreint, svā sem hēr sęgir:

\begin{verse}
	Breiðablik heita,\gap þar er Baldr hęfir\\
	\gap sēr of gǫ̈rva sali;\\
	ī þvī landi\gap er ek liggja veit\\
	\gap fæsta feikn-stafi.
\end{verse}

\end{linenumbers}

\emptypage

\chapter{The Death of Balder}

\resetlinenumber
\begin{linenumbers}

Þat er upphaf þessar sǫgu, at Baldr inn gōða dreymði\\
drauma stōra ok hættliga um līf sitt.  En er hann sagði\\
āsunum draumana, þā bāru þeir saman rāð sīn, ok var þat\\
gǫ̈rt at beiða griða Baldri fyrir alls konar hāska; ok Frigg\\
tōk svardaga til þess, at eira skyldu Baldri eldr ok vatn, jārn\\
ok alls konar mālmr, steinar, jǫrðin, viðirnir, sōttirnar, dȳrin,\\
fuglarnir, eitr, ormar.  En er þetta var gǫ̈rt ok vitat, þā var\\
þat skęmtun Baldrs ok āsanna at hann skyldi standa upp ā\\
þingum, en allir aðrir skyldu sumir skjōta ā hann, sumir\\
hǫggva til, sumir bęrja grjōti.  En hvat sem at var gǫ̈rt,\\
sakaði hann ekki, ok þōtti þetta ǫllum mikill frami.  En er\\
þetta sā Loki Laufeyjar-son, þā līkaði honum illa er Baldr\\
sakaði ekki.  Hann gekk til Fęn-salar til Friggjar, ok brā\\
sēr ī konu līki; þā spyrr Frigg ef sū kona vissi hvat æsir\\
hǫfðusk at ā þinginu.  Hon sagði at allir skutu at Baldri,\\
ok þat, at hann sakaði ekki.  Þā mælti Frigg: `eigi munu\\
vāpn eða viðir granda Baldri; eiða hęfi ek þęgit af ǫllum\\
þeim.'  Þā spyrr konan: `hafa allir hlutir eiða unnit at eira\\
Baldri?'  Þā svarar Frigg: `vęx viðar-teinungr einn fyrir\\
vestan Val-hǫll; sā er Mistilteinn kallaðr; sā þōtti mēr ungr\\
at kręfja eiðsins.'  Þvī næst hvarf konan ā braut; en Loki\\
tōk Mistiltein ok sleit upp, ok gekk til þings.  En Hǫðr stōð\\
utarliga ī mannhringinum, þvī at hann var blindr.  Þā mælti\\
Loki við hann: `hvī skȳtr þū ekki at Baldri?'  Hann svarar:\\
`þvī at ek sē eigi, hvar Baldr er, ok þat annat, at ek em\\
vapnlauss.'  Þā mælti Loki: `gǫ̈r-ðu þō ī līking annarra,\\
manna, ok veit Baldri sœmð sem aðrir męnn; ek mun vīsa\\
þēr til, hvar hann stęndr; skjōt at honum vęndi þessum.'\\
Hǫðr tōk Mistiltein, ok skaut at Baldri at tilvīsun Loka;\\
flaug skotit ī gęgnum hann, ok fell hann dauðr til jarðar; ok\\
hęfir þat mest ū-happ verit unnit með goðum ok mǫnnum.\\
Þā er Baldr var fallinn, þā fellusk ǫllum āsum orð-tǫk, ok\\
svā hęndr at taka til hans; ok sā hvęrr til annars, ok vāru\\
allir með einum hug til þess er unnit hafði verkit; en engi\\
mātti hęfna: þar var svā mikill griða-staðr.  En þā er\\
æsirnir freistuðu at mæla, þā var hitt þō fyrr, at grātrinn\\
kom upp, svā at engi mātti ǫðrum sęgja með orðunum frā\\
sīnum harmi.  En Ōðinn bar þeim mun vęrst þenna skaða,\\
sem hann kunni mesta skyn, hvęrsu mikil af-taka ok missa\\
āsunum var ī frā-falli Baldrs.  En er goðin vitkuðusk, þā\\
mælti Frigg ok spurði, hvęrr sā væri með āsum, er eignask\\
vildi allar āstir hennar ok hylli, ok vili hann rīða ā hęl-veg\\
ok freista ef hann fāi fundit Baldr, ok bjōða Hęlju ūt-lausn,\\
ef hon vill lāta fara Baldr heim ī Ās-garð.  En sā er nęfndr\\
Hęrmōðr inn hvati, sonr Ōðins, er til þeirar farar varð.  Þā\\
var tękinn Sleipnir, hestr Ōðins, ok leiddr fram, ok steig\\
Hęrmōðr ā þann hest, ok hleypði braut.

En æsirnir tōku līk Baldrs ok fluttu til sævar.  Hring-\\
horni hēt skip Baldrs, hann var allra skipa mestr; hann\\
vildu goðin fram sętja, ok gǫ̈ra þar ā bāl-fǫr Baldrs; en\\
skipit gekk hvęrgi fram.  Þā var sęnt ī Jǫtunheima ęptir\\
gȳgi þeiri er Hyrrokin hēt; en er hon kom, ok reið vargi,\\
ok hafði hǫgg-orm at taumum, þā hljōp hon af hestinum,\\
en Ōðinn kallaði til ber-sęrki fjōra at gæta hestsins, ok fengu\\
þeir eigi haldit, nema þeir fęldi hann.  Þā gekk Hyrrokin ā\\
fram-stafn nǫkkvans, ok hratt fram ī fyrsta við-bragði, svā\\
at eldr hraut ōr hlunnunum, ok lǫnd ǫll skulfu.  Þā varð\\
Þōrr reiðr, ok greip hamarinn, ok mundi þā brjōta hǫfuð\\
hęnnar, āðr en goðin ǫll bāðu hęnni friðar.  Þā var borit\\
ūt ā skipit līk Baldrs; ok er þat sā kona hans, Nanna, Neps\\
dōttir, þā sprakk hon af harmi, ok dō; var hon borin ā\\
bālit, ok slęgit ī eldi.  Þā stōð Þōrr at, ok vīgði bālit með\\
Mjǫllni; en fyrir fōtum hans rann dvergr nakkvarr, sā er\\
Litr nęfndr; en Þōrr spyrndi fœti sīnum ā hann, ok hratt\\
honum ī eldinn, ok brann hann.  En at þessi bręnnu sōtti\\
margs konar þjōð: fyrst at sęgja frā Ōðni, at með honum\\
fōr Frigg ok valkyrjur ok hrafnar hans; en Freyr ōk ī kęrru\\
með gęlti þeim er Gullin-bursti heitir eða Slīðrug-tanni; en\\
Heimdallr reið hesti þeim er Gull-toppr heitir; en Freyja\\
kǫttum sīnum.  Þar kömr ok mikit fōlk hrīmþursa, ok berg-\\
risar.  Ōðinn lagði ā bālit gullhring þann er Draupnir\\
heitir; honum fylgði sīðan sū nāttūra, at hina nīundu hvęrja\\
nātt drupu af honum ātta gullhringar jafn-hǫfgir.  Hestr\\
Baldrs var leiddr ā bālit með ǫllu reiði.

En þat er at sęgja frā Hęrmōði, at hann reið nīu nætr\\
dökkva dala ok djūpa, svā at hann sā ekki, fyrr en hann\\
kom til ārinnar Gjallar, ok reið ā Gjallar-brūna; hon er\\
þǫkð lȳsi-gulli.  Mōðguðr er nęfnd mær sū er gætir brūar-\\
innar; hon spurði hann at nafni eða ætt, ok sagði at hinn\\
fyrra dag riðu um brūna fimm fylki dauðra manna; `en eigi\\
dynr brūin minnr undir einum þēr, ok eigi hęfir þū lit dauðra\\
manna; hvī rīðr þū hēr ā hęlveg?'  Hann svarar at `ek\\
skal rīða til hęljar at leita Baldrs, eða hvārt hęfir þū nakkvat\\
sēt Baldr ā hęlvegi?'  En hon sagði at Baldr hafði þar\\
riðit um Gjallarbrū; `en niðr ok norðr liggr hęlvegr.'  Þā\\
reið Hęrmōðr þar til er hann kom at hęl-grindum; þā steig\\
hann af hestinum, ok gyrði hann fast, steig upp, ok keyrði\\
hann sporum, en hestrinn hljōp svā hart, ok yfir grindina, at\\
hann kom hvęrgi nær.  Þā reið Hęrmōðr heim til hallar-\\
innar, ok steig af hesti, gekk inn ī hǫllina, sā þar sitja ī\\
ǫndvegi Baldr, brōður sinn; ok dvalðisk Hęrmōðr þar um\\
nāttina.  En at morgni þā beiddisk Hęrmōðr af Hęlju at\\
Baldr skyldi rīða heim með honum, ok sagði hvęrsu mikill\\
grātr var með āsum.  En Hęl sagði at þat skyldi svā reyna,\\
hvārt Baldr var svā āst-sæll sem sagt er; `ok ef allir hlutir ī\\
heiminum, kykvir ok dauðir, grāta hann, þā skal hann fara\\
til āsa aptr, en haldask með Hęlju, ef nakkvarr mællr við,\\
eða vill eigi grāta.'  Þā stōð Hęrmōðr upp, en Baldr leiðir\\
hann ūt ōr hǫllinni, ok tōk hringinn Draupni, ok sęndi Ōðni\\
til minja, en Nanna sęndi Frigg ripti ok ęnn fleiri gjafar,\\
Fullu fingr-gull.  Þā reið Hęrmōðr aptr leið sīna, ok kom ī\\
Āsgarð, ok sagði ǫll tīðindi þau er hann hafði sēt ok heyrt.

Þvī næst sęndu æsir um allan heim örind-reka, at biðja\\
at Baldr væri grātinn ōr hęlju; en allir gǫ̈rðu þat, męnninir,\\
ok kykvendin, ok jǫrðin, ok steinarnir, ok trē, ok allr mālmr;\\
svā sem þū munt sēt hafa, at þessir hlutir grāta, þā er þeir\\
koma ōr frosti ok ī hita.  Þā er sęndi-męnn fōru heim, ok\\
hǫfðu vel rekit sīn örindi, finna þeir ī hęlli nǫkkurum hvar\\
gȳgr sat; hon nęfndisk Þǫkk.  Þeir biðja hana grāta Baldr\\
ōr hęlju.  Hon svarar:

\begin{verse}
   `Þǫkk mun grāta\gap þurrum tārum\\
    Baldrs bālfarar;\\
    kyks nē dauðs\gap naut-k-a-k karls sonar;\\
    haldi Hęl þvī es hęfir!'
\end{verse}

En þess geta męnn, at þar hafi verit Loki Laufeyjar-son,\\
er flest hęfir ilt gǫ̈rt með āsum.

\end{linenumbers}

\emptypage

\chapter{Hēðinn and Hǫgni}

\resetlinenumber
\begin{linenumbers}

Konungr sā er Hǫgni er nęfndr ātti dōttur, er Hildr hēt.\\
Hana tōk at hęr-fangi konungr sā er Hēðinn hēt, Hjarranda-\\
son.  Þā var Hǫgni konungr farinn ī konunga-stęfnu; en er\\
hann spurði at hęrjat var ī rīki hans, ok dōttir hans var ī\\
braut tękin, þā fōr hann með sīnu liði at leita Hēðins, ok\\
spurði til hans at Hēðinn hafði siglt norðr með landi.  Þā\\
er Hǫgni konungr kom ī Noreg, spurði hann at Hēðinn\\
hafði siglt vestr um haf.  Þā siglir Hǫgni ęptir honum allt\\
til Orkn-eyja; ok er hann kom þar sem heitir Hā-ey, þā\\
var þar fyrir Hēðinn með lið sitt.  Þā fōr Hildr ā fund fǫður\\
sīns, ok bauð honum męn at sætt af hęndi Hēðins, en ī ǫðru\\
orði sagði hon at Hēðinn væri būinn at bęrjask, ok ætti\\
Hǫgni af honum engrar vægðar vān.  Hǫgni svārar stirt\\
dōttur sinni; en er hon hitti Hēðin, sagði hon honum, at\\
Hǫgni vildi enga sætt, ok bað hann būask til orrostu, ok\\
svā gǫ̈ra þeir hvārir-tvęggju, ganga upp ā eyna, ok fylkja\\
liðinu.  Þā kallar Hēðinn ā Hǫgna, māg sinn, ok bauð\\
honum sætt ok mikit gull at bōtum.  Þā svarar Hǫgni:\\
`of sīð bauzt-u þetta, ef þū vill sættask, þvī at nū hęfi ek\\
dręgit Dāins-leif, er dvergarnir gǫ̈rðu, er manns bani skal\\
verða, hvęrt sinn er bęrt er, ok aldri bilar ī hǫggvi, ok ekki\\
sār grœr, ef þar skeinisk af.'  Þā svarar Hēðinn: `sverði\\
hœlir þū þar, en eigi sigri; þat kalla ek gott hvęrt er drōttin-\\
holt er.'  Þā hōfu þeir orrostu þā er Hjaðninga-vīg er kallat,\\
ok bǫrðusk þann dag allan, ok at kveldi fōru konungar til\\
skipa.  En Hildr gekk of nāttina til valsins, ok vakði upp\\
með fjǫlkyngi alla þā er dauðir vāru; ok annan dag gengu\\
konungarnir ā vīg-vǫllinn ok bǫrðusk, ok svā allir þeir er\\
fellu hinn fyrra daginn.  Fōr svā sū orrosta hvęrn dag ęptir\\
annan, at allir þeir er fellu, ok ǫll vāpn þau er lāgu ā vīgvęlli,\\
ok svā hlīfar, urðu at grjōti.  En er dagaði, stōðu upp allir\\
dauðir męnn, ok bǫrðusk, ok ǫll vāpn vāru þā nȳ.  Svā er\\
sagt ī kvæðum, at Hjaðningar skulu svā bīða ragna-rökrs.

\end{linenumbers}

\emptypage

\chapter{The Death of Olaf Tryggvason}

\resetlinenumber
\begin{linenumbers}

Sveinn konungr tjūgu-skęgg ātti Sigrīði hina stōr-rāðu.\\
Sigrīðr var hinn mesti ū-vinr Ōlāfs konungs Tryggva-sonar;\\
ok fann þat til saka at Ōlāfr konungr hafði slitit einka-mālum\\
við hana, ok lostit hana ī and-lit.  Hon ęggjaði mjǫk Svein\\
konung til at halda orrostu við Ōlāf konung Tryggvason, ok\\
kom hon svā sīnum for-tǫlum at Sveinn konungr var full-\\
kominn at gǫ̈ra þetta rāð.  Ok snimma um vārit sęndi\\
Sveinn konungr męnn austr til Svī-þjōðar ā fund Ōlāfs\\
konungs Svīa-konungs, māgs sīns, ok Eirīks jarls; ok lēt sęgja\\
þeim at Ōlāfr, Noregs konungr, hafði leiðangr ūti, ok ætlaði\\
at fara um sumarit til Vind-lands.  Fylgði þat orð-sęnding\\
Dana-konungs, at þeir Svīakonungr ok Eirīkr jarl skyldi\\
hęr ūti hafa, ok fara til mōts við Svein konung, skyldu\\
þeir þā allir samt lęggja til orrostu við Ōlāf konung Tryggva-\\
son.  En Ōlāfr Svīakonungr ok Eirīkr jarl vāru þessar\\
fęrðar al-būnir, ok drōgu þā saman skipa-hęr mikinn af Svīa-\\
vęldi, fǫru þvī liði suðr til Dan-markar ok kvāmu þar svā,\\
at Ōlāfr konungr Tryggvason hafði āðr austr siglt.  Þeir\\
Svīakonungr ok Eirīkr jarl heldu til fundar við Danakonung,\\
ok hǫfðu þā allir saman ū-grynni hęrs.

Sveinn konungr, þā er hann hafði sęnt ęptir hęrinum,\\
þā sęndi hann Sigvalda jarl til Vindlands at njōsna um fęrð\\
Ōlāfs konungs Tryggvasonar, ok gildra svā til, at fundr\\
þeira Sveins konungs mætti verða.  Fęrr þā Sigvaldi jarl\\
leið sīna, ok kom fram ā Vindlandi, fōr til Jōmsborgar, ok\\
sīðan ā fund Ōlāfs konungs Tryggvasonar.  Vāru þar mikil\\
vināttu-māl þeira ā meðal, kom jarl sēr ī hinn mesta kærleik\\
við Ōlāf konung.  Āstrīðr kona jarls, dōttir Burizleifs konungs,\\
var vinr mikill Ōlāfs konungs, ok var þat mjǫk af hinum\\
fyrrum tęngðum, er Ōlafr konungr hafði ātt Geiru, systur\\
hęnnar.  Sigvaldi jarl var maðr vitr ok rāðugr; en er hann\\
kom sēr ī rāða-gęrð við Ōlāf konung, þā dvalði hann mjǫk\\
fęrðina hans austan at sigla, ok fann til þess mjǫk ȳmsa hluti.\\
En lið Ōlāfs konungs lēt geysi illa, ok vāru męnn mjǫk\\
heim-fūsir, er þeir lāgu albūnir, en veðr byr-væn.  Sigvaldi\\
jarl fekk njōsn leyniliga af Danmǫrk, at þā var austan kominn\\
hęrr Svīakonungs, ok Eirīkr jarl hafði þā ok būinn sinn hęr,\\
ok þeir hǫfðingjarnir mundu þā koma austr undir Vindland,\\
ok þeir hǫfðu ā kveðit, at þeir mundu bīða Ōlāfs konungs\\
við ey þā er Svǫlðr heitir, svā þat, at jarl skyldi svā til stilla\\
at þeir mætti þar finna Ōlāf konung.

Þā kom pati nakkvarr til Vindlands, at Sveinn Dana-\\
konungr hęfði hęr ūti, ok gǫ̈rðisk brātt sā kurr, at Sveinn\\
Danakonungr mundi vilja finna Ōlāf konung.  En Sigvaldi\\
jarl sęgir konungi: `ekki er þat rāð Sveins konungs at\\
lęggja til bardaga við þik með Dana-hęr einn saman, svā\\
mikinn hęr sem þēr hafið.  En ef yðr er nakkvarr grunr ā\\
þvī, at ū-friðr muni fyrir, þā skal ek fylgja yðr með mīnu\\
liði, ok þōtti þat styrkr vera fyrr, hvar sem Jōms-vīkingar\\
fylgðu hǫfðingjum; mun ek fā þēr ellifu skip vel skipuð.'\\
Konungr jātti þessu.  Var þā lītit veðr ok hag-stœtt; lēt\\
konungr þā leysa flotann, ok blāsa til brott-lǫgu.  Drōgu\\
męnn þā segl sīn, ok gengu meira smā-skipin ǫll, ok sigldu\\
þau undan ā haf ūt.  En jarl sigldi nær konungs-skipinu,\\
ok kallaði til þeira, bað konung sigla ęptir sēr: `mēr er\\
kunnast,' sęgir hann, `hvar djūpast er um eyja-sundin, en\\
þēr munuð þess þurfa með þau in stōru skipin.'  Sigldi\\
þā jarl fyrir með sīnum skipum.  Hann hafði ellifu skip,\\
en konungr sigldi ęptir honum með sinum stōr-skipum,\\
hafði hann þar ok ellifu skip, en allr annarr hęrrinn sigldi\\
ūt ā hafit.  En er Sigvaldi jarl sigldi utan at Svǫlðr, þā\\
röri ā mōti þeim skūta ein.  Þeir sęgja jarli at hęrr Dana-\\
konungs lā þar ī hǫfninni fyrir þeim.  Þā lēt jarl hlaða\\
seglunum, ok rōa þeir inn undir eyna.

Sveinn Danakonungr ok Ōlāfr Svīakonungr ok Eirīkr\\
jarl vāru þar þā með allan hęr sinn; þā var fagrt veðr\\
ok bjart sōl-skin.  Gengu þeir nū upp ā hōlminn allir\\
hǫfðingjar með miklar sveitir manna, ok sā er skipin sigldu\\
ūt ā hafit mjǫk mǫrg saman.  Ok nū sjā þeir hvar siglir\\
eitt mikit skip ok glæsiligt; þā mæltu bāðir konungarnir:\\
`þetta er mikit skip ok ākafliga fagrt, þetta mun vera Ormrinn\\
langi.'  Eirīkr jarl svarar ok sęgir: `ekki er þetta Ormr hinn\\
langi.'  Ok svā var sem hann sagði; þetta skip ātti Eindriði\\
af Gimsum.  Lītlu sīðar sā þeir hvar annat skip sigldi miklu\\
meira en hit fyrra.  Þā mælti Sveinn konungr: `hræddr er\\
Ōlāfr Tryggvason nū, eigi þorir hann at sigla með hǫfuðin\\
ā skipi sīnu.'  Þā sęgir Eirīkr jarl: `ekki er þetta konungs\\
skip, kęnni ek þetta skip ok seglit, þvī at stafat er seglit, þat\\
ā Erlingr Skjālgsson; lātum sigla þā, bętra er oss skarð ok\\
missa ī flota Ōlāfs konungs en þetta skip þar svā būit.'  En\\
stundu sīðar sā þeir ok kęndu skip Sigvalda jarls, ok viku\\
þan þannig at hōlmanum.  Þā sā þeir hvar sigldu þrjū skip,\\
ok var eitt mikit skip.  Mælti þā Sveinn konungr, biðr þā\\
ganga til skipa sinna, sęgir at þar fęrr Ormrinn langi.  Eirīkr\\
jarl mælti: `mǫrg hafa þeir ǫnnur stōr skip ok glæsilig en\\
Orm hinn langa, bīðum ęnn.'  Þā mæltu mjǫk margir męnn:\\
`eigi vill Eirīkr jarl nū bęrjask, ok hęfna fǫður sīns; þetta\\
er skǫmm mikil, svā at spyrjask mun um ǫll lǫnd, ef vēr\\
liggjum hēr með jafn-miklu liði, en Ōlāfr konungr sigli ā\\
hafit ūt hēr hjā oss sjālfum.'  En er þeir hǫfðu þetta talat\\
um hrīð, þā sā þeir hvar sigldu fjogur skip, ok eitt af\\
þeim var dręki all-mikill ok mjǫk gull-būinn.  Þā stōð upp\\
Sveinn konungr, ok mælti: `hātt mun Ormrinn bera mik ī\\
kveld, honum skal ek stȳra.'  Þā mæltu margir, at Ormrinn\\
var furðu mikit skip ok frītt, ok rausn mikil at lāta gǫ̈ra\\
slīkt skip.  Þā mælti Eirīkr jarl, svā at nakkvarir męnn\\
heyrðu: `þōtt Ōlāfr konungr hęfði ekki meira skip en þetta,\\
þā mundi Sveinn konungr þat aldri fā af honum með einn\\
saman Danahęr.'  Dreif þā fōlkit til skipanna, ok rāku af\\
tjǫldin, ok ætluðu at būask skjōtliga.  En er hǫfðingjar rœddu\\
þetta milli sīn, sem nū er sagt, þā sā þeir, hvar sigldu þrjū\\
skip all-mikil, ok fjōrða sīðast, ok var þat Ormrinn langi.\\
En þau hin stōru skip, er āðr hǫfðu siglt, ok þeir hugðu\\
at Ormrinn væri, þat var hit fyrra Traninn, en hit sīðara\\
Ormrinn skammi.  En þā er þeir sā Orminn langa, kęndu\\
allir, ok mælti þā engi ī mōt, at þar mundi sigla Ōlāfr\\
Tryggvason; gengu þā til skipanna, ok skipuðu til at-\\
lǫgunnar.  Vāru þat einkamāl þeira hǫfðingja, Sveins konungs,\\
Ōlāfs konungs, Eirīks jarls, at sinn þriðjung Noregs skyldi\\
eignask hvęrr þeira, ef þeir fęldi Ōlāf konung Tryggvason;\\
en sā þeira hǫfðingja er fyrst gengi ā Orminn, skyldi eignask\\
alt þat hlut-skipti er þar fengisk, ok hvęrr þeira þau skip\\
er sjālfr hryði.  Eirīkr jarl hafði barða einn geysi mikinn,\\
er hann var vanr at hafa ī viking; þar var skęgg ā ofan-\\
verðu barðinu hvārutvęggja, en niðr frā jārn-spǫng þykk ok\\
svā breið sem barðit, ok tōk alt ī sæ ofan.

Þā er þeir Sigvaldi jarl röru inn undir hōlminn, þā sā\\
þat þeir Þorkęll dyðrill af Trananum ok aðrir skip-stjōrn-\\
ar-męnn, þeir er með honum fōru, at jarl snöri skipum\\
undir hōlmann; þā hlōðu þeir ok seglum, ok röru ęptir\\
honum, ok kǫlluðu til þeira, spurðu, hvī þeir fōru svā.  Jarl\\
sęgir, at hann vill bīða Ōlāfs konungs: `ok er meiri vān at\\
ūfriðr sē fyrir oss.'  Lētu þeir þā fljōta skipin, þar til er\\
Þorkęll nęfja kom með Orminn skamma, ok þau þrjū skip\\
er honum fylgðu.  Ok vāru þeim sǫgð hin sǫmu tīðindi;\\
hlōðu þeir þā ok sīnum seglum, ok lētu fljōta, ok biðu\\
Ōlāfs konungs.  En þā er konungrinn sigldi innan at hōl-\\
manum, þā röri allr hęrrinn ūt ā sundit fyrir þā.  En er\\
þeir sā þat, þa bāðu þeir konunginn sigla leið sīna, en\\
lęggja eigi til orrostu við svā mikinn hęr.  Konungr svarar\\
hātt, ok stōð upp ī lyptingunni: `lāti ofan seglit, ekki skulu\\
mīnir męnn hyggja ā flōtta, ek hęfi aldri flȳit ī orrostu, rāði\\
Guð fyrir līfi mīnu, en aldri mun ek ā flōtta lęggja.'  Var svā\\
gǫ̈rt sem konungr mælti.

Ōlāfr konungr lēt blāsa til sam-lǫgu ǫllum skipum sīnum.\\
Var konungs skip ī miðju liði, en þar ā annat borð Ormrinn\\
skammi, en ā annat borð Traninn.  En þā er þeir tōku\\
at tęngja stafna ā Orminum langa ok Orminum skamma,\\
ok er konungr sā þat, kallaði hann hātt, bað þā lęggja\\
fram bętr hit mikla skipit, ok lāta þat eigi aptast vera allra\\
skipa ī hęrinum.  Þā svarar Ūlfr hinn rauði: `ef Orminn\\
skal þvī lęngra fram lęggja, sem hann er lęngri en ǫnnur\\
skip, þā mun ā-vint verða um sǫxin ī dag.'  Konungr sęgir:\\
`eigi vissa ek at ek ætta stafnbūann bæði rauðan ok ragan.'\\
Ūlfr mælti: `vęr þū eigi meir baki lyptingina en ek mun\\
stafninn.'  Konungr helt ā boga, ok lagði ǫr ā stręng, ok\\
snöri at Ūlfi.  Ūlfr mælti: `skjōt annan veg, konungr ī þannig\\
sem meiri er þǫrfin; þēr vinn ek þat er ek vinn.'

Ōlāfr konungr stōð ī lyptingu ā Orminum, bar hann hātt\\
mjǫk; hann hafði gyltan skjǫld ok gull-roðinn hjālm; var\\
hann auð-kęndr frā ǫðrum mǫnnum: hann hafði rauðan\\
kyrtil stuttan utan yfir brynju.  En er Ōlāfr konungr sā at\\
riðluðusk flotarnir, ok upp vāru sętt męrki fyrir hǫfðingjum,\\
þā spyrr hann: `hvęrr er hǫfðingi fyrir liði þvī er gęgnt\\
oss er?'  Honum var sagt at þar var Sveinn konungr\\
tjūguskęgg með Danahęr.  Konungr svarar: `ekki hræðumk\\
vēr bleyður þær, engi er hugr ī Dǫnum.  En hvęrr hǫfðingi\\
fylgir þeim męrkjum er þar eru ūt īfrā ā hœgra veg?'  Honum\\
var sagt at þar var Ōlāfr konungr með Svīa-hęr.  Ōlāfr\\
konungr sęgir: `bętra væri Svīum heima at sleikja um blōt-\\
bolla sīna en ganga ā Orminn undir vāpn yður.  En hvęrir\\
eigu þau hin stōru skip, er þar liggja ūt ā bak-borða Dǫnum?'\\
`Þar er,' sęgja þeir `Eirīkr jarl Hākonar-son.'  Þā svaraði\\
Ōlāfr konungr: `hann mun þykkjask eiga við oss skapligan\\
fund, ok oss er vān snarpligrar orrostu af þvī liði; þeir eru\\
Norð-męnn, sem vēr erum.'

Sīðan greiða konungar at-rōðr.  Lagði Sveinn konungr\\
sitt skip mōti Orminum langa, en Ōlāfr konungr Sœnski\\
lagði ūt frā, ok stakk stǫfnum at yzta skipi Ōlāfs konungs\\
Tryggvasonar, en ǫðrum megin Eirīkr jarl.  Tōksk þar þā\\
hǫrð orrosta.  Sigvaldi jarl lēt skotta við sīn skip, ok lagði\\
ekki til orrostu.

Þessi orrosta var hin snarpasta ok all-mann-skœð.  Fram-\\
byggjar ā Orminum langa ok Orminum skamma ok Trananum\\
fœrðu akkeri ok stafn-ljā ī skip Sveins konungs, en āttu\\
vāpnin at bera niðr undir fœtr sēr; hruðu þeir ǫll þau skip\\
er þeir fengu haldit.  En konungrinn Sveinn ok þat lið er\\
undan komsk flȳði ā ǫnnur skip, ok þar næst lǫgðu þeir\\
frā ōr skot-māli.  Ok fōr þessi hęrr svā sem gat Ōlāfr\\
konungr Tryggvason.  Þā lagði þar at ī staðinn Ōlāfr\\
Svīakonungr; ok þegar er þeir koma nær stōrskipum, þā\\
fōr þeim sem hinum, at þeir lētu lið mikit ok sum skip sīn,\\
ok lǫgðu frā við svā būit.  En Eirīkr jarl sī-byrði Barðanum\\
við hit yzta skip Ōlāfs konungs, ok hrauð hann þat, ok hjō\\
þegar þat ōr tęngslum, en lagði þā at þvī, er þar var næst,\\
ok barðisk til þess er þat var hroðit.  Tōk þā liðit at hlaupa\\
af hinum smærum skipunum, ok upp ā stōrskipin.  En Eirīkr\\
jarl hjō hvęrt ōr tęngslunum, svā sem hroðit var.  En Danir\\
ok Svīar lǫgðu þā ī skotmāl ok ǫllum megin at skipum Ōlāfs\\
konungs, en Eirīkr jarl lā āvalt sībyrt við skipin, ok ātti\\
hǫgg-orrostu.  En svā sem męnn fellu ā skipum hans, þā\\
gengu aðrir upp ī staðinn, Svīar ok Danir.  Þā var orrosta\\
hin snarpasta, ok fell þā mjǫk liðit, ok kom svā at lykðum,\\
at ǫll vāru hroðin skip Ōlāfs konungs Tryggvasonar nema\\
Ormrinn langi; var þar þā alt lið ā komit, þat er vīgt var\\
hans manna.  Þā lagði Eirīkr jarl Barðanum at Orminum\\
langa sībyrt, ok var þar hǫggorrosta.

Eirlīkr jarl var ī fyrir-rūmi ā skipi sīnu, ok var þar fylkt\\
með skjald-borg.  Var þā bæði hǫggorrosta, ok spjōtum lagit,\\
ok kastat ǫllu þvī er til vāpna var, en sumir skutu boga-skoti\\
eða hand-skoti.  Var þa svā mikill vāpnaburðr ā Orminn, at\\
varla mātti hlīfum við koma, er svā þykt flugu spjōt ok ǫrvar;\\
þvī at ǫllum megin lǫgðu hęrskip at Orminum.  En męnn\\
Ōlāfs konungs vāru þā svā ōðir, at þeir hljōpu upp ā borðin,\\
til þess at nā með sverðs-hǫggum at drepa fōlkit.  En margir\\
lǫgðu eigi svā undir Orminn, at þeir vildi ī hǫggorrostu vera.\\
En Ōlāfs męnn gengu flestir ūt af borðunum, ok gāðu eigi\\
annars en þeir bęrðisk ā slēttum vęlli, ok sukku niðr með\\
vāpnum sīnum.

Einarr þambar-skęlfir var ā Orminum aptr ī krappa-rūmi;\\
hann skaut af boga, ok var allra manna harð-skeytastr.\\
Einarr skaut at Eirīki jarli, ok laust ī stȳris-hnakkann fyrir\\
ofan hǫfuð jarli, ok gekk alt upp ā reyr-bǫndin.  Jarl leit til,\\
ok spurði ef þeir vissi, hvęrr skaut.  En jafn-skjōtt kom\\
ǫnnur ǫr svā nær jarli, at flaug milli sīðunna ok handarinnar,\\
ok svā aptr ī hǫfða-fjǫlina, at langt stōð ūt broddrinn.  Þā\\
mælti jarl við mann þann er sumir nęfna Finn, en sumir\\
sęgja at hann væri Finskr, sā var hinn mesti bog-maðr:\\
`skjōt-tu mann þann hinn mikla ī krapparūminu!'  Finnr\\
skaut, ok kom ǫrin ā boga Einars miðjan, ī þvī bili er Einarr\\
drō it þriðja sinn bogann.  Brast þā boginn ī tvā hluti.  Þā\\
mælti Ōlāfr konungr: `hvat brast þar svā hātt?'  Einarr\\
svarar: `Noregr ōr hęndi þēr, konungr!'  `Eigi mun svā\\
mikill brestr at orðinn,' sęgir konungr, `tak boga minn, ok\\
skjōt af,' ok kastaði boganum til hans.  Einarr tōk bogann,\\
ok drō þegar fyrir odd ǫrvarinnar, ok mælti: `ofveikr,\\
ofveikr allvalds boginn!' ok kastaði aptr boganum; tōk þā\\
skjǫld sinn ok sverð, ok barðisk.

Ōlāfr konungr Tryggvason stōð ī lypting ā Orminum, ok\\
skaut optast um daginn, stundum bogaskoti, en stundum\\
gaflǫkum, ok jafnan tveim sęnn.  Hann sā fram ā skipit, ok\\
sā sīna męnn reiða sverðin ok hǫggva tītt, ok sā at illa bitu;\\
mælti þā hātt: `hvārt reiði þēr svā slæliga sverðin, er ek sē\\
at ekki bīta yðr?'  Maðr svarar: `sverð vār eru slæ ok\\
brotin mjǫk.'  Þā gekk konungr ofan ī fyrirrūmit ok lauk\\
upp hāsætis-kistuna, tōk þar ōr mǫrg sverð hvǫss, ok fekk\\
mǫnnum.  En er hann tōk niðr hinni hœgri hęndi, þā sā\\
męnn at blōð rann ofan undan bryn-stūkunni; en engi vissi\\
hvar hann var sārr.

Mest var vǫrnin ā Orminum ok mannskœðust af fyrirrūms-\\
mǫnnum ok stafnbūum; þar var hvārttvęggja, valit mest\\
mann-fōlkit ok hæst borðin.  En lið fell fyrst um mitt skipit.\\
Ok þā er fātt stōð manna upp um siglu-skeið, þā rēð Eirīkr\\
jarl til upp-gǫngunnar, ok kom upp ā Orminn við fimtānda\\
mann.  Þā kom ī mōt honum Hyrningr, māgr Ōlāfs konungs,\\
með sveit manna, ok varð þar inn harðasti bardagi, ok lauk\\
svā, at jarl hrǫkk ofan aptr ā Barðann; en þeir męnn er\\
honum hǫfðu fylgt fellu sumir, en sumir vāru særðir.  Þar\\
varð ęnn in snarpasta orrosta, ok fellu þā margir męnn ā\\
Orminum.  En er þyntisk skipan ā Orminum til varnarinnar,\\
þā rēð Eirīkr jarl annat sinn til uppgǫngu ā Orminn.  Varð\\
þā ęnn hǫrð við-taka.  En er þetta sā stafnbūar ā Orminum,\\
þā gengu þeir aptr ā skipit, ok snūask til varnar mōti jarli, ok\\
veita harða viðtǫku.  En fyrir þvī at þā var svā mjǫk fallit\\
lið ā Orminum, at vīða vāru auð borðin, tōku þā jarls męnn\\
vīða upp at ganga.  En alt þat lið er þā stōð upp til varnar\\
ā Orminum sōtti aptr ā skipit, þar sem konungr var.

Kolbjǫrn stallari gekk upp ī lypting til konungs; þeir\\
hǫfðu mjǫk līkan klæða-būnað ok vāpna, Kolbjǫrn var ok\\
allra manna mestr ok frīðastr.  Varð nū ęnn ī fyrirrūminu\\
in snarpasta orrosta.  En fyrir þā sǫk at þā var svā mikit\\
fōlk komit upp ā Orminn af liði jarls sem vera mātti ā skipinu,\\
en skip hans lǫgðu at ǫllum megin utan at Orminum, en\\
lītit fjǫl-męnni til varnar mōti svā miklum hęr, nū þōtt þeir\\
męnn væri bæði stęrkir ok frœknir, þā fellu nū flestir ā lītilli\\
stundu.  En Ōlāfr konungr sjālfr ok þeir Kolbjǫrn bāðir\\
hljōpu þā fyrir borð, ok ā sitt borð hvārr.  En jarls męnn\\
hǫfðu lagt utan at smā-skūtur, ok drāpu þā er ā kaf hljōpu.\\
Ok þā er konungr sjālfr hafði ā kaf hlaupit, vildu þeir taka\\
hann hǫndum, ok fœra Eirīki jarli.  En Ōlāfr konungr brā\\
yfir sik skildinum, ok steypðisk ī kaf; en Kolbjǫrn stallari\\
skaut undir sik skildinum, ok hlīfði sēr svā við vāpnum er\\
lagt var af skipum þeim er undir lāgu, ok fell hann svā\\
ā sæinn at skjǫldrinn varð undir honum, ok komsk hann þvī\\
eigi ī kaf svā skjōtt, ok varð hann hand-tękinn ok dręginn\\
upp ī skūtuna, ok hugðu þeir at þar væri konungrinn.  Var\\
hann þā leiddr fyrir jarl.  En er þess varð jarl varr at þar\\
var Kolbjǫrn, en eigi Ōlāfr konungr, þā vāru Kolbirni grið\\
gefin.  En ī þessi svipan hljōpu allir fyrir borð af Orminum,\\
þeir er þā vāru ā līfi, Ōlāfs konungs męnn; ok sęgir Hall-\\
freðr vandræða-skāld, at Þorkęll nęfja, konungs brōðir, hljōp\\
sīðast allra manna fyrir borð.

Svā var fyrr ritat, at Sigvaldi jarl kom til fǫruneytis við\\
Ōlāf konung ī Vindlandi, ok hafði tīu skip, en þat hit ellifta,\\
er ā vāru męnn Āstrīðar konungs-dōttur, konu jarls.  En\\
þā er Ōlāfr konungr hafði fyrir borð hlaupit, þā œpði\\
hęrrinn allr sigr-ōp, ok þā lustu þeir ārum ī sæ Sigvaldi\\
jarl ok hans męnn, ok röru til bardaga.  En sū Vinda-\\
snękkjan, er Āstrīðar męnn vāru ā, röri brott ok aptr undir\\
Vindland; ok var þat margra manna māl þegar, at Ōlāfr\\
konungr mundi hafa steypt af sēr brynjunni ī kafi, ok kafat\\
svā ūt undan langskipunum, lagizk sīðan til Vindasnękkj-\\
unnar, ok hęfði męnn Āstrīðar flutt hann til lands.  Ok\\
eru þar margar frā-sagnir um fęrðir Ōlāfs konungs gǫ̈rvar\\
sīðan af sumum mǫnnum.  En hvęrn veg sem þat hęfir\\
verit, þā kom Ōlāfr konungr Tryggvason aldri sīðan til rīkis\\
ī Noregi.

\end{linenumbers}

\emptypage

\chapter{Auðun}

\resetlinenumber
\begin{linenumbers}

Maðr hēt Auðun, Vest-firzkr at kyni ok fē-lītill; hann fōr\\
utan vestr þar ī fjǫrðum með um-rāði Þorsteins bōnda gōðs,\\
ok Þōris stȳri-manns, er þar hafði þegit vist of vetrinn með\\
Þorsteini.  Auðun var ok þar, ok starfaði fyrir honum Þōri,\\
ok þā þessi laun af honum---utan-fęrðina ok hans um-sjā.\\
Hann Auðun lagði mestan hluta fjār þess er var fyrir mōður\\
sīna, āðr hann stigi ā skip, ok var kveðit ā þriggja vetra\\
bjǫrg.  Ok nū fara þeir utan heðan, ok fęrsk þeim vel, ok\\
var Auðun of vetrinn ęptir með Þōri stȳrimanni; hann ātti\\
bū ā Mœri.  Ok um sumarit ęptir fara þeir ūt til Grœn-lands,\\
ok eru þar of vetrinn.  Þess er við getit at Auðun kaupir\\
þar bjarn-dȳri eitt, gǫ̈rsimi mikla, ok gaf þar fyrir alla\\
eigu sīna.  Ok nū of sumarit ęptir þā fara þeir aptr til\\
Noregs, ok verða vel reið-fara; hęfir Auðun dȳr sitt með\\
sēr, ok ætlar nū at fara suðr til Danmęrkr ā fund Sveins\\
konungs, ok gefa honum dȳrit.  Ok er hann kom suðr ī\\
landit, þar sem konungr var fyrir, þā gęngr hann upp af\\
skipi, ok leiðir ęptir sēr dȳrit, ok leigir sēr hęr-bęrgi.  Haraldi\\
konungi var sagt brātt at þar var komit bjarndȳri, gǫ̈rsimi\\
mikil, `ok ā Īs-lęnzkr maðr.'  Konungr sęndir þegar męnn\\
ęptir honum, ok er Auðun kom fyrir konung, kvęðr hann\\
konung vel; konungr tōk vel kvęðju hans, ok spurði sīðan:\\
`āttu gǫ̈rsimi mikla ī bjarndȳri?'  Hann svarar, ok kvezk\\
eiga dȳrit eitthvęrt.  Konungr mælti: `villtu sęlja oss dȳrit\\
við slīku verði sem þū keyptir?'  Hann svarar: `eigi vil ek\\
þat, herra!'  `Villtu þā,' sęgir konungr, `at ek gefa þēr tvau\\
verð slīk, ok mun þat rēttara, ef þū hęfir þar við gefit alla\\
þīna eigu.'  `Eigi vil ek þat, herra!' sęgir hann.  Konungr\\
mælti: `villtu gefa mēr þā?'  Hann svarar: `eigi, herra!'\\
Konungr mælti: `hvat villtu þā af gǫ̈ra?'  Hann svarar:\\
`fara,' sęgir hann, `til Danmęrkr, ok gefa Sveini konungi.'\\
Haraldr konungr sęgir: `hvārt er, at þū ert maðr svā ūvitr\\
at þū hęfir eigi heyrt ūfrið þann er ī milli er landa þessa,\\
eða ætlar þū giptu þīna svā mikla, at þū munir þar komask\\
með gǫ̈rsimar, er aðrir fā eigi komizk klakk-laust, þō at\\
nauð-syn eigi til?'  Auðun svarar: `herra! þat er ā yðru\\
valdi, en engu jātum vēr ǫðru en þessu er vēr hǫfum āðr\\
ætlat.'  Þā mælti konungr: `hvī mun eigi þat til, at þū farir\\
leið þīna, sem þū vill, ok kom þā til mīn, er þū fęrr aptr,\\
ok sęg mēr, hvęrsu Sveinn konungr launar þēr dȳrit, ok\\
kann þat vera, at þū sēr gæfu-maðr.'  `Þvī heit ek þēr,'\\
sagði Auðun.

Hann fęrr nū sīðan suðr með landi, ok ī Vīk austr, ok þā\\
til Danmęrkr; ok er þā uppi hvęrr pęnningr fjārins, ok verðr\\
hann þā biðja matar bæði fyrir sik ok fyrir dȳrit.  Hann\\
kömr ā fund ār-manns Sveins konungs, þess er Āki hēt,\\
ok bað hann vista nakkvarra bæði fyrir sik ok fyrir dȳrit:\\
`ek ætla,' sęgir hann, `at gefa Sveini konungi dȳrit.'  Āki\\
lēzk sęlja mundu honum vistir, ef hann vildi.  Auðun kvezk\\
ekki til hafa fyrir at gefa; `en ek vilda þō,' sęgir hann, `at\\
þetta kvæmisk til leiðar at ek mætta dȳrit fœra konungi.'\\
`Ek mun fā þēr vistir, sem it þurfið til konungs fundar;\\
en þar ī mōti vil ek eiga hālft dȳrit, ok māttu ā þat līta,\\
at dȳrit mun deyja fyrir þēr, þars it þurfuð vistir miklar, en\\
fē sē farit, ok er būit við at þū hafir þā ekki dȳrsins.'  Ok\\
er hann lītr ā þetta, sȳnisk honum nakkvat ęptir sem\\
ārmaðrinn mælti fyrir honum, ok sættask þeir ā þetta, at\\
hann sęlr Āka hālft dȳrit, ok skal konungr sīðan meta alt\\
saman.  Skulu þeir fara bāðir nū ā fund konungs; ok svā\\
gǫ̈ra þeir: fara nū bāðir ā fund konungs, ok stōðu fyrir\\
borðinu.  Konungr īhugāði, hvęrr þessi maðr myndi vera,\\
er hann kęndi eigi, ok mælti sīðan til Auðunar: `hvęrr\\
er-tu?' sęgir hann.  Hann svarar: `ek em Īslęnzkr maðr,\\
herra,' sęgir hann, `ok kominn nū utan af Grœnlandi, ok nū\\
af Noregi, ok ætlaða-k at fœra yðr bjarndȳri þetta; keypta-k\\
þat með allri eigu minni, ok nū er þō ā orðit mikit fyrir\\
mēr; ek ā nū hālft eitt dȳrit,' ok sęgir konungi sīðan, hvęrsu\\
farit hafði með þeim Āka ārmanni hans.  Konungr mælti:\\
`er þat satt, Āki, er hann sęgir?'  `Satt er þat,' sęgir hann.\\
Konungr mælti: `ok þōtti þēr þat til liggja, þar sem ek\\
sętta-k þik mikinn mann, at hępta þat eða tālma er maðr\\
gǫ̈rðisk til at fœra mēr gǫ̈rsimi, ok gaf fyrir alla eign, ok\\
sā þat Haraldr konungr at rāði at lāta hann fara ī friði, ok er\\
hann vārr ūvinr?  Hygg þū at þā, hvē sannligt þat var þinnar\\
handar, ok þat væri makligt, at þū værir drepinn; en ek\\
mun nū eigi þat gǫ̈ra, en braut skaltu fara þegar ōr landinu,\\
ok koma aldri aptr sīðan mēr ī aug-sȳn!  En þēr, Auðun!\\
kann ek slīka þǫkk, sem þū gefir mēr alt dȳrit, ok ver hēr\\
með mēr.'  Þat þękkisk hann, ok er með Sveini konungi\\
um hrīð.

Ok er liðu nakkvarir stundir, þā mælti Auðun við konung:\\
`braut fȳsir mik nū, herra!'  Konungr svarar hęldr seint:\\
`hvat villtu þā,' sęgir hann, `ef þū vill eigi með oss vera?'\\
Hann sęgir: `suðr vil ek ganga.'  `Ef þū vildir eigi svā gott\\
rāð taka,' sęgir konungr, `þā myndi mēr fyrir þykkja ī, er þū\\
fȳsisk ī braut'; ok nū gaf konungr honum silfr mjǫk mikit,\\
ok fōr hann suðr sīðan með Rūm-fęrlum, ok skipaði konungr\\
til um fęrð hans, bað hann koma til sīn, er kvæmi aptr.\\
Nū fōr hann fęrðar sinnar, unz hann kömr suðr ī Rōma-borg.\\
Ok er hann hęfir þar dvalizk, sem hann tīðir, þā fęrr hann\\
aptr; tękr þā sōtt mikla, gǫ̈rir hann þā ākafliga magran;\\
gęngr þā upp alt fēit þat, er konungr hafði gefit honum\\
til fęrðarinnar; tękr sīðan upp staf-karls stīg, ok biðr sēr\\
matar.  Hann er þā kollōttr ok hęldr ū-sælligr; hann kömr\\
aptr ī Danmörk at pāskum, þangat sem konungr er þā\\
staddr; en ei þorði hann at lāta sjā sik; ok var ī kirkju-\\
skoti, ok ætlaði þā til fundar við konung, er hann gengi\\
til kirkju um kveldit; ok nū er hann sā konunginn ok\\
hirðina fagrliga būna, þā þorði hann eigi at lāta sjā sik.\\
Ok er konungr gekk til drykkju ī hǫllina, þā mataðisk Auðun\\
ūti, sem siðr er til Rūmfęrla, meðan þeir hafa eigi kastat staf\\
ok skreppu.  Ok nū of aptaninn, er konungr gekk til kveld-\\
sǫngs, ætlaði Auðun at hitta hann, ok svā mikit sem honum\\
þōtti fyrr fyrir, jōk nū miklu ā, er þeir vāru druknir\\
hirðmęnninir; ok er þeir gengu inn aptr, þā þękði konungr\\
mann, ok þōttisk finna at eigi hafði frama til at ganga fram\\
at hitta hann.  Ok nū er hirðin gekk inn, þā veik konungr\\
ūt, ok mælti: `gangi sā nū fram, er mik vill finna; mik\\
grunar at sā muni vera maðrinn.'  Þā gekk Auðun fram,\\
ok fell til fōta konungi, ok varla kęndi konungr hann; ok\\
þegar er konungr veit, hvęrr hann er, tōk konungr ī hǫnd\\
honum Auðuni, ok bað hann vel kominn, `ok hęfir þū mikit\\
skipazk,' segir hann, `sīðan vit sāmk'; leiðir hann ęptir sēr\\
inn, ok er hirðin sā hann, hlōgu þeir at honum; en konungr\\
sagði: `eigi þurfu þēr at honum at hlæja, þvī at bętr hęfir\\
hann sēt fyr sinni sāl hęldr en ēr.'  Þā lēt konungr gǫ̈ra\\
honum laug, ok gaf honum sīðan hlæði, ok er hann nū með\\
honum.  Þat er nū sagt einhvęrju sinni of vārit at konungr\\
bȳðr Auðuni at vera með sēr ā-lęngðar, ok kvezk myndu\\
gǫ̈ra hann skutil-svein sinn, ok lęggja til hans gōða virðing.\\
Auðun sęgir: `Guð þakki yðr, herra! sōma þann allan er\\
þēr vilið til mīn lęggja; en hitt er mēr ī skapi at fara ūt\\
til Īslands.'  Konungr sęgir: `þetta sȳnisk mēr undarliga\\
kosit.'  Auðun mælti: `eigi mā ek þat vita, herra!' sęgir\\
hann, `at ek hafa hēr mikinn sōma með yðr, en mōðir mīn\\
troði stafkarls stīg ūt ā Īslandi; þvī at nū er lokit bjǫrg þeiri\\
er ek lagða til, āðr ek fœra af Īslandi.'  Konungr svarar:\\
`vel er mælt,' sęgir hann, `ok mannliga, ok muntu verða\\
giptu-maðr; þessi einn var svā hlutrinn, at mēr myndi eigi\\
mis-līka at þū fœrir ī braut heðan; ok ver nū með mēr þar til\\
er skip būask.'  Hann gǫ̈rir svā.

Einn dag, er ā leið vārit, gekk Sveinn konungr ofan ā\\
bryggjur, ok vāru męnn þā at, at būa skip til ȳmissa landa,\\
ī austr-veg eða Sax-land, til Svīþjōðar eða Noregs.  Þā koma\\
þeir Auðun at einu skipi fǫgru, ok vāru męnn at, at būa\\
skipit.  Þā spurði konungr: `hvęrsu līzk þēr, Auðun! ā\\
þetta skip?'  Hann svarar: `vel, herra!'  Konungr mælti:\\
`þetta skip vil ek þēr gefa, ok launa bjarndȳrit.'  Hann\\
þakkaði gjǫfina ęptir sinni kunnustu; ok er leið stund, ok\\
skipit var albūit, þā mælti Sveinn konungr við Auðun: `þō\\
villtu nū ā braut, þā mun ek nū ekki lętja þik, en þat hęfi ek\\
spurt, at ilt er til hafna fyrir landi yðru, ok eru vīða öræfi ok\\
hætt skipum; nū brȳtr þū, ok tȳnir skipinu ok fēnu; lītt sēr\\
þat þā ā, at þū hafir fundit Svein konung, ok gefit honum\\
gǫ̈rsimi.'  Sīðan sęldi konungr honum leðr-hosu fulla af\\
silfri, `ok ertu þā ęnn eigi fē-lauss með ǫllu, þōtt þū brjōtir\\
skipit, ef þū fær haldit þessu.  Verða mā svā ęnn,' sęgir\\
konungr, `at þū tȳnir þessu fē; lītt nȳtr þū þā þess, er þū\\
fannt Svein konung, ok gaft honum gǫ̈rsimi.'  Sīðan drō\\
konungr hring af hęndi sēr, ok gaf Auðuni, ok mælti: `þō\\
at svā illa verði, at þū brjōtir skipit ok tȳnir fēnu, eigi\\
ertu fēlauss, ef þū kömsk ā land, þvī at margir męnn hafa\\
gull ā sēr ī skips-brotum, ok sēr þā at þū hęfir fundit Svein\\
konung, ef þū hęldr hringinum; en þat vil ek rāða þēr,'\\
sęgir hann, `at þū gefir eigi hringinn, nema þū þykkisk eiga\\
svā mikit gott at launa nǫkkurum gǫfgum manni, þā gef\\
þeim hringinn, þvī at tignum mǫnnum sōmir at þiggja, ok\\
far nū heill!'

Sīðan lætr hann ī haf, ok kömr ī Noreg, ok lætr flytja\\
upp varnað sinn, ok þurfti nū meira við þat en fyrr, er\\
hann var ī Noregi.  Hann fęrr nū sīðan ā fund Haralds\\
konungs, ok vill ęfna þat er hann hēt honum, āðr hann\\
fōr til Danmęrkr, ok kvęðr konung vel.  Haraldr konungr\\
tōk vel kvęðju hans, ok `sęzk niðr,' sęgir hann, `ok drekk\\
hēr með oss'; ok svā gǫ̈rir hann.  Þā spurði Haraldr kon-\\
ungr: `hvęrju launaði Sveinn konungr þēr dȳrit?'  Auðun\\
svarar: `þvī, herra! at hann þā at mēr.'  Konungr sagði:\\
`launat mynda ek þēr þvī hafa; hvęrju launaði hann ęnn?'\\
Auðun svarar: `gaf hann mēr silfr til suðr-gǫngu.'  Þā sęgir\\
Haraldr konungr: `mǫrgum mǫnnum gefr Sveinn konungr\\
silfr til suðrgǫngu eða annarra hluta, þōtt ekki fœri honum\\
gǫ̈rsimar; hvat er ęnn fleira?'  `Hann bauð mēr,' sęgir\\
Auðun, `at görask skutilsveinn hans, ok mikinn sōma til\\
mīn at lęggja.'  `Vel var þat mælt,' sęgir konungr, `ok\\
launa, myndi hann ęnn fleira.'  Auðun sagði: `gaf hann mēr\\
knǫrr með farmi þeim er hingat er bęzt varit ī Noreg.'  `Þat\\
var stōr-mannligt,' sęgir konungr, `en launat mynda ek þēr\\
þvī hafa.  Launaði hann þvī fleira?'  Auðun svaraði: `gaf\\
hann mēr leðrhosu fulla af silfri, ok kvað mik þā eigi fēlausan,\\
ef ek helda þvī, þō at skip mitt bryti við Īsland.'  Konungr\\
sagði: `þat var ā-gætliga gǫ̈rt, ok þat mynda ek ekki gǫ̈rt\\
hafa; lauss mynda ek þykkjask, ef ek gæfa þer skipit; hvārt\\
launaði hann fleira?'  `Svā var vīst, herra!' sęgir Auðun,\\
`at hann launaði: hann gaf mēr hring þenna er ek hęfi\\
ā hęndi, ok kvað svā mega at berask, at ek tȳnda fēnu\\
ǫllu, ok sagði mik þā eigi fēlausan, ef ek ætta hringinn,\\
ok bað mik eigi lōga, nema ek ætta nǫkkurum tignum manni\\
svā gott at launa, at ek vilda gefa; en nū hęfi ek þann\\
fundit, þvī at þū āttir kost at taka hvārttvęggja frā mēr,\\
dȳrit ok svā līf mitt, en þū lēzt mik fara þangat ī friði,\\
sem aðrir nāðu eigi.'  Konungr tōk við gjǫfinni með blīðu,\\
ok gaf Auðuni ī mōti gōðar gjafir, āðr en þeir skilðisk.\\
Auðun varði fēnu til Īslands-fęrðar ok fōr ūt þegar um\\
sumarit til Īslands, ok þōtti vera inn mesti gæfumaðr.

\end{linenumbers}

\emptypage

\chapter{Þrymskviða}

\resetlinenumber
\begin{linenumbers}

\begin{verse}
1. Vreiðr var þā Ving-þōrr,\gap er hann vaknaði,\\
ok sīns hamars\gap of saknaði:\\
skęgg nam at hrista,\gap skǫr nam at dȳja,\\
rēð Jarðar burr\gap um at þreifask.

2. Ok hann þat orða\gap alls fyrst of kvað:\\
`heyr-ðu nū, Loki!\gap hvat ek nū mæli,\\
er engi veit\gap jarðar hvęrgi\\
nē upp-himins:\gap āss er stolinn hamri!'

3. Gengu þeir fagra\gap Freyju tūna,\\
ok hann þat orða\gap alls fyrst of kvað:\\
`muntu mēr, Freyja!\gap fjaðr-hams ljā,\\
ef ek minn hamar\gap mætta-k hitta?'
\end{verse}

\noindent Freyja kvað:

\begin{verse}
4. `Þō munda-k gefa þēr,\gap þōtt ōr gulli væri,\\
ok-þō sęlja\gap at væri ōr silfri.'

5. Flō þā Loki,\gap fjaðrhamr dunði,\\
unz fyr utan kom\gap āsa garða,\\
ok fyr innan kom\gap jǫtna heima.

6. Þrymr sat ā haugi,\gap þursa drōttinn,\\
greyjum sīnum\gap gull-bǫnd snöri\\
ok mǫrum sīnum\gap mǫn jafnaði.
\end{verse}

\noindent Þrymr kvað:

\begin{verse}
7. `Hvat er með āsum?\gap hvat er með ālfum?\\
hvī er-tu einn kominn\gap ī Jǫtunheima?'
\end{verse}

\noindent Loki kvað:

\begin{verse}
`Ilt er með āsum,\gap ilt er með ālfum;\\
hęfir þū Hlō-riða\gap hamar of fōlginn?'
\end{verse}

\noindent Þrymr kvað:

\begin{verse}
8. `Ek hęfi Hlōriða\gap hamar of fōlginn\\
ātta rǫstum\gap fyr jǫrð neðan;\\
hann engi maðr\gap aptr of heimtir,\\
nema fœri mēr\gap Freyju at kvān.'

9. Flō þā Loki,\gap fjaðrhamr dunði,\\
unz fyr utan kom\gap jǫtna heima\\
ok fyr innan kom\gap āsa garða;\\
mœtti hann Þōr\gap miðra garða,\\
ok hann þat orða\gap alls fyrst of kvað:

10. `Hęfir þū örindi\gap sem ęrfiði?\\
sęg-ðu ā lopti\gap lǫng tīðindi:\\
opt sitjanda\gap sǫgur of fallask,\\
ok liggjandi\gap lygi of bęllir.'
\end{verse}

\noindent Loki kvað:

\begin{verse}
11. `Hęfi-k ęrfiði\gap ok örindi:\\
Þrymr hęfir þinn hamar,\gap þursa drōttinn;\\
hann engi maðr\gap aptr of heimtir,\\
nema honum fœri\gap Freyju at kvān.'

12. Ganga þeir fagra\gap Freyju at hitta,\\
ok hann þat orða\gap alls fyrst of kvað:\\
`bitt-u þik, Freyja,\gap brūðar līni!\\
vit skulum aka tvau\gap ī Jǫtunheima.'

13. Vreið varð þā Freyja\gap ok fnāsaði,\\
allr āsa salr\gap undir bifðisk,\\
stǫkk þat it mikla\gap męn Brīsinga:\\
`mik veizt-u verða\gap ver-gjarnasta,\\
ef ek ęk með þēr\gap ī Jǫtunheima.'

14. Sęnn vāru æsir\gap allir ā þingi\\
ok āsynjur\gap allar ā māli,\\
ok of þat rēðu\gap rīkir tīvar,\\
hvē þeir Hlōriða\gap hamar of sœtti.

15. Þā kvað þat Heimdallr,\gap hvītastr āsa\\
(vissi hann vel fram,\gap sem vanir aðrir):\\
`bindum vēr Þōr þā\gap brūðar līni,\\
hafi hann it mikla\gap męn Brīsinga!

16. Lātum und honum\gap hrynja lukla\\
ok kvenn-vāðir\gap of knē falla,\\
en ā brjōsti\gap breiða steina,\\
ok hagliga\gap of hǫfuð typpum!'

17. Þā kvað þat Þōrr,\gap þrūðugr āss:\\
`mik munu æsir\gap argan kalla,\\
ef ek bindask læt\gap brūðar līni.'

18. Þā kvað þat Loki,\gap Laufeyjar sonr:\\
`þęgi þū [nū], Þōrr!\gap þeira orða;\\
þegar munu jǫtnar\gap Āsgarð būa,\\
nema þū þinn hamar\gap þēr of heimtir.'

19. Bundu þeir Þōr þā\gap brūðar līni\\
ok inu mikla\gap męni Brīsinga.

20. Lētu und honum\gap hrynja lukla\\
ok kvenn-vāðir\gap of knē falla,\\
en ā brjōsti\gap breiða steina,\\
ok hagliga\gap of hǫfuð typðu.

21. Þā kvað þat Loki,\gap Laufeyjar sonr:\\
`mun ek ok með þēr\gap ambātt vera,\\
vit skulum aka tvær\gap ī Jǫtunheima.'

22. Sęnn vāru hafrar\gap heim of reknir,\\
skyndir at skǫklum,\gap skyldu vel ręnna:\\
bjǫrg brotnuðu,\gap brann jǫrð loga,\\
ōk Ōðins sonr\gap ī Jǫtunheima.

23. Þā kvað þat Þrymr,\gap þursa drōttinn:\\
`standið upp, jǫtnar!\gap ok strāið bękki!\\
nū fœra mēr\gap Freyju at kvān,\\
Njarðar dōttur,\gap ōr Nōa-tūnum.

24. Ganga hēr at garði\gap gull-hyrndar kȳr,\\
öxn al-svartir\gap jǫtni at gamni;\\
fjǫlð ā ek meiðma,\gap fjǫlð ā ek męnja,\\
einnar mēr Freyju\gap āvant þykkir.'

25. Var þar at kveldi\gap of komit snimma,\\
ok fyr jǫtna\gap ǫl fram borit;\\
einn āt oxa,\gap ātta laxa,\\
krāsir allar,\gap þær er konur skyldu,\\
drakk Sifjar verr\gap sāld þrjū mjaðar.

26. Þā kvað þat Þrymr,\gap þursa drōttinn:\\
`hvar sāttu brūðir\gap bīta hvassara?\\
sāk-a-k brūðir\gap bīta breiðara,\\
nē inn meira mjǫð\gap mey of drekka.'

27. Sat in al-snotra\gap ambātt fyrir,\\
er orð of fann\gap við jǫtuns māli:\\
`āt vætr Freyja\gap ātta nāttum,\\
svā var hon ōð-fūs\gap ī Jǫtunheima.'

28. Laut und līnu,\gap lysti at kyssa,\\
en hann utan stǫkk\gap ęnd-langan sal:\\
`hvī eru ǫndōtt\gap augu Freyju?\\
þykkir mēr ōr augum\gap eldr of bręnna.'

29. Sat in alsnotra\gap ambātt fyrir,\\
er orð of fann\gap við jǫtuns māli:\\
`svaf vætr Freyja\gap ātta nāttum,\\
svā var hon ōðfūs\gap ī Jǫtunheima.'

30. Inn kom in arma\gap jǫtna systir,\\
hin er brūð-fjār\gap of biðja þorði:\\
`lāttu þēr af hǫndum\gap hringa rauða,\\
ef þū ǫðlask vill\gap āstir mīnar,\\
āstir mīnar,\gap alla hylli!'

31. Þā kvað þat Þrymr,\gap þursa drōttinn:\\
`berið inn hamar\gap brūði at vīgja,\\
lęggið Mjǫllni\gap ī meyjar knē,\\
vīgið okkr saman\gap Vārar hęndi!'

32. Hlō Hlōriða\gap hugr ī brjōsti,\\
er harð-hugaðr\gap hamar of þękði;\\
Þrym drap hann fyrstan,\gap þursa drōttin,\\
ok ætt jǫtuns\gap alla lamði.

33. Drap hann ina ǫldnu\gap jǫtna systur,\\
hin er brūðfjār\gap of beðit hafði;\\
hon skell of hlaut\gap fyr skillinga,\\
en hǫgg hamars\gap fyr hringa fjǫlð.\\
Svā kom Ōðins sonr\gap ęndr at hamri.
\end{verse}

\end{linenumbers}

\emptypage

\chapter{Notes}

The references marked Gr.\ are to the paragraphs of the Grammar.


\section{Thor}

Line 3.\ \textbf{Hann ā þar rīki er Þrūð-vangar heita}, `he reigns
(there) where it is called Þ.,' i.e.\ in the place which is called
Þ\@.  The plur.\ \textit{heita} agrees with \textit{þrūðvangar}, as in l.\ 14
below: \textit{þat eru jārnglōfar} `that is (his) iron gloves.'

l.\ 5.\ \textbf{þat er hūs mest, svā at męnn hafa gǫ̈rt}, `that is the
largest house, so that men have made (it),' i.e.\ the largest
house that has been built.  Note the plur.\ \textit{hūs} of a single
house; each chamber was originally regarded as a house, being
often a detached building.

l.\ 13.\ \textbf{spęnnir þeim}, Gr.\ § 154; cp.\ line 49 of
``Thor and Ūtgarðaloki.''

\section{Thor and Ūtgarðaloki}

l.\ 1.\ \textbf{fōr með hafra sīna\ldots ok með honum sā, āss er\ldots}  We
see here that \textit{með} generally takes an acc.\ to denote passive,
and a dat.\ to denote voluntary accompaniment.

l.\ 5.\ \textbf{soðit} refers to some such subst.\ as \textit{slātr} (meat)
understood.

l.\ 11.\ \textbf{sprętti ā\ldots} \textit{ā} is here an adv.

l.\ 12.\ \textbf{til męrgjar.}  \textit{til} here implies intention---to get at
the marrow.

l.\ 20.\ \textbf{þat er sā augnanna}, `the little he saw of the eyes.'---
Thor frowned till his eyebrows nearly covered his eyes, and the
man felt as if he were going to fall down dead at the mere sight
of them.

l.\ 21.\ The second \textbf{hann} refers, of course, to Thor.

l.\ 34.\ \textbf{til myrkrs}, till it was dark.

l.\ 36.\ \textbf{þeir}, the masc.\ instead of the neut.\ pl., as in l.\ 32
\textbf{foll.}, showing that \textit{leituðu} is meant to refer only to
the men of the party, and not to include Rǫskva.  (Gr.\ § 179.)

l.\ 46.\ \textbf{sēr hvar lā maðr}, `saw where a man lay,' i.e.\ saw a man
lying.

l.\ 51.\ \textbf{einu sinni}, for once in his life.

l.\ 52.\ \textbf{nęfndisk Skrȳmir}, said his name was Skr.

l.\ 66.\ \textbf{būið til} (\textit{prp.}) \textbf{nātt-verðar yðr}, prepare supper for
yourselves.

l.\ 77.\ \textbf{Er þat þēr satt at sęgja}.  \textit{satt} is in apposition to
\textit{þat}---`that is to be told you as the truth, (namely) that\ldots'

l.\ 88.\ \textbf{sjā sik}, see himself alive.

l.\ 104.\ \textbf{þann} = \textit{þann veg}, that way, course.

l.\ 108.\ \textbf{at æsirnir bæði þā heila hittask}.  The full sense is,
`that Thor and Loki expressed a wish that they and Skrȳmir might
meet again safe and sound.'

l.\ 111.\ \textbf{sęttu hnakkann ā bak sēr aptr}, threw back the
backs of their heads till they touched their backs, i.e.\ threw
back their heads.

l.\ 118.\ \textbf{œrit stōra}, `rather big,' i.e.\ very big.

l.\ 120.\ \textbf{glotti um tǫnn}, `grinned round a tooth,' i.e.\ showed
his teeth in a malicious grin.  Two MSS.\ read \textit{við} instead of
\textit{um}.

l.\ 121.\ \textbf{er annan veg en ek hygg, at\ldots?}  is it otherwise than
as I think, namely that\ldots?  i.e.\ am I not right in thinking
that\ldots?

l.\ 127.\ \textbf{engi er hēr sā inni er\ldots} = engi er hēr-inni sā-er\ldots

l.\ 129.\ \textbf{freista skal}, Gr.\ § 192.

l.\ 140.\ \textbf{kallar þess meiri vān at hann sē\ldots}  `says that there
is more probability of that, namely that he is\ldots than of the
contrary,' i.e.\ says that he will have to be\ldots

l.\ 150.\ \textbf{fōthvatari en svā}, `more swift-footed than so---under
these circumstances,' i.e.\ than you.

l.\ 156.\ \textbf{ok er Þjālfl eigi þā kominn\ldots} = þā er Þjālfi eigi
kominn\ldots

l.\ 172.\ \textbf{at sinni}, this time.

l.\ 172.\ \textbf{hann}, acc.

l.\ 173.\ \textbf{laut ōr horninu}, bent back from the horn.

l.\ 188.\ \textbf{mestr} refers to \textit{drykkr} understood.

l.\ 218.\ \textbf{kalli}, Gr.\ § 192.

l.\ 241.\ \textbf{þō} refers to \textit{uni}.


\section{Balder}

l.\ 7.\ \textbf{engi} agrees with \textit{dōmr}.


\section{The Death of Balder}

l.\ 11.\ \textbf{hann}, acc.

l.\ 20.\ \textbf{ungr}, too young.

l.\ 31.\ \textbf{ū-happ} is in apposition to \textit{þat}; cp.\ line 7
of ``Balder''.

l.\ 33.\ \textbf{vāru með einum hug til\ldots}  had the same feelings
towards.

l.\ 36.\ \textbf{var\ldots fyrr}, was beforehand, prevented.

l.\ 42.\ \textbf{vili}, subj.\ `whether he will';---change of construction.

l.\ 55.\ \textbf{nema}, `unless,' here = `until.'

l.\ 89.\ \textbf{heim}.  This use of \textit{heim} in the sense of `someone
else's home.' is frequent.  Cp.\ our `drive a nail home.'

l.\ 94.\ \textbf{svā} refers to \textit{ok ef allir hlutir\ldots}, the \textit{ok}
being pleonastic.

l.\ 108.\ \textbf{hvar}, cp.\ line 46 of ``Thor and Ūtgarðaloki.''

l.\ 113.\ \textbf{karl}, `old man,' here = Odin.


\section{Hēðinn and Hǫgni}

l.\ 6.\ \textbf{Hēðinn} = hann; this use of a proper name instead of a
pronoun is frequent.

l.\ 9.\ \textbf{þar}, cp.\ line 3 of ``Thor.''


\section{The Death of Olaf Tryggvason}

l.\ 17.\ \textbf{svā at}, so that, i.e.\ just when.

l.\ 30.\ \textbf{er}, namely that.

l.\ 33.\ \textbf{austan at sigla}, is in a kind of apposition to
\textit{fęrðina}.

l.\ 40.\ \textbf{svā}, also.

l.\ 48.\ \textbf{muni fyrir}, awaits you, is impending.

l.\ 53.\ \textbf{meira}, adv., better, faster.

l.\ 149.\ \textbf{bar hann hātt}, impers.\ w.\ acc.; he was in a conspicuous
place.

l.\ 244.\ \textbf{við fimtānda mann}, one of fifteen, with fourteen men.

l.\ 259.\ \textbf{vāpna} is governed by the second half of the genitival
compound \textit{klæða-būnað}, which is here considered as two
independent words.


\section{Auðun}

l.\ 17.\ \textbf{var fyrir}, was to be found.

l.\ 26.\ \textbf{tvau verð slīk}, double the price you gave.

l.\ 55.\ \textbf{fē} is probably dat.\ here, but may be nom.

l.\ 56.\ \textbf{ęptir sem\ldots}, according as, in accordance with what.

l.\ 64.\ \textbf{nū\ldots nū}, lately\ldots just now.

l.\ 116.\ \textbf{hęldr} is here used pleonastically in a kind of
apposition to the preceding \textit{bętr}.

l.\ 125.\ \textbf{en}, and = while.

l.\ 129.\ \textbf{þessi einn var svā hlutrinn, at\ldots}, this single thing
is the case, namely that\ldots i.e.\ the only thing is that\ldots


\section{Þrymskviða}

l.\ 7.\ \textbf{jarðar} is governed by \textit{hvęrgi}.

l.\ 9.\ \textbf{tūna}.  Poetical construction of gen.\ to denote goal of
motion.

l.\ 15.\ \textbf{þō} goes with the following \textit{at} = \textit{ok sęlja, þōat (þōtt)
væri ōr silfri}.

l.\ 32.\ \textbf{fœri} may be either sg.\ or pl.\ 3 pers.



\part{Glossary}

\emptypage

\chapter*{Introductory Notes}

æ \textit{follows} að, ð \textit{follows} d, ę \textit{follows} e, œ
\textit{follows} oð, ǫ \textit{follows} o, ö \textit{follows} ǫ, þ
\textit{follows} t.

The declensions of nouns are only occasionally given.

(-rs) etc.\ means that the \textit{r} of the nom.\ is kept in inflection.

\emptypage

\chapter*{A}

\noindent
\textbf{-a} \textit{adv.} not.

\noindent
\textbf{ā} \textit{sf.} river.

\noindent
\textbf{ā} \textit{} \textit{see} \textbf{eiga}.

\noindent
\textbf{ā} \textit{prp.} \textit{w.\ acc.\ and dat.} on, in.

\noindent
\textbf{āðr} \textit{adv., cj.} before.

\noindent
\textbf{aðrir} \textit{} \textit{see} \textbf{annar}.

\noindent
\textbf{æsir} \textit{} \textit{see} \textbf{āss}.

\noindent
\textbf{ætla} \textit{wv.} 3, consider, deem: `ætlask fyrir,' intend.

\noindent
\textbf{ætt} \textit{sf.} 2, race, descent, family.

\noindent
\textbf{ætta} \textit{} \textit{see} \textbf{eiga}.

\noindent
\textbf{af} \textit{prp.} \textit{w.\ dat.} from; of; with; \textit{adv.} `drekka af,' drink off.

\noindent
\textbf{af-hūs} \textit{sn.} out-house, side room.

\noindent
\textbf{afl} \textit{sn.} strength, might.

\noindent
\textbf{af-taka} \textit{wf.} damage, injury.

\noindent
\textbf{āgætliga} \textit{adv.} splendidly.

\noindent
\textbf{aka} \textit{sv.} 2, drive (a chariot, etc.).

\noindent
\textbf{ākafliga} \textit{adv.} vehemently, hard---`kalla a.' call loudly.

\noindent
\textbf{akarn} \textit{sn.} acorn.

\noindent
\textbf{akkeri} \textit{sn.} anchor.

\noindent
\textbf{ālar-ęndir} \textit{sm.} thong-end, end of a strap.

\noindent
\textbf{al-būinn} \textit{adj.\ w.\ gen.} quite ready.

\noindent
\textbf{aldinn} \textit{adj.} old.

\noindent
\textbf{aldri} \textit{adv.} never.

\noindent
\textbf{ā-lęngðar} \textit{adv.} for some time.

\noindent
\textbf{ālfr} \textit{sm.} elf.

\noindent
\textbf{ā-lit} \textit{snpl.} appearance, countenance [līta].

\noindent
\textbf{all-harðr} \textit{adj.} very hard, very violent.

\noindent
\textbf{all-lītill} \textit{adj.} very little.

\noindent
\textbf{all-mann-skœðr} \textit{adj.} (very injurious to men), very murderous
    (of a battle) [skaði, `injury'].

\noindent
\textbf{all-mikill} \textit{adj.} very great.

\noindent
\textbf{allr} \textit{adj.} all, whole; `með ǫllu,' entirely; `alls fyrst,'
    first of all.

\noindent
\textbf{all-stōrum} \textit{adv.} very greatly.

\noindent
\textbf{all-valdr} \textit{sm.} monarch, king.

\noindent
\textbf{al-snotr} \textit{adj.} very clever.

\noindent
\textbf{al-svartr} \textit{adj.} very black, coal-black.

\noindent
\textbf{alt} \textit{adv.} quite.

\noindent
\textbf{ambātt} \textit{sf.} 2, female slave, maid.

\noindent
\textbf{and-lit} \textit{sn.} face [līta].

\noindent
\textbf{annarr} \textit{prn.} second; following, next; other; one of the
	two---`annar\ldots annarr,' one\ldots the other.

\noindent
\textbf{aptann} \textit{sm.} evening.

\noindent
\textbf{aptastr} \textit{adj.} most behind.

\noindent
\textbf{aptr} \textit{adv.} back, backwards, behind.

\noindent
\textbf{ār} \textit{sf.} oar.

\noindent
\textbf{ār} \textit{sn.} year.

\noindent
\textbf{ār} \textit{} \textit{see} \textbf{ā}.

\noindent
\textbf{argr} \textit{adj.} cowardly, base.

\noindent
\textbf{ār-maðr} \textit{sm.} steward.

\noindent
\textbf{armr} \textit{adj.} wretched.

\noindent
\textbf{ās-męgin} \textit{sn.} divine strength.

\noindent
\textbf{āss} \textit{sm.} 3, (Scandinavian) god.

\noindent
\textbf{āst} \textit{sf.} 2, affection, love, \textit{often in pl.}

\noindent
\textbf{āst-sæll} \textit{adj.} beloved, popular.  [sæll, `happy'].

\noindent
\textbf{āsynja} \textit{wf.} (Scandinavian) goddess [āss].

\noindent
\textbf{āt} \textit{} \textit{see} \textbf{eta}.

\noindent
\textbf{at} \textit{prp.} \textit{w.\ dat.} at, by; to, towards, up to; for; in accordance
    with, after.

\noindent
\textbf{at} \textit{adv.} to.

\noindent
\textbf{-at} \textit{adv.} not.

\noindent
\textbf{at-laga} \textit{wf.} attack [lęggja].

\noindent
\textbf{at-rōðr (-rar)} \textit{sm.} 2, rowing against, attack.

\noindent
\textbf{ātta} \textit{num.} eight.

\noindent
\textbf{ātta} \textit{} \textit{see} \textbf{eiga}.

\noindent
\textbf{auð-kęndr} \textit{adj.} easy to be recognized, easily
	distinguishable.

\noindent
\textbf{auðr} \textit{adj.} desert, deserted, without men.

\noindent
\textbf{auð-sēnn} \textit{adj.} evident.

\noindent
\textbf{auga} \textit{wn.} eye.

\noindent
\textbf{aug-sȳn} \textit{sf.} sight.

\noindent
\textbf{auka} \textit{sv.} 1, increase; \textit{impers., w.\ dat.\ of what is
	added} `jōk nū miklu ā,' much was added to it (his hesitation
	increased).

\noindent
\textbf{austan} \textit{adv.} from the east.

\noindent
\textbf{aust-maðr} \textit{sm.} Easterner, Norwegian.

\noindent
\textbf{austr} \textit{sn.} the east---`ī au.,' eastwards.

\noindent
\textbf{austr} \textit{adv.} eastwards.

\noindent
\textbf{austr-vegr} \textit{sm.} the East, \textit{especially} Russia.

\noindent
\textbf{āvalt} \textit{adv.} continually, all the time.

\noindent
\textbf{ā-vanr} \textit{adj.} wanting; \textit{impers.\ neut.\ in} `einnar
	mēr Freyju āvant þykkir,' Freyja alone I seem to want.

\noindent
\textbf{ā-vinnr} \textit{adj.} toilsome, \textit{only in the impers.\ neut.}
	`mun ā-vint verða um sǫxin,' it will be a hard fight at the prow.

\emptypage

\chapter*{B}

\noindent
\textbf{bað} \textit{} \textit{see} \textbf{biðja}.

\noindent
\textbf{bāðir} \textit{prn.} both, \textit{neut.\ as adv.\ in} `bæði\ldots ok,'
	both\ldots and.

\noindent
\textbf{bāðu} \textit{} \textit{see} \textbf{biðja}.

\noindent
\textbf{bǣða} \textit{} \textit{see} \textbf{biðja}.

\noindent
\textbf{bǣði} \textit{} \textit{see} \textbf{bāðir}.

\noindent
\textbf{baggi} \textit{wm.} bag; bundle.

\noindent
\textbf{bak} \textit{sn.} back, `vęrja eitt baki,' defend a thing with
	the back, i.e.\ turns one's back to it = be a coward.

\noindent
\textbf{bak-borði} \textit{wm.} larboard.

\noindent
\textbf{bāl} \textit{sn.} flame; funeral pile.

\noindent
\textbf{bāl-fǫr} \textit{sf.} funeral.

\noindent
\textbf{bani} \textit{wm.} death.

\noindent
\textbf{bar} \textit{} \textit{see} \textbf{bera}.

\noindent
\textbf{bardagi} \textit{wm.} battle.

\noindent
\textbf{barð} \textit{sn.} edge, rim; projection in the prow of a ship
	formed by the continuation of the keel.

\noindent
\textbf{barða} \textit{} \textit{see} \textbf{bęrja}.

\noindent
\textbf{barði} \textit{wm.} war-ship with a sharp prow, ram.

\noindent
\textbf{barn} \textit{sn.} child.

\noindent
\textbf{batt} \textit{} \textit{see} \textbf{binda}.

\noindent
\textbf{bauð} \textit{} \textit{see} \textbf{bjōða}.

\noindent
\textbf{beiða} \textit{wv.} 1, \textit{w.\ gen.\ of thing and dat.\ of pers.
	benefited}, ask, demand.

\noindent
\textbf{bein} \textit{sn.} bone.

\noindent
\textbf{bękkr} \textit{sm.} 2, bench.

\noindent
\textbf{bęlla} \textit{wv.} 1, occupy oneself with, deal in, \textit{generally
	in a bad sense}.

\noindent
\textbf{bera} \textit{sv.} 4, carry, take; bear, endure.  \textbf{berask at},
	happen.  \textbf{b.\ fram}, bring forward, out.  \textbf{b.\ vāpn niðr},
	shoot down.  \textbf{b.\ rāð sīn saman}, hold council, deliberate.

\noindent
\textbf{berg-risi} \textit{wm.} hill-giant.

\noindent
\textbf{bęrja} \textit{wv.} 1b, strike---`b.\ grjōti,' stone.  \textbf{bęrjask},
	fight.

\noindent
\textbf{bęrr} \textit{adj.} bare, unsheathed (of a sword).

\noindent
\textbf{ber-sęrkr} \textit{sm.} 2, wild fighter, champion.  [Literally
    `bear-shirt,' i.e.\ one clothed in a bear's skin.]

\noindent
\textbf{bętr} \textit{} \textit{see} \textbf{vel}.

\noindent
\textbf{bętri} \textit{} \textit{see} \textbf{gōðr}.

\noindent
\textbf{bęzt} \textit{} \textit{see} \textbf{vel}.

\noindent
\textbf{bęztr} \textit{} \textit{see} \textbf{gōðr}.

\noindent
\textbf{beygja} \textit{wv.} 1, bend, arch.

\noindent
\textbf{bīða} \textit{sv.} 6, \textit{w.\ gen.} wait for; \textit{w.\ acc.}
	abide, undergo.

\noindent
\textbf{biðja} \textit{sv.} 5, ask, beg, pray, \textit{w.\ gen.\ of thing, acc.
	of the pers.\ asked, and dat.\ of the person benefited}; express a
    wish, bid---`bað hann vel kominn' (vera \textit{understood}), bad
    him welcome; call on, challenge, command, tell.

\noindent
\textbf{bifask} \textit{wv.} 1, tremble, shake.

\noindent
\textbf{bil} \textit{sn.} moment of time.

\noindent
\textbf{bila} \textit{wv.} 2, fail.

\noindent
\textbf{bilt} \textit{neut.\ adj.} \textit{only in the impers.} `einum verðr
	bilt,' one hesitates, is taken aback, is afraid.

\noindent
\textbf{binda} \textit{sv.} 3, bind, tie up; dress.

\noindent
\textbf{birta} \textit{wv.} 1, show [bjartr].

\noindent
\textbf{bīta} \textit{sv.} 6, bite; cut.

\noindent
\textbf{bittu} \textit{} \textit{see} \textbf{binda}.

\noindent
\textbf{bitu} \textit{} \textit{see} \textbf{bīta}.

\noindent
\textbf{bjarn-dȳri} \textit{sn.} bear.

\noindent
\textbf{bjartr} \textit{adj.} bright, clear.

\noindent
\textbf{bjō} \textit{} \textit{see} \textbf{būa}.

\noindent
\textbf{bjōða} \textit{sv.} 7, \textit{w.\ acc.\ and dat.} offer,
	propose---`b.\ einum fang,' challenge to wrestling; invite.
	\textbf{b.\ upp}, give up.

\noindent
\textbf{bjǫrg} \textit{sf.} help; means of subsistence, store of food.

\noindent
\textbf{blāsa} \textit{sv.} 1, blow; blow trumpet as signal.

\noindent
\textbf{bleyða} \textit{wf.} coward.

\noindent
\textbf{blīða} \textit{wf.} gentleness, friendliness.

\noindent
\textbf{blindr} \textit{adj.} blind.

\noindent
\textbf{blōð} \textit{sn.} blood.

\noindent
\textbf{blōt-bolli} \textit{wm.} sacrificial bowl.

\noindent
\textbf{bœtr} \textit{} \textit{see} \textbf{bōt}.

\noindent
\textbf{boga-skot} \textit{sn.} bowshot.

\noindent
\textbf{bogi} \textit{wm.} bow.

\noindent
\textbf{bog-maðr} \textit{sm.} bowman, archer.

\noindent
\textbf{bōndi} \textit{sm.} 4, yeoman, householder, (free) man [būa].

\noindent
\textbf{borð} \textit{sn.} side of a ship, board; rim, the margin between
	the rim of a vessel and the liquid in it---`nū er gott beranda
    b.\ ā horninu,' now there is a good margin for carrying the
    horn, i.e.\ its contents are so diminished that it can be
    lifted without spilling.

\noindent
\textbf{borg} \textit{sf.} fortress, castle.

\noindent
\textbf{borg-hlið} \textit{sn.} castle gate.

\noindent
\textbf{bōt} \textit{sf.} 3, mending, improvement; \textit{plur.} \textbf{bœtr},
	compensation.

\noindent
\textbf{bǫrn} \textit{} \textit{see} \textbf{barn}.

\noindent
\textbf{brā} \textit{sf.} eyelid.

\noindent
\textbf{brā} \textit{} \textit{see} \textbf{bregða}.

\noindent
\textbf{bragð} \textit{sn.} trick, stratagem [bregða].

\noindent
\textbf{brann} \textit{} \textit{see} \textbf{bręnna}.

\noindent
\textbf{brast} \textit{} \textit{see} \textbf{bresta}.

\noindent
\textbf{brātt} \textit{adv.} quickly.

\noindent
\textbf{braut} \textit{sf.} way---\textbf{ā braut}, \textit{adv.} away.

\noindent
\textbf{braut} \textit{} \textit{see} \textbf{brjōta}.

\noindent
\textbf{braut, brott} \textit{adv.} away.

\noindent
\textbf{bregða} \textit{sv.} 3, \textit{w.\ dat.} jerk, pull, push;
	\textbf{b.\ upp}, lift, raise (to strike). change, transform.

\noindent
\textbf{breiðr} \textit{adj.} broad.

\noindent
\textbf{bręnna} \textit{wf.} burning; incremation.

\noindent
\textbf{bręnna} \textit{sv.} 3, burn \textit{intr.}

\noindent
\textbf{bręnna} \textit{wv.} 1, burn \textit{trans.}

\noindent
\textbf{bresta} \textit{sv.} 3, break, crack, burst.

\noindent
\textbf{brestr} \textit{sm.} crack; loss.

\noindent
\textbf{brjōst} \textit{sn.} breast.

\noindent
\textbf{brjōta} \textit{sv.} 7, break---`b.\ (skip)' suffer shipwreck,
	\textit{also impers.} `skip (acc.) brȳtr,' the ship is wrecked.

\noindent
\textbf{broddr} \textit{sm.} point.

\noindent
\textbf{brōðir} \textit{sm.} 4, brother.

\noindent
\textbf{brotinn} \textit{} \textit{see} \textbf{brjōta}.

\noindent
\textbf{brotna} \textit{wv.} 3, break \textit{intr.}

\noindent
\textbf{brott} \textit{} \textit{see} \textbf{braut}.

\noindent
\textbf{brott-laga} \textit{sf.} retreat.

\noindent
\textbf{brū} \textit{sf.} bridge.

\noindent
\textbf{brūð-fē} \textit{sn.} bridal gift.

\noindent
\textbf{brūðr} \textit{sf.} 2, bride.

\noindent
\textbf{brūn} \textit{sf.} 3, eyebrow.

\noindent
\textbf{bryggja} \textit{wf.} pier.

\noindent
\textbf{brynja} \textit{wf.} corslet.

\noindent
\textbf{brȳnn} \textit{} \textit{see} \textbf{brūn}.

\noindent
\textbf{bryn-stūka} \textit{wf.} corslet-sleeve.

\noindent
\textbf{bryti} \textit{} \textit{see} \textbf{brjōta}.

\noindent
\textbf{brȳtr} \textit{} \textit{see} \textbf{brjōta}.

\noindent
\textbf{bū} \textit{sn.} dwelling, home.

\noindent
\textbf{būa} \textit{sv.} 1, dwell; inhabit, possess, prepare.
	\textbf{būask}, get ready, prepare \textit{intr.} `er būit við at\ldots'
	it is likely to be that\ldots, there is danger of\ldots

\noindent
\textbf{buðu} \textit{} \textit{see} \textbf{bjōða}.

\noindent
\textbf{būinn} \textit{adj.} ready; in a certain condition---`(við) svā
	būit' \textit{adv.} under such circumstances; capable, fit for---`vel
	at sēr būinn,' very capable, very good (at).

\noindent
\textbf{bundu} \textit{} \textit{see} \textbf{binda}.

\noindent
\textbf{burr} \textit{sm.} son.

\noindent
\textbf{bȳðr} \textit{} \textit{see} \textbf{bjōða}.

\noindent
\textbf{bȳr} \textit{} \textit{see} \textbf{būa}.

\noindent
\textbf{byrja} \textit{wv.} 3, begin.

\noindent
\textbf{byr-vænn} \textit{adj.} promising a fair wind.

\emptypage

\chapter*{D}

\noindent
\textbf{daga} \textit{wv.} 3, dawn.

\noindent
\textbf{dagan} \textit{sf.} dawn.

\noindent
\textbf{dagr} \textit{sm.} day.

\noindent
\textbf{dalr} \textit{sm.} valley.

\noindent
\textbf{dauðr} \textit{adj.} dead.

\noindent
\textbf{dęgi} \textit{} \textit{see} \textbf{dagr}.

\noindent
\textbf{deyja} \textit{sv.} 2, die.

\noindent
\textbf{djūpr} \textit{adj.} deep.

\noindent
\textbf{dō} \textit{} \textit{see} \textbf{deyja}.

\noindent
\textbf{dōmr} \textit{sm.} decision.

\noindent
\textbf{dōttir} \textit{sf.} 3, daughter.

\noindent
\textbf{dǫgurðr} \textit{sm.} breakfast [-urð = -verðr, \textit{cp.} nāttverðr].

\noindent
\textbf{dōkkr (-vir)} \textit{adj.} dark.

\noindent
\textbf{draga} \textit{sv.} 2, draw, drag.  \textbf{d.\ saman}, collect.

\noindent
\textbf{drakk} \textit{} \textit{see} \textbf{drekka}.

\noindent
\textbf{drap} \textit{} \textit{see} \textbf{drepa}.

\noindent
\textbf{draumr} \textit{sm.} dream.

\noindent
\textbf{dręginn} \textit{} \textit{see} \textbf{draga}.

\noindent
\textbf{dreif} \textit{} \textit{see} \textbf{drīfa}.

\noindent
\textbf{dręki} \textit{wm.} dragon; dragon-ship, ship with a dragon's head
	as a beak.

\noindent
\textbf{drekka} \textit{sv.} 3, drink.

\noindent
\textbf{drepa} \textit{sv.} 5, strike; kill.

\noindent
\textbf{dreyma} \textit{wv.} 1, \textit{impers.\ w.\ acc.\ of pers.\ and acc.\ of the
	thing} dream [draumr].

\noindent
\textbf{drīfa} \textit{sv.} 6, drive; hasten.

\noindent
\textbf{drjūpa} \textit{sv.} 7, drop.

\noindent
\textbf{drō} \textit{} \textit{see} \textbf{draga}.

\noindent
\textbf{drōgu} \textit{} \textit{see} \textbf{draga}.

\noindent
\textbf{drōttinn} \textit{sm.} lord.

\noindent
\textbf{drōttin-hollr} \textit{adj.} faithful to its master.

\noindent
\textbf{drukkinn} \textit{adj.\ (ptc.)} drunk.

\noindent
\textbf{drupu} \textit{} \textit{see} \textbf{drjūpa}.

\noindent
\textbf{drykkja} \textit{wf.} drinking [drekka].

\noindent
\textbf{drykkju-maðr} \textit{sm.} drinker.

\noindent
\textbf{drykkr} \textit{sm.} 2, draught.

\noindent
\textbf{duna} \textit{wv.} 3, resound.

\noindent
\textbf{dunði} \textit{} \textit{see} \textbf{dynja}.

\noindent
\textbf{dvalða} \textit{} \textit{see} \textbf{dvęlja}.

\noindent
\textbf{dvęlja} \textit{wv.} 1b, delay.  \textbf{dvęljask}, dwell, stop.

\noindent
\textbf{dvergr} \textit{sm.} dwarf.

\noindent
\textbf{dyðrill} \textit{sm.} (?).

\noindent
\textbf{dȳja} \textit{wv.} 1b, shake.

\noindent
\textbf{dynja} \textit{wv.} 1b, resound.

\noindent
\textbf{dȳr} \textit{sn.} animal, beast.

\noindent
\textbf{dyrr} \textit{sfnpl.} door.

\emptypage

\chapter*{E}

\noindent
\textbf{eða} \textit{cj.} or.

\noindent
\textbf{ef} \textit{cj.} if.

\noindent
\textbf{ęfna} \textit{wv.} 1, perform, carry out.

\noindent
\textbf{ęggja} \textit{wv.} 3, incite.

\noindent
\textbf{eiðr} \textit{sm.} oath.

\noindent
\textbf{eiga} \textit{wf.} property.

\noindent
\textbf{eiga} \textit{swv.} possess, have: have as wife, be married to;
	have \textit{in the sense of} must.

\noindent
\textbf{eigi} \textit{adv.} not; no.

\noindent
\textbf{eign} \textit{sf.} property.

\noindent
\textbf{eignask} \textit{wv.} 3, appropriate, gain.

\noindent
\textbf{eik} \textit{sf.} 3, oak.

\noindent
\textbf{einka-māl} \textit{snpl.} personal agreement, special treaty.

\noindent
\textbf{einn} \textit{num., prn.} one; the same; a certain, a; alone,
	only---`einn saman,' alone, mere.

\noindent
\textbf{einn-hvęrr} \textit{prn.} a certain, some, a.

\noindent
\textbf{eira} \textit{wv.} 1, \textit{w.\ dat.} spare.

\noindent
\textbf{eitr} \textit{sn.} poison.

\noindent
\textbf{ek} \textit{prn.} I.

\noindent
\textbf{ęk} \textit{} \textit{see} \textbf{aka}.

\noindent
\textbf{ekki} \textit{prn.\ neut.} nothing; \textit{adv.} not.

\noindent
\textbf{eldr} \textit{sm.} fire.

\noindent
\textbf{ęlli} \textit{wf.} old age.

\noindent
\textbf{ellifti} \textit{adj.} eleventh.

\noindent
\textbf{ellifu} \textit{num.} eleven.

\noindent
\textbf{em} \textit{} \textit{see} \textbf{vera}.

\noindent
\textbf{en} \textit{cj.} but; and.

\noindent
\textbf{en} \textit{adv.} than, \textit{after compar.}

\noindent
\textbf{ęndask} \textit{wv.} 1, end, suffice for.

\noindent
\textbf{ęndi, ęndir} \textit{sm.} end.

\noindent
\textbf{ęnd-langr} \textit{adj.} the whole length---`ęndlangan sal,' the
	whole length of the hall.

\noindent
\textbf{ęndr} \textit{adv.} again.

\noindent
\textbf{engi} \textit{prn.} none, no.

\noindent
\textbf{ęnn} \textit{adv.} yet, still; besides; after all.

\noindent
\textbf{ęptir} \textit{prp.} \textit{w.\ acc.} after (of time).  \textit{w.
	dat.} along, over; in quest of, after; according to, by.
	\textit{adv.} afterwards; behind [aptr].

\noindent
\textbf{ęptri} \textit{adj.\ compar.} hind.

\noindent
\textbf{er} \textit{prn.\ rel.} who, which, \textit{rel.\ adv.} where;
	when; because, that.

\noindent
\textbf{er} \textit{} \textit{see} \textbf{vera}.

\noindent
\textbf{er} \textit{} \textit{see} \textbf{þū}.

\noindent
\textbf{ęrfiði} \textit{sn.} work; trouble.

\noindent
\textbf{ert} \textit{} \textit{see} \textbf{vera}.

\noindent
\textbf{eru} \textit{} \textit{see} \textbf{vera}.

\noindent
\textbf{eta} \textit{sv.} 5, eat.

\noindent
\textbf{ey} \textit{sf.} island.

\emptypage

\chapter*{F}

\noindent
\textbf{fā} \textit{sv.} 1, grasp; receive, get; give; be able.
	\textbf{fāsk}, wrestle.  \textbf{fāsk ā}, be obtained, be.

\noindent
\textbf{faðir} \textit{sm.} 4, father.

\noindent
\textbf{fær} \textit{} \textit{see} \textbf{fā}.

\noindent
\textbf{fæstr} \textit{} \textit{see} \textbf{fār}.

\noindent
\textbf{fagnaðr} \textit{sm.} 2, joy; entertainment, hospitality.

\noindent
\textbf{fagr} \textit{adj.} beautiful, fair, fine.

\noindent
\textbf{fagrliga} \textit{adv.} finely.

\noindent
\textbf{fall} \textit{sn.} fall.

\noindent
\textbf{falla} \textit{sv.} 1, fall.  \textbf{fallask}, be forgotten,
	fail.

\noindent
\textbf{fang} \textit{sn.} embrace, grasp; wrestling.

\noindent
\textbf{fann} \textit{} \textit{see} \textbf{finna}.

\noindent
\textbf{fār} \textit{adj.} few---\textit{neut.} fātt \textit{w.\ gen.}:
	`fātt manna,' few men.

\noindent
\textbf{fara} \textit{} go, travel---\textit{w.\ gen.\ in such constr.\ as}
	`f.\ fęrðar sinnar,' go his way; fare (well, ill); happen, turn out;
    experience; `f.\ með einu,' deal with, treat; destroy, use
    up \textit{w.\ dat.}  \textbf{farask} \textit{impers.\ in} `fęrsk þeim vel,'
	they have a good passage.

\noindent
\textbf{farmr} \textit{sm.} lading, cargo.

\noindent
\textbf{fastr} \textit{adj.} firm, fast, strong.

\noindent
\textbf{fē} \textit{sn.} property, money.

\noindent
\textbf{fęðrum} \textit{} \textit{see} \textbf{faðir}.

\noindent
\textbf{fekk} \textit{} \textit{see} \textbf{fā}.

\noindent
\textbf{fęgrð} \textit{sf.} beauty [fagr].

\noindent
\textbf{fęgrstr} \textit{} \textit{see} \textbf{fagr}.

\noindent
\textbf{feikn-stafir} \textit{smpl.} 2, wickedness.

\noindent
\textbf{fela} \textit{sv.} 3, hide.

\noindent
\textbf{fēlagi} \textit{wm.} companion.

\noindent
\textbf{fē-lauss} \textit{adj.} penniless.

\noindent
\textbf{fē-lītill} \textit{adj.} with little money, poor.

\noindent
\textbf{fell} \textit{} \textit{see} \textbf{falla}.

\noindent
\textbf{fęlla} \textit{wv.} 1, fell, throw down; kill [falla].

\noindent
\textbf{fengu, fęnginn} \textit{} \textit{see} \textbf{fā}.

\noindent
\textbf{fęr} \textit{} \textit{see} \textbf{fara}.

\noindent
\textbf{fęrð} \textit{sf.} 2, journey [fara].

\noindent
\textbf{fer-skeyttr} \textit{adj.} four-cornered.

\noindent
\textbf{fimm} \textit{num.} five.

\noindent
\textbf{fimtāndi} \textit{adj.} fifteenth.

\noindent
\textbf{finna} \textit{sv.} 3, find; meet, go to see; notice, see.

\noindent
\textbf{fjaðr-hamr} \textit{sm.} feathered (winged) coat.

\noindent
\textbf{fjall} \textit{sn.} mountain.

\noindent
\textbf{fjār} \textit{} \textit{see} \textbf{fē}.

\noindent
\textbf{fjara} \textit{sf.} ebb-tide; beach.

\noindent
\textbf{fjogur} \textit{} \textit{see} \textbf{fjōrir}.

\noindent
\textbf{fjōrði} \textit{adj.} fourth.

\noindent
\textbf{fjōrir} \textit{num.} four.

\noindent
\textbf{fjǫlð} \textit{wf.} quantity.

\noindent
\textbf{fjǫl-kyngi} \textit{sn.} magic.

\noindent
\textbf{fjǫl-męnni} \textit{sn.} multitude; troop [maðr].

\noindent
\textbf{fjǫrðr} \textit{sm.} 3, firth.

\noindent
\textbf{fjǫrur} \textit{} \textit{see} \textbf{fjara}.

\noindent
\textbf{flā} \textit{sv.} 2, flay, skin.

\noindent
\textbf{flaug} \textit{} \textit{see} \textbf{fljūga}.

\noindent
\textbf{flęginn} \textit{} \textit{see} \textbf{flā}.

\noindent
\textbf{flestr} \textit{} \textit{see} \textbf{margr}.

\noindent
\textbf{fljōta} \textit{sv.} 7, float, drift.

\noindent
\textbf{fljūga} \textit{sv.} 7, fly.

\noindent
\textbf{flō} \textit{} \textit{see} \textbf{fljūga}.

\noindent
\textbf{floti} \textit{wm.} fleet [fljōtal.

\noindent
\textbf{flōtti} \textit{wm.} flight [flȳja].

\noindent
\textbf{flugu} \textit{} \textit{see} \textbf{fljūga}.

\noindent
\textbf{fluttu} \textit{} \textit{see} \textbf{flytja}.

\noindent
\textbf{flȳja} \textit{wv.} 1, flee.

\noindent
\textbf{flytja} \textit{wv.} 1b, remove, bring.

\noindent
\textbf{fnāsa} \textit{wv.} 3, snort.

\noindent
\textbf{fœra} \textit{wv.} 1, bring, take; fasten [fara].

\noindent
\textbf{fœrir} \textit{} \textit{see} \textbf{fara}.

\noindent
\textbf{fœti} \textit{} \textit{see} \textbf{fōt}.

\noindent
\textbf{fōlginn} \textit{} \textit{see} \textbf{fela}.

\noindent
\textbf{fōlk} \textit{sn.} multitude, troop; people.

\noindent
\textbf{fōr} \textit{} \textit{see} \textbf{fara}.

\noindent
\textbf{for-tǫlur} \textit{wpl.} representations, arguments [tala].

\noindent
\textbf{fōstra} \textit{wf.} nurse.

\noindent
\textbf{fōt-hvatr} \textit{adj.} swift-footed.

\noindent
\textbf{fōtr} \textit{sm.} foot; leg.

\noindent
\textbf{fǫður} \textit{} \textit{see} \textbf{faðir}.

\noindent
\textbf{fǫr} \textit{sf.} journey [fara].

\noindent
\textbf{fǫru-neyti} \textit{sn.} company [njōta].

\noindent
\textbf{frā} \textit{prp.} from, away from; about, concerning.  `ī frā'
	\textit{adv.} away.

\noindent
\textbf{frændi} \textit{sm.} 4, relation.

\noindent
\textbf{frā-fall} \textit{sn.} death.

\noindent
\textbf{fram} \textit{adv.} forward, forth.  \textit{compar.} \textbf{framar},
	ahead.

\noindent
\textbf{framastr} \textit{adj.\ superl.} chief, most distinguished.

\noindent
\textbf{frami} \textit{wm.} advantage, courage.

\noindent
\textbf{framiðr} \textit{} \textit{see} \textbf{fręmja}.

\noindent
\textbf{fram-stafn} \textit{sm.} prow.

\noindent
\textbf{frā-sǫgn} \textit{sf.} narrative, relation.

\noindent
\textbf{freista} \textit{wv.} 3, \textit{w.\ gen.} try, test.

\noindent
\textbf{fręmja} \textit{wv.} 1b, perform [fram].

\noindent
\textbf{friðr} \textit{sm.} 2, peace.

\noindent
\textbf{frīðr} \textit{adj.} beautiful, fine.

\noindent
\textbf{frōðr} \textit{adj.} learned, wise.

\noindent
\textbf{frœkn} \textit{adj.} bold, daring.

\noindent
\textbf{frost} \textit{sn.} frost.

\noindent
\textbf{fugl} \textit{sm.} bird.

\noindent
\textbf{full-kominn} \textit{adj.\ (ptc.)} complete; ready (for).

\noindent
\textbf{fullr} \textit{adj.} full.

\noindent
\textbf{fundr} \textit{sm.} 2, meeting [finna].

\noindent
\textbf{fundu} \textit{} \textit{see} \textbf{finna}.

\noindent
\textbf{furðu} \textit{adv.} awfully, very.

\noindent
\textbf{fūss} \textit{adj.} eager.

\noindent
\textbf{fylgja} \textit{wv.} 1, \textit{w.\ dat.} follow; accompany.

\noindent
\textbf{fylki} \textit{sn.} troop [fōlk].

\noindent
\textbf{fylkja} \textit{wv.} 1, \textit{w.\ dat.} draw up (troops) [fōlk].

\noindent
\textbf{fylla} \textit{wv.} 1, fill [fullr].

\noindent
\textbf{fyr, fyrir} \textit{prp.} \textit{w.\ acc.\ and dat.} before; beyond,
	over---`f.\ borð,' overboard; instead of---`koma f.'  \textit{adv.} be
	given as compensation; for; because of.  `f.\ þvī at,' because.
    `lītill f.\ sēr,' insignificant.

\noindent
\textbf{fyrir-rūm} \textit{sn.} fore-hold, chief-cabin.

\noindent
\textbf{fyrirrūms-maðr} \textit{sm.} man in the fore-hold.

\noindent
\textbf{fyrr} \textit{adv.} \textit{compar.} before, formerly.  \textit{superl.}
	\textbf{fyrst}, first.

\noindent
\textbf{fyrri} \textit{adj.} \textit{compar.} former.  \textit{superl.}
	\textbf{fyrstr}, first.

\noindent
\textbf{fȳsa} \textit{wv.} 1, hasten \textit{trans.---impers.} `braut fȳsir
	mik,' I feel a desire to go away [fūss].

\emptypage

\chapter*{G}

\noindent
\textbf{gā} \textit{wv.} 1, \textit{w.\ gen.} heed, care for.

\noindent
\textbf{gæfa} \textit{wf.} luck [gefa].

\noindent
\textbf{gæta} \textit{wv.} 1, watch, take care of [geta].

\noindent
\textbf{gaf} \textit{} \textit{see} \textbf{gefa}.

\noindent
\textbf{gæfu-maðr} \textit{sm.} lucky man.

\noindent
\textbf{gaflak} \textit{sn.} javelin.

\noindent
\textbf{gamall} \textit{adj.} old.

\noindent
\textbf{gaman} \textit{sn.} amusement, joy.

\noindent
\textbf{ganga} \textit{sv.} 1, go, \textit{with gen.\ of goal in poetry};
	attack---`g.\ ā skip,' board a ship.  \textbf{g.\ af}, be finished.
	\textbf{g.\ til}, come up.  \textbf{g.\ upp}, land; board a ship; be
	used up, expended (of money).

\noindent
\textbf{garðr} \textit{sm.} enclosure, court; dwelling.

\noindent
\textbf{gat} \textit{} \textit{see} \textbf{geta}.

\noindent
\textbf{gefa} \textit{sv.} 5, give.

\noindent
\textbf{gęgnum, ī gęgnum} \textit{prp.} \textit{w.\ gen.} through.

\noindent
\textbf{gekk} \textit{} \textit{see} \textbf{ganga}.

\noindent
\textbf{gęlti} \textit{} \textit{see} \textbf{gǫltr}.

\noindent
\textbf{gęnginn, gęngr, gengu} \textit{} \textit{see} \textbf{ganga}.

\noindent
\textbf{geta} \textit{sv.} 5, \textit{w.\ gen.} mention, speak of; guess,
	suppose.

\noindent
\textbf{geysi} \textit{adv.} excessively.

\noindent
\textbf{gipta} \textit{wf.} luck [gefa].

\noindent
\textbf{giptu-maðr} \textit{sm.} lucky man.

\noindent
\textbf{gjafar} \textit{} \textit{see} \textbf{gjǫf}.

\noindent
\textbf{gjǫf} \textit{sf.} gift.

\noindent
\textbf{glotta} \textit{wv.} 2, smile maliciously, grin.

\noindent
\textbf{glæsiligr} \textit{adj.} magnificent.

\noindent
\textbf{gnȳr} \textit{sm.} din, noise.

\noindent
\textbf{gōðr} \textit{adj.} good---`gott er til eins,' it is easy to get
	at, obtain.

\noindent
\textbf{gōlf} \textit{sn.} floor; apartment, room.

\noindent
\textbf{gott} \textit{} \textit{see} \textbf{gōðr}.

\noindent
\textbf{gǫfugr} \textit{adj.} distinguished [gefa].

\noindent
\textbf{gǫltr} \textit{sm.} 3, boar.

\noindent
\textbf{gǫmul} \textit{} \textit{see} \textbf{gamall}.

\noindent
\textbf{gǫ̈ra} \textit{wv.} 1c, do, make.  \textbf{gǫ̈rask}, set about
	doing; be made into, become.  \textbf{gǫ̈ra at}, accomplish,
	carry out.

\noindent
\textbf{gǫ̈rsimi} \textit{wf.} article of value, treasure.

\noindent
\textbf{granda} \textit{wv.} 3, \textit{w.\ dat.} injure.

\noindent
\textbf{grār} \textit{adj.} gray.

\noindent
\textbf{gras} \textit{sn.} grass; plant, flower.

\noindent
\textbf{grāta} \textit{sv.} 1, weep, mourn for.

\noindent
\textbf{grātr} \textit{sm.} weeping.

\noindent
\textbf{greiða} \textit{wv.} 1, put in order, arrange.

\noindent
\textbf{greip} \textit{} \textit{see} \textbf{grīpa}.

\noindent
\textbf{gres-jārn} \textit{sn.} iron wire (?).

\noindent
\textbf{grey} \textit{sn.} dog.

\noindent
\textbf{grið} \textit{snpl.} peace, security.

\noindent
\textbf{griða-staðr} \textit{sm.} sanctuary.

\noindent
\textbf{grind} \textit{sf.} 3, lattice door, wicket.

\noindent
\textbf{grīpa} \textit{sv.} 6, seize.

\noindent
\textbf{gripr} \textit{sm.} 2, article of value, treasure.

\noindent
\textbf{grjōt} \textit{sn.} stone (collectively).

\noindent
\textbf{grōa} \textit{sv.} 1, grow; heal.

\noindent
\textbf{grœr} \textit{} \textit{see} \textbf{grōa}.

\noindent
\textbf{gruna} \textit{wv.} 3, \textit{impers.}---`mik grunar,' I
	suspect, think.

\noindent
\textbf{grunr} \textit{sm.} 2, suspicion.

\noindent
\textbf{gull} \textit{sn.} gold.

\noindent
\textbf{gull-band} \textit{sn.} gold band.

\noindent
\textbf{gull-būinn} \textit{adj.} adorned with gold.

\noindent
\textbf{gull-hringr} \textit{sm.} gold ring.

\noindent
\textbf{gull-hyrndr} \textit{adj.\ (ptc.)} with gilt horns.

\noindent
\textbf{gull-roðinn} \textit{adj.\ (ptc.)} gilt.

\noindent
\textbf{gȳgr} \textit{sf.} giantess.

\noindent
\textbf{gyltr} \textit{adj.\ (ptc.)} gilt.

\noindent
\textbf{gyrða} \textit{wv.} 1, gird.

\emptypage

\chapter*{H}

\noindent
\textbf{hæstr} \textit{} \textit{see} \textbf{hār}.

\noindent
\textbf{hætta} \textit{wv.} 1, \textit{w.\ dat.} desist from, stop.

\noindent
\textbf{hættligr} \textit{adj.} dangerous, threatening.

\noindent
\textbf{hættr} \textit{adj.} dangerous.

\noindent
\textbf{haf} \textit{sn.} sea.

\noindent
\textbf{hafa} \textit{wv.} 1, have; `h.\ einn nær einu,' bring near to,
	expose to; use, utilize.  \textbf{at hafask}, undertake.  \textbf{til
	hafa}, have at hand.

\noindent
\textbf{hafna} \textit{} \textit{see} \textbf{hǫfn}.

\noindent
\textbf{hafr} \textit{sm.} goat.

\noindent
\textbf{hafr-staka} \textit{wf.} goatskin.

\noindent
\textbf{hagliga} \textit{adv.} neatly.

\noindent
\textbf{hagr} \textit{sm.} condition; advantage---`þēr mun h.\ ā vera,'
	will avail thee, be profitable to you.

\noindent
\textbf{hag-stœðr} \textit{adj.} favourable.

\noindent
\textbf{halda} \textit{sv.} 1, \textit{w.\ dat.} hold (also with prp.
	\textit{ā}); keep.  \textit{w.\ acc.} observe, keep (laws, etc.).
	\textit{intr.} take a certain direction, go.

\noindent
\textbf{hālfr} \textit{adj.} half.

\noindent
\textbf{hālfu} \textit{adv.} by half, half as much again.

\noindent
\textbf{hallar} \textit{} \textit{see} \textbf{hǫll}.

\noindent
\textbf{hallar-gōlf} \textit{sn.} hall floor.

\noindent
\textbf{haltr} \textit{adj.} lame.

\noindent
\textbf{hamarr} \textit{sm.} hammer.

\noindent
\textbf{hamars-muðr} \textit{sm.} thin end of hammer.

\noindent
\textbf{hamar-skapt} \textit{sn.} handle of a hammer.

\noindent
\textbf{hamar-spor} \textit{sn.} mark made by a hammer.

\noindent
\textbf{hana} \textit{} \textit{see} \textbf{hann}.

\noindent
\textbf{handar} \textit{} \textit{see} \textbf{hǫnd}.

\noindent
\textbf{hand-skot} \textit{sn.} throwing with the hand.

\noindent
\textbf{hand-tękinn} \textit{ptc.\ pret.} taken by hand, taken alive.

\noindent
\textbf{hann} \textit{prn.} he.

\noindent
\textbf{hār} \textit{sn.} hair.

\noindent
\textbf{hār} \textit{adj.} high.

\noindent
\textbf{harð-hugaðr} \textit{adj.} stern of mood.

\noindent
\textbf{harðr} \textit{adj.} hard; strong.

\noindent
\textbf{harð-skeytr} \textit{adj.} strong-shooting [skjōta].

\noindent
\textbf{harmr} \textit{sm.} grief.

\noindent
\textbf{hāski} \textit{wm.} danger.

\noindent
\textbf{hā-sæti} \textit{sn.} high seat, dais [sitja].

\noindent
\textbf{hāsætis-kista} \textit{sf.} chest under the dais.

\noindent
\textbf{hātt} \textit{adv.} high; loudly.

\noindent
\textbf{haugr} \textit{sm.} mound.

\noindent
\textbf{hauss} \textit{sm.} skull.

\noindent
\textbf{heðan} \textit{adv.} hence.

\noindent
\textbf{hęfi} \textit{} \textit{see} \textbf{hafa}.

\noindent
\textbf{hęfja} \textit{sv.} 2, raise, lift; begin.

\noindent
\textbf{hęfna} \textit{wv.} 1, \textit{w.\ gen.} revenge, avenge.

\noindent
\textbf{heill} \textit{adj.} sound, safe, prosperous.

\noindent
\textbf{heil-rǣði} \textit{sn.} sound advice, good advice [rāð].

\noindent
\textbf{heim} \textit{adv.} to one's home, home (domum).

\noindent
\textbf{heima} \textit{adv.} at home (domi).

\noindent
\textbf{heima-maðr} \textit{sm.} man of the household.

\noindent
\textbf{heim-fūss} \textit{adj.} longing to go home, homesick.

\noindent
\textbf{heimr} \textit{sm.} home, dwelling; world.

\noindent
\textbf{heimta} \textit{wv.} 1, fetch, obtain, get back.

\noindent
\textbf{heita} \textit{sv.} 1, call; \textit{w.\ dat.\ of pers.\ and dat.\ of
	thing} promise; \textit{intr.} (\textit{pres.} heiti) be named,
	called.

\noindent
\textbf{hęldr} \textit{} \textit{see} \textbf{halda}.

\noindent
\textbf{hęldr} \textit{adv.\ compar.} more willingly, rather, sooner,
	more; rather, very.

\noindent
\textbf{hęl-grindr} \textit{sf.} the doors of hell.

\noindent
\textbf{hęllir} \textit{sm.} cave.

\noindent
\textbf{helt} \textit{} \textit{see} \textbf{halda}.

\noindent
\textbf{hęl-vegr} \textit{sm.} road to hell.

\noindent
\textbf{hęlzt} \textit{adv.\ superl.} most willingly, soonest, especially,
	most [hęldr].

\noindent
\textbf{hęndi, hęndr} \textit{} \textit{see} \textbf{hǫnd}.

\noindent
\textbf{hęnnar, hęnni} \textit{} \textit{see} \textbf{hann}.

\noindent
\textbf{hępta} \textit{wv.} 1, bind; hinder, stop.

\noindent
\textbf{hēr} \textit{adv.} here---`h.\ af,' etc.\ here-of.

\noindent
\textbf{hęr-bergi} \textit{sn.} quarters, lodgings.

\noindent
\textbf{hęrða} \textit{wv.} harden, clench.

\noindent
\textbf{hęr-fang} \textit{sn.} booty.

\noindent
\textbf{hęrja} \textit{wv.} 3, make war, ravage [hęrr].

\noindent
\textbf{hęrr} \textit{sm.} army, fleet.

\noindent
\textbf{herra} \textit{sm.} lord.

\noindent
\textbf{hęr-skip} \textit{sn.} war-ship.

\noindent
\textbf{hestr} \textit{sm.} horse.

\noindent
\textbf{hēt} \textit{} \textit{see} \textbf{heita}.

\noindent
\textbf{heyra} \textit{wv.} 1, hear.

\noindent
\textbf{himinn} \textit{sm.} heaven.

\noindent
\textbf{hingat} \textit{adv.} hither.

\noindent
\textbf{hinn} \textit{prn.} that.

\noindent
\textbf{hirð} \textit{sf.} court.

\noindent
\textbf{hirð-maðr} \textit{sm.} courtier.

\noindent
\textbf{hiti} \textit{wm.} heat.

\noindent
\textbf{hitta} \textit{wv.} come upon, find, meet; \textit{trans.} go
	to.\  \textbf{hittask}, meet, \textit{intr.}

\noindent
\textbf{hjā} \textit{prp.} \textit{w.\ dat.} by, at; in comparison with.

\noindent
\textbf{hjālmr} \textit{sm.} helmet.

\noindent
\textbf{hjōn} \textit{snpl.} household.

\noindent
\textbf{hlaða} \textit{sv.} 2, load, heap up; `h.\ seglum,' take in sails.

\noindent
\textbf{hlæja} \textit{sv.} 2, laugh.

\noindent
\textbf{hlaupa} \textit{sv.} 1, jump, leap; run.

\noindent
\textbf{hlaut} \textit{} \textit{see} \textbf{hljōta}.

\noindent
\textbf{hleyp} \textit{} \textit{see} \textbf{hlaupa}.

\noindent
\textbf{hleypa} \textit{wv.} 1, make to run (i.e.\ the horse), gallop.

\noindent
\textbf{hlīf} \textit{sf.} shield.

\noindent
\textbf{hlīfa} \textit{wv.} 1, \textit{w.\ dat.} shelter, cover, protect.

\noindent
\textbf{hljōp} \textit{} \textit{see} \textbf{hlaupa}.

\noindent
\textbf{hljōta} \textit{sv.} 7, get, receive.

\noindent
\textbf{hlōðu} \textit{} \textit{see} \textbf{hlaða}.

\noindent
\textbf{hlōgu} \textit{} \textit{see} \textbf{hlæja}.

\noindent
\textbf{hlunnr} \textit{sm.} roller (for launching ships).

\noindent
\textbf{hluti} \textit{wm.} portion [hljōta].

\noindent
\textbf{hlutr} \textit{sm.} 2, share; portion, part, piece; thing [hljōta].

\noindent
\textbf{hlut-skipti} \textit{sn.} booty.

\noindent
\textbf{hnakki} \textit{wm.} back of head.

\noindent
\textbf{hōf} \textit{} \textit{see} \textbf{hęfja}.

\noindent
\textbf{hollr} \textit{adj.} gracious, faithful.

\noindent
\textbf{hōlmr} \textit{sm.} small island.

\noindent
\textbf{hon} \textit{} \textit{see} \textbf{hann}.

\noindent
\textbf{honum} \textit{} \textit{see} \textbf{hann}.

\noindent
\textbf{horn} \textit{sn.} horn.

\noindent
\textbf{hœgri} \textit{adj.\ compar.} right (hand).

\noindent
\textbf{hœla} \textit{wv.} 1, \textit{w.\ dat.} praise, boast of.

\noindent
\textbf{hǫfðingi} \textit{wm.} chief [hǫfuð].

\noindent
\textbf{hǫfða-fjǫl} \textit{sf.} head-board (especially of a bedstead).

\noindent
\textbf{hǫfðu} \textit{} \textit{see} \textbf{hafa}.

\noindent
\textbf{hǫfn} \textit{sf.} harbour.

\noindent
\textbf{hǫfuð} \textit{sn.} head.

\noindent
\textbf{hǫgg} \textit{sn.} stroke.

\noindent
\textbf{hǫgg-ormr} \textit{sm.} viper.

\noindent
\textbf{hǫgg-orrosta} \textit{wf.} `cutting-fight,' hand-to-hand fight.

\noindent
\textbf{hǫggva} \textit{sv.} 1, hew, cut, strike.

\noindent
\textbf{hǫll} \textit{sf.} hall.

\noindent
\textbf{hǫnd} \textit{sf.} 3, hand; side---`hvārra-tvęggju handar,' on both
    sides, for both parties.

\noindent
\textbf{hræddr} \textit{adj.} frightened, afraid [\textit{ptc.} of hræðask].

\noindent
\textbf{hræðask} \textit{wv.} 1, be frightened, fear.

\noindent
\textbf{hræzla} \textit{wf.} fear [hræðask].

\noindent
\textbf{hrafn} \textit{sm.} raven.

\noindent
\textbf{hratt} \textit{} \textit{see} \textbf{hrinda}.

\noindent
\textbf{hrauð} \textit{} \textit{see} \textbf{hrjōða}.

\noindent
\textbf{hraut} \textit{} \textit{see} \textbf{hrjōta}.

\noindent
\textbf{hreyfa} \textit{wv.} 1, move.

\noindent
\textbf{hrīð} \textit{sf.} period of time.

\noindent
\textbf{hrīm-þurs} \textit{sm.} frost giant.

\noindent
\textbf{hrinda} \textit{sv.} 3, push, launch (ship).

\noindent
\textbf{hrista} \textit{wv.} 1, shake.

\noindent
\textbf{hrjōða} \textit{sv.} 7, strip, clear, disable.

\noindent
\textbf{hrjōta} \textit{sv.} 7, start, burst out.

\noindent
\textbf{hrǫkk} \textit{} \textit{see} \textbf{hrökkva}.

\noindent
\textbf{hrökkva} \textit{sv.} 3, start back.

\noindent
\textbf{hryði} \textit{} \textit{see} \textbf{hrjōta}.

\noindent
\textbf{hrynja} \textit{wv.} 1b, fall down.

\noindent
\textbf{hrȳtr} \textit{} \textit{see} \textbf{hrjōta}.

\noindent
\textbf{hugða} \textit{} \textit{see} \textbf{hyggja}.

\noindent
\textbf{hugi} \textit{wm.} thought.

\noindent
\textbf{hugr} \textit{sm.} mind, heart; courage, spirit.

\noindent
\textbf{hugsa} \textit{wv.} 3, consider, think.

\noindent
\textbf{hundrað} \textit{sn.} hundred.

\noindent
\textbf{hurð} \textit{sf.} 2, door.

\noindent
\textbf{hūs} \textit{sn.} room; house.

\noindent
\textbf{hvar} \textit{adv.} where; that.

\noindent
\textbf{hvārr} \textit{prn.} which of two; each of the two, both.

\noindent
\textbf{hvārr-tvęggja} \textit{prn.} each of the two, both.

\noindent
\textbf{hvārt} \textit{adv.} whether, \textit{both in direct and indirect
	questions}.

\noindent
\textbf{hvārt-tvęggja} \textit{adv.} `hv\ldots ok,' both\ldots and.

\noindent
\textbf{hvass} \textit{adj.} sharp.

\noindent
\textbf{hvat} \textit{prn.\ neut.} what.

\noindent
\textbf{hvart} \textit{adj.} brisk, bold.

\noindent
\textbf{hvē} \textit{adv.} how.

\noindent
\textbf{hverfa} \textit{sv.} 3, turn, go.

\noindent
\textbf{hvęr-gi} \textit{adv.} nowhere---`hv.\ jarðar,' nowhere on
	earth; in no respect, by no means.

\noindent
\textbf{hvęrnig} \textit{adv.} how [= hvęrn veg.].

\noindent
\textbf{hvęrr} \textit{prn.} who.

\noindent
\textbf{hvęrsu} \textit{adv.} how.

\noindent
\textbf{hvī} \textit{adv.} why.

\noindent
\textbf{hvirfill} \textit{sm.} crown of head.

\noindent
\textbf{hvītna} \textit{wv.} 2, whiten.

\noindent
\textbf{hvītr} \textit{adj.} white.

\noindent
\textbf{hyggja} \textit{wv.} 1b, think, mean, determine [hugr].

\noindent
\textbf{hylli} \textit{wf.} favour [hollr].

\emptypage

\chapter*{I}

\noindent
\textbf{ī} \textit{prp.} in.

\noindent
\textbf{ī-huga} \textit{wv.} 3, try to remember, consider [hugr].

\noindent
\textbf{illa} \textit{adv.} ill, badly.

\noindent
\textbf{illr} \textit{adj.} ill, bad.

\noindent
\textbf{inn} \textit{art.} the.

\noindent
\textbf{inn} \textit{adv.} in \textit{compar.} \textbf{innar}, further in.

\noindent
\textbf{inna} \textit{wv.} 1, accomplish.

\noindent
\textbf{innan} \textit{adv.} within, inside.  \textbf{fyrir innan} \textit{prp.
	w.\ gen.} within, in.

\noindent
\textbf{inni} \textit{adv.} in.

\noindent
\textbf{it} \textit{} \textit{see} \textbf{þū}.

\noindent
\textbf{it} \textit{} \textit{see} \textbf{inn}.

\noindent
\textbf{ī-þrōtt} \textit{sf.} feat.

\emptypage

\chapter*{J}

\noindent
\textbf{jafn-breiðr} \textit{adj.} equally broad.

\noindent
\textbf{jafn-hǫfugr} \textit{adj.} equally heavy [hęfja].

\noindent
\textbf{jafn-mikill} \textit{adj.} equally great.

\noindent
\textbf{jafn-skjōtt} \textit{adv.} equally quick.

\noindent
\textbf{jafna} \textit{wv.} 3, smooth; compare \textit{w.\ dat.\ of
	thing compared.}

\noindent
\textbf{jafnan} \textit{adv.} always.

\noindent
\textbf{jarl} \textit{sm.} earl.

\noindent
\textbf{jārn} \textit{sn.} iron.

\noindent
\textbf{jārn-glōfi} \textit{sm.} iron gauntlet.

\noindent
\textbf{jārn-spǫng} \textit{sf.} iron plate.

\noindent
\textbf{jāta} \textit{wv.} 1, \textit{w.\ dat.} agree to.

\noindent
\textbf{jōk} \textit{} \textit{see} \textbf{auka}.

\noindent
\textbf{jǫrð} \textit{sf.} earth.

\noindent
\textbf{jǫtun-heimar} \textit{smpl.} home, world of the giants.

\noindent
\textbf{jǫtunn,} \textit{sm.} giant.

\emptypage

\chapter*{K}

\noindent
\textbf{kær-leikr} \textit{sm.} love, affection [kærr, `dear'].

\noindent
\textbf{kaf} \textit{sn.} diving; deep water, water under the surface.

\noindent
\textbf{kafa} \textit{wv.} 3, dive.

\noindent
\textbf{kalla} \textit{wv.} 3, cry out, call; assert, maintain; name, call.

\noindent
\textbf{kann} \textit{} \textit{see} \textbf{kunna}.

\noindent
\textbf{kapp} \textit{sn.} competition.

\noindent
\textbf{karl} \textit{sm.} man; old man.

\noindent
\textbf{kasta} \textit{wv.} 3, cast, throw.

\noindent
\textbf{kaupa} \textit{wv.} 2, buy.

\noindent
\textbf{kęngr} \textit{sm.} bend.

\noindent
\textbf{kęnna} \textit{wv.} 1, know; perceive.

\noindent
\textbf{kęrling} \textit{sf.} old woman [karl].

\noindent
\textbf{kęrra} \textit{wf.} chariot.

\noindent
\textbf{kętill} \textit{sm.} kettle.

\noindent
\textbf{keypta} \textit{} \textit{see} \textbf{kaupa}.

\noindent
\textbf{keyra} \textit{wv.} 1, drive.

\noindent
\textbf{kirkja} \textit{wf.} church.

\noindent
\textbf{kirkju-skot} \textit{sn.} wing of a church.

\noindent
\textbf{kjōsa} \textit{sv.} 7, choose.

\noindent
\textbf{klakk-laust} \textit{adv.} uninjured.

\noindent
\textbf{klæða} \textit{wv.} 1, clothe.

\noindent
\textbf{klæða-būnaðr} \textit{sm.} apparel.

\noindent
\textbf{klæði} \textit{snpl.} clothes.

\noindent
\textbf{knē} \textit{sn.} knee.

\noindent
\textbf{knīfr} \textit{sm.} knife.

\noindent
\textbf{knǫrr} \textit{sm.} merchant-ship.

\noindent
\textbf{knūði} \textit{} \textit{see} \textbf{knȳja}.

\noindent
\textbf{knūi} \textit{wm.} knuckle.

\noindent
\textbf{knūtr} \textit{sm.} knot.

\noindent
\textbf{knȳja} \textit{wv.} 1b, press with knuckles or knees; exert
	oneself [knūi].

\noindent
\textbf{kōlf-skot} \textit{sn.} (distance of a) bolt-shot.

\noindent
\textbf{kollōttr} \textit{adj.} bald.

\noindent
\textbf{koma} \textit{sv.} 4, come; happen, turn out; \textit{w.\ dat.}
	bring into a certain condition.  \textbf{k.\ fyrir}, be paid in
	atonement.  \textbf{komask}, make one's way (by dint of exertion).

\noindent
\textbf{kona} \textit{wf.} woman; wife.

\noindent
\textbf{konr} \textit{sm.} kind---`alls konar,' all kinds; `nakkvars konar,'
	of some kind.

\noindent
\textbf{konunga-stęfna} \textit{wf.} congress of kings.

\noindent
\textbf{konungr} \textit{sm.} king.

\noindent
\textbf{konungs-dōttir} \textit{sf.} king's daughter.

\noindent
\textbf{konungs-skip} \textit{sn.} king's ship.

\noindent
\textbf{kosinn} \textit{} \textit{see} \textbf{kjōsa}.

\noindent
\textbf{kost-gripr} \textit{sm.} precious thing, treasure.

\noindent
\textbf{kostr} \textit{sm.} 2, choice---`at ǫðru kosti,' otherwise;
	power [kjōsa].

\noindent
\textbf{kǫgur-sveinn} \textit{sm.} little boy, urchin.

\noindent
\textbf{kǫpp} \textit{} \textit{see} \textbf{kapp}.

\noindent
\textbf{kǫpur-yrði} \textit{sn.} boasting [orð].

\noindent
\textbf{kǫttr} \textit{sm.} 3, cat.

\noindent
\textbf{köm} \textit{} \textit{see} \textbf{koma}.

\noindent
\textbf{krappa-rūm} \textit{sn.} back cabin.

\noindent
\textbf{krappr} \textit{adj.} narrow.

\noindent
\textbf{kraptr} \textit{sm.} strength.

\noindent
\textbf{krās} \textit{sf.} 2, delicacy.

\noindent
\textbf{kręfja} \textit{wv.} 1b, \textit{w.\ acc.\ of pers.\ and gen.\ of
	thing}, demand.

\noindent
\textbf{kunna} \textit{swv.} know; feel; venture; like to.

\noindent
\textbf{kunnandi} \textit{wf.} knowledge, accomplishments.

\noindent
\textbf{kunnusta} \textit{wf.} knowledge, power.

\noindent
\textbf{kunnr} \textit{adj.} known.

\noindent
\textbf{kurr} \textit{sm.} murmur, rumour.

\noindent
\textbf{kvæði} \textit{sn.} poem.

\noindent
\textbf{kvæma} \textit{} \textit{see} \textbf{koma}.

\noindent
\textbf{kvān} \textit{sf.} wife.

\noindent
\textbf{kveða} \textit{sv.} 5, say.  \textbf{kv.\ ā} settle, agree on.

\noindent
\textbf{kvęðja} \textit{wf.} salutation [kveða].

\noindent
\textbf{kvęðja} \textit{wv.} 1b, greet.

\noindent
\textbf{kveld} \textit{sn.} evening---`ī kv.,' this evening.

\noindent
\textbf{kveld-sǫngr} \textit{sm.} vespers.

\noindent
\textbf{kvenn-vāðir} \textit{sfpl.} woman's clothes.

\noindent
\textbf{kviðr} \textit{sm.} 3, stomach, belly.

\noindent
\textbf{kvisa} \textit{wv.} 3, whisper.

\noindent
\textbf{kvistr} \textit{sm.} 3, branch, twig.

\noindent
\textbf{kvǫddu} \textit{} \textit{see} \textbf{kvęðja}.

\noindent
\textbf{kykr (-vir)} \textit{adj.} living.

\noindent
\textbf{kykvendi} \textit{sn.} living creature, animal.

\noindent
\textbf{kȳll} \textit{sm.} bag.

\noindent
\textbf{kyn} \textit{sn.} race, lineage.

\noindent
\textbf{kȳr} \textit{sf.} 3, cow.

\noindent
\textbf{kyssa} \textit{wv.} 1, kiss.

\emptypage

\chapter*{L}

\noindent
\textbf{lā} \textit{} \textit{see} \textbf{liggja}.

\noindent
\textbf{lægri} \textit{} \textit{see} \textbf{lāgr}.

\noindent
\textbf{lær-lęggr} \textit{sm.} thigh-bone.

\noindent
\textbf{læt} \textit{} \textit{see} \textbf{lāta}.

\noindent
\textbf{lagða} \textit{} \textit{see} \textbf{lęggja}.

\noindent
\textbf{lāgr} \textit{adj.} low; short of stature [liggja].

\noindent
\textbf{lags-maðr} \textit{sm.} companion.

\noindent
\textbf{lagu} \textit{} \textit{see} \textbf{liggja}.

\noindent
\textbf{lamða, lamit} \textit{} \textit{see} \textbf{lęmja}.

\noindent
\textbf{land} \textit{sn.} land, country.

\noindent
\textbf{land-skjālfti} \textit{wm.} earthquake.

\noindent
\textbf{langr} \textit{adj.} long, far.

\noindent
\textbf{lāt} \textit{snpl.} noise.

\noindent
\textbf{lāta} \textit{sv.} 1, let go; leave; lose; allow; cause, let;
	behave, act; say.

\noindent
\textbf{lauf} \textit{sn.} foliage.

\noindent
\textbf{laufs-blað} \textit{sn.} (blade of foliage), leaf.

\noindent
\textbf{laug} \textit{sf.} bath.

\noindent
\textbf{laun} \textit{snpl.} reward.

\noindent
\textbf{launa} \textit{wv.} 3, reward, requite \textit{w.\ dat.\ of the
	thing given and of the pers., and acc.\ of the thing requited.}

\noindent
\textbf{lauss} \textit{adj.} loose; shaky, unsteady; free from obligation.

\noindent
\textbf{laust} \textit{} \textit{see} \textbf{ljōsta}.

\noindent
\textbf{laut} \textit{} \textit{see} \textbf{lūta}.

\noindent
\textbf{lax} \textit{sm.} salmon.

\noindent
\textbf{leðr-hosa} \textit{wf.} leather bag.

\noindent
\textbf{lęggja} \textit{wv.} 1b, lay, put; `l.\ eitt fyrir einn,' give,
	settle on; `l.\ sik fram,' exert oneself; \textit{intr.\ w.} skip
    \textit{understood} sail, row---\textbf{l.\ at}, land; attack; \textbf{l.\ frā},
    retreat, draw off; pierce, make a thrust.  \textbf{lęggjask}, set
    out, proceed; swim [liggja].

\noindent
\textbf{leið} \textit{sf.} way---`koma einu til leiðar,' carry out [līða].

\noindent
\textbf{leið} \textit{} \textit{see} \textbf{līða}.

\noindent
\textbf{leiða} \textit{wv.} 1, lead, conduct [līða].

\noindent
\textbf{leiðangr (-rs)} \textit{sm.} levy [leið].

\noindent
\textbf{leiga} \textit{wv.} 1, borrow.

\noindent
\textbf{leikr} \textit{sm.} game; athletic sports, contest.

\noindent
\textbf{leita} \textit{wv.} 3, \textit{w.\ gen.\ and dat.} seek; take to, have
	recourse to.  \textbf{leitask}, feel one's way [līta].

\noindent
\textbf{lęmja} \textit{wv.} 1b, break.

\noindent
\textbf{lęngð} \textit{sf.} length [langr].

\noindent
\textbf{lęngi} \textit{adv.} long (of time) [langr].

\noindent
\textbf{lęngstr} \textit{} \textit{see} \textbf{langr}.

\noindent
\textbf{lēt} \textit{} \textit{see} \textbf{lāta}.

\noindent
\textbf{lętja} \textit{wv.} 1b, \textit{w.\ acc.\ of pers.\ and gen.\ of thing},
	hinder, dissuade.

\noindent
\textbf{lētta} \textit{wv.} 1, \textit{w.\ dat.} lift.

\noindent
\textbf{leyniliga} \textit{adv.} secretly.

\noindent
\textbf{leysa} \textit{wv.} 1, loosen, untie, open [lauss].

\noindent
\textbf{lið} \textit{sn.} troop.

\noindent
\textbf{līða} \textit{sv.} 6, go; pass (of time); \textit{impers.} \textbf{līðr},
	\textit{w.\ dat.} fare, get on.  \textit{impers.} `līðr ā (nāttina),' (the
	night) is drawing to a close.

\noindent
\textbf{līf} \textit{sn.} life---`ā līfi,' alive.

\noindent
\textbf{lifa} \textit{wv.} 2, live.

\noindent
\textbf{liggja} \textit{sv.} 5, lie.  \textbf{l.\ til}, be fitting.

\noindent
\textbf{līk} \textit{sn.} body; corpse.

\noindent
\textbf{līka} \textit{wv.} 3, \textit{w.\ dat.} please.

\noindent
\textbf{līki} \textit{sn.} form [līk].

\noindent
\textbf{līking} \textit{sf.} likeness, similarity [līkr].

\noindent
\textbf{līknsamr} \textit{adj.} gracious.

\noindent
\textbf{līkr} \textit{adj.} like.

\noindent
\textbf{līn} \textit{sn.} linen; linen headdress.

\noindent
\textbf{list} \textit{sf.} art.

\noindent
\textbf{līta} \textit{sv.} 6, look at; regard, consider---`l.\ til
	eins.' turn to, acknowledge greeting.  \textbf{lītask} \textit{impers.
	w.\ dat.} seem.

\noindent
\textbf{litask} \textit{wv.} 3, look round one [līta].

\noindent
\textbf{lītill} \textit{adj.} little, small---`lītit veðr,' not very
	windy weather.  \textbf{lītlu} \textit{adv.} by a little, a little.

\noindent
\textbf{lītil-ræði} \textit{sm.} degradation [rāð].

\noindent
\textbf{litr} \textit{sm.} 3, colour, complexion; appearance [līta].

\noindent
\textbf{līzk} \textit{} \textit{see} \textbf{līta}.

\noindent
\textbf{ljā} \textit{wv.} 1, \textit{w.\ gen.\ and dat.} lend.

\noindent
\textbf{ljōsta} \textit{sv.} 7, strike, \textit{w.\ dat.\ of instr.\ and acc.\ of
	the thing struck}---`l.\ ārum ī sæ,' dip the oars into the sea,
	begin to row.

\noindent
\textbf{lofa} \textit{wv.} 3, praise.

\noindent
\textbf{lōga} \textit{wv.} 3, \textit{w.\ dat.} part with.

\noindent
\textbf{logi} \textit{wm.} flame.

\noindent
\textbf{lokinn} \textit{} \textit{see} \textbf{lūka}.

\noindent
\textbf{lopt} \textit{sn.} air---`ā l.,' up.

\noindent
\textbf{lūka} \textit{sv.} 7, lock, close; \textit{impers.} `lȳkr einu,'
	it is finished, exhausted.  `l.\ upp,' unlock, open.

\noindent
\textbf{lukla} \textit{} \textit{see} \textbf{lykill}.

\noindent
\textbf{lustu} \textit{} \textit{see} \textbf{ljōsta}.

\noindent
\textbf{lūta} \textit{sv.} 7, bend, bow.

\noindent
\textbf{lygi} \textit{wf.} lie, falsehood.

\noindent
\textbf{lykð} \textit{sf.} ending---`at lykðum,' finally.

\noindent
\textbf{lykill} \textit{sm.} key [lūka].

\noindent
\textbf{lypta} \textit{wv.} 1, \textit{w.\ dat.} lift [lopt].

\noindent
\textbf{lypting} \textit{sf.} raised place (castle) on the poop of a warship
    [lypta].

\noindent
\textbf{lȳsa} \textit{wv.} 1, shine.

\noindent
\textbf{lȳsi-gull} \textit{sn.} bright gold.

\noindent
\textbf{lȳst} \textit{} \textit{see} \textbf{ljōsta}.

\noindent
\textbf{lysta} \textit{wv.} 1, \textit{impers.\ w.\ acc.} desire.

\emptypage

\chapter*{M}

\noindent
\textbf{mā} \textit{} \textit{see} \textbf{mega}.

\noindent
\textbf{maðr} \textit{sm.} 4, man.

\noindent
\textbf{mæla (mælta)} \textit{wv.} 1, speak---`m.\ við einu,' refuse; suggest.

\noindent
\textbf{mær} \textit{sf.} virgin, maid.

\noindent
\textbf{mætta} \textit{} \textit{see} \textbf{mega}.

\noindent
\textbf{magr (-ran)} \textit{adj.} thin.

\noindent
\textbf{māgr} \textit{sm.} kinsman, relation, connection.

\noindent
\textbf{makligr} \textit{adj.} fitting.

\noindent
\textbf{māl} \textit{sn.} narrative; \textit{in plur.} poem; proper time, time.

\noindent
\textbf{mālmr} \textit{sm.} metal.

\noindent
\textbf{mann} \textit{} \textit{see} \textbf{maðr}.

\noindent
\textbf{mann-fōlk} \textit{sn.} troops, crew.

\noindent
\textbf{mann-hringr} \textit{sm.} ring of men.

\noindent
\textbf{mannliga} \textit{adv.} manlily.

\noindent
\textbf{margr} \textit{adj.} many, much.

\noindent
\textbf{mark} \textit{sn.} mark; importance.

\noindent
\textbf{marka} \textit{wv.} 3, infer.

\noindent
\textbf{marr} \textit{sm.} horse.

\noindent
\textbf{mart} \textit{} \textit{see} \textbf{margr}.

\noindent
\textbf{matask} \textit{wv.} 3, eat a meal.

\noindent
\textbf{matr} \textit{sm.} 2, food.

\noindent
\textbf{mātta} \textit{} \textit{see} \textbf{mega}.

\noindent
\textbf{māttr} \textit{sm.} 2, might, strength.

\noindent
\textbf{með} \textit{prp.} \textit{w.\ acc.\ and dat.} with.

\noindent
\textbf{meðal} \textit{s.} middle---ā m.  \textit{w.\ gen.} between.

\noindent
\textbf{meðan} \textit{adv.} whilst.

\noindent
\textbf{mega} \textit{swv.} can, may.

\noindent
\textbf{megin} \textit{} ---`ǫðrum m.,' on the other side; `ǫllum m.,' on all
	sides [\textit{corruption of} vegum].

\noindent
\textbf{męgin-gjarðar} \textit{sfpl.} girdle of strength [mega].

\noindent
\textbf{meiðmar} \textit{sfpl.} treasures.

\noindent
\textbf{meiri} \textit{} \textit{see} \textbf{mikill}.

\noindent
\textbf{męn} \textit{sn.} necklace, piece of jewelry.

\noindent
\textbf{męnn} \textit{} \textit{see} \textbf{maðr}.

\noindent
\textbf{mēr} \textit{} \textit{see} \textbf{ek}.

\noindent
\textbf{męrgr} \textit{sm.} 2, marrow.

\noindent
\textbf{męrki} \textit{sn.} mark; banner [mark].

\noindent
\textbf{mest} \textit{} \textit{see} \textbf{mikill}.

\noindent
\textbf{meta} \textit{sv.} 5, measure; estimate.

\noindent
\textbf{mey} \textit{} \textit{see} \textbf{mær}.

\noindent
\textbf{mið-garðr} \textit{sm.} (middle enclosure), world.

\noindent
\textbf{miðgarðs-ormr} \textit{sm.} world-serpent.

\noindent
\textbf{miðr} \textit{adj.} middle.

\noindent
\textbf{mik} \textit{} \textit{see} \textbf{ek}.

\noindent
\textbf{mikill} \textit{adj.} big, tall, great.  \textbf{mikit} \textit{adv.}
	much, very.

\noindent
\textbf{miklu} \textit{adv.} (\textit{instr.}) much.

\noindent
\textbf{milli, ā milli} \textit{prp.} between, among.

\noindent
\textbf{minjar} \textit{sfpl.} remembrance, memorial.

\noindent
\textbf{minn} \textit{} \textit{see} \textbf{ek}.

\noindent
\textbf{minni} \textit{} \textit{see} \textbf{lītill}.

\noindent
\textbf{minnr} \textit{adv.} less.

\noindent
\textbf{mis-līka} \textit{wv.} 3, \textit{w.\ dat.} displease, not please.

\noindent
\textbf{missa} \textit{wf.} loss, want.

\noindent
\textbf{missa} \textit{wv.} 1, \textit{w.\ gen.} lose; do without.

\noindent
\textbf{mistil-teinn} \textit{sm.} mistletoe [teinn, `twig'].

\noindent
\textbf{mitt} \textit{} \textit{see} \textbf{ek}.

\noindent
\textbf{mjǫðr} \textit{sm.} 3, mead.

\noindent
\textbf{mjǫk} \textit{adv.} very.

\noindent
\textbf{mōðir} \textit{sf.} 3, mother.

\noindent
\textbf{mōðr} \textit{sm.} anger.

\noindent
\textbf{mœtask} \textit{wv.} 1, meet \textit{intr.} [mōt].

\noindent
\textbf{mǫn} \textit{sf.} mane.

\noindent
\textbf{morginn} \textit{sm.} morning.

\noindent
\textbf{mǫrk} \textit{sf.} 3, forest.

\noindent
\textbf{mǫrum} \textit{} \textit{see} \textbf{marr}.

\noindent
\textbf{mōt} \textit{sn.} meeting.  \textbf{ī mōti} \textit{prp.\ w.\ dat.}
	against.

\noindent
\textbf{mǫtu-neyti} \textit{sn.} community of food---`lęggja m.\ sitt,'
	make their provision into a common store [matr; njōta].

\noindent
\textbf{mundu} \textit{} \textit{see} \textbf{munu}.

\noindent
\textbf{munnr} \textit{sm.} mouth.

\noindent
\textbf{munr} \textit{sm.} difference---`þeim mun,' to that extent.

\noindent
\textbf{munu} \textit{swv.} will, may (of futurity and probability).

\noindent
\textbf{myndi} \textit{} \textit{see} \textbf{munu}.

\noindent
\textbf{myrkr} \textit{sn.} darkness.

\noindent
\textbf{myrkr} \textit{adj.} dark.

\emptypage

\chapter*{N}

\noindent
\textbf{nā} \textit{wv.} 1, reach, obtain; succeed in.

\noindent
\textbf{nǣr} \textit{adv.} \textit{w.\ dat.} near; nearly.  \textit{superl.}
	\textbf{næst}---`þvī n.,' thereupon.

\noindent
\textbf{nǣtr} \textit{} \textit{see} \textbf{nātt}.

\noindent
\textbf{nafn} \textit{sn.} name.

\noindent
\textbf{nakkvarr} \textit{prn.} some, a certain.  \textbf{nakkvat} \textit{adv.}
	somewhat; perhaps.

\noindent
\textbf{nātt} \textit{sf.} 3, night.

\noindent
\textbf{nātt-bōl} \textit{sn.} night-quarters.

\noindent
\textbf{nātt-langt} \textit{adv.} the whole night long.

\noindent
\textbf{nātt-staðr} \textit{sm.} night-quarters.

\noindent
\textbf{nāttūra} \textit{wf.} nature, peculiarity.

\noindent
\textbf{nātt-verðr} \textit{sm.} 2, supper.

\noindent
\textbf{nauð-syn} \textit{sf.} necessity.

\noindent
\textbf{naut} \textit{} \textit{see} \textbf{njōta}.

\noindent
\textbf{neðan} \textit{adv.} below.  \textbf{fyrir n.} \textit{prp.\
	w.\ dat.} below.

\noindent
\textbf{nęfja} \textit{adj.} long-nosed (?).

\noindent
\textbf{nęfna} \textit{wv.} 1, name, call.  \textbf{nęfnask}, name
	oneself, give one's name as [nafn].

\noindent
\textbf{nema} \textit{sv.} 4, take; begin.

\noindent
\textbf{nema} \textit{adv.} except, unless.

\noindent
\textbf{nest} \textit{sn.} provisions.

\noindent
\textbf{nest-baggi} \textit{sm.} provision-bag.

\noindent
\textbf{niðr} \textit{adv.} down, downwards.

\noindent
\textbf{nīundi} \textit{adj.} ninth.

\noindent
\textbf{njōsn} \textit{sf.} spying; news.

\noindent
\textbf{njōsna} \textit{wv.} 3, spy; get intelligence.

\noindent
\textbf{njōta} \textit{sv.} 7, enjoy, profit.

\noindent
\textbf{norðr} \textit{adv.} northwards.

\noindent
\textbf{nǫkkvi} \textit{wm.} vessel, small ship.

\noindent
\textbf{nū} \textit{adv.} now; therefore, so.

\noindent
\textbf{nȳ-vaknaðr} \textit{adj.\ (ptc.)} newly awoke.

\noindent
\textbf{nȳr} \textit{adj.} new.

\noindent
\textbf{nȳta} \textit{wv.} 1, profit [njōta].

\emptypage

\chapter*{O}

\noindent
\textbf{oddr} \textit{sm.} point.

\noindent
\textbf{ōð-fūss} \textit{adj.} madly eager.

\noindent
\textbf{ōðr} \textit{adj.} mad, furious.

\noindent
\textbf{œ̄pa} \textit{wv.} 1, shout [ōp, `shout'].

\noindent
\textbf{œrit} \textit{adv.} enough; very.

\noindent
\textbf{of} \textit{prp.} \textit{w.\ dat.} over; during; with respect
	to, about.  \textit{adv.} too (of excess).

\noindent
\textbf{of} \textit{adv.,} \textit{often used in poetry as a mere expletive.}

\noindent
\textbf{of-veikr} \textit{adj.} too weak.

\noindent
\textbf{ofan} \textit{adv.} above; down.

\noindent
\textbf{ofan-verðr} \textit{adj.} upper, on the top.

\noindent
\textbf{ōk} \textit{} \textit{see} \textbf{aka}.

\noindent
\textbf{ok} \textit{conj.} and; also---`ok\ldots ok,' both\ldots and; but.

\noindent
\textbf{okkr} \textit{} \textit{see} \textbf{þū}.

\noindent
\textbf{opt} \textit{adv.} often.  \textit{compar.} \textbf{optar}, oftener,
	again.

\noindent
\textbf{ōr} \textit{prp.} \textit{w.\ dat.} out of.

\noindent
\textbf{orð} \textit{sn.} word---`ī ǫðru orði,' otherwise.

\noindent
\textbf{orðinn} \textit{} \textit{see} \textbf{verða}.

\noindent
\textbf{orð-sęnding} \textit{sf.} verbal message.

\noindent
\textbf{orð-tak} \textit{sn.} expression, word.

\noindent
\textbf{ormr} \textit{sm.} serpent, dragon; ship with a dragon's head.

\noindent
\textbf{orrosta} \textit{wf.} battle.

\noindent
\textbf{oss} \textit{} \textit{see} \textbf{ek}.

\noindent
\textbf{ōtta} \textit{wf.} the end of night, just before dawn.

\noindent
\textbf{ōttalaust} \textit{adj.} without fear.

\noindent
\textbf{ōx} \textit{} \textit{see} \textbf{vaxa}.

\noindent
\textbf{oxi} \textit{wm.} ox.

\noindent
\textbf{ǫðlask} \textit{wv.} 3, obtain.

\noindent
\textbf{ǫðru} \textit{} \textit{see} \textbf{annarr}.

\noindent
\textbf{ǫl} \textit{sn.} ale.

\noindent
\textbf{ǫldnu} \textit{} \textit{see} \textbf{aldinn}.

\noindent
\textbf{ǫndōttr} \textit{adj.} fierce.

\noindent
\textbf{ǫnd-vegi} \textit{sn.} high seat, dais.

\noindent
\textbf{ǫnnur} \textit{} \textit{see} \textbf{annar}.

\noindent
\textbf{ǫr} \textit{sf.} arrow.

\noindent
\textbf{ör-æfi} \textit{sn.} harbourless coast.

\noindent
\textbf{örendi} \textit{sn.} errand.

\noindent
\textbf{örendi} \textit{sn.} holding the breath, breath.

\noindent
\textbf{örind-reki} \textit{sm.} messenger [reka].

\noindent
\textbf{öxn} \textit{} \textit{see} \textbf{oxi}.

\emptypage

\chapter*{P}

\noindent
\textbf{pati} \textit{sm.} rumour.

\noindent
\textbf{pęnningr} \textit{sm.} penny.

\emptypage

\chapter*{R}

\noindent
\textbf{rāð} \textit{sn.} advice; what is advisable---`sjā eitt at rāði,'
    consider advisable; plan, policy, resolution.

\noindent
\textbf{rāða} \textit{sv.} 1, advise, \textit{w.\ acc.\ of thing and dat.\ of pers.};
    consider, deliberate; undertake, begin \textit{w.\ prp.} til \textit{or
    infin.}; dispose of, have control over \textit{w.\ prp.} fyrir.

\noindent
\textbf{rāða-gǫ̈rð} \textit{sf.} deliberation, decision.

\noindent
\textbf{rāðugr} \textit{adj.} sagacious.

\noindent
\textbf{ragna-rökr} \textit{sn.} twilight of the gods, end of the world.  [ragna
    \textit{gen.\ of} ręgin \textit{neut.\ plur.} `gods.']

\noindent
\textbf{ragr} \textit{adj.} cowardly.

\noindent
\textbf{rāku} \textit{} \textit{see} \textbf{reka}.

\noindent
\textbf{rann} \textit{} \textit{see} \textbf{ręnna}.

\noindent
\textbf{rās} \textit{sf.} race.

\noindent
\textbf{rauðr} \textit{adj.} red.

\noindent
\textbf{rausn} \textit{sf.} magnificence, anything magnificent.

\noindent
\textbf{rēð} \textit{} \textit{see} \textbf{rāða}.

\noindent
\textbf{reið} \textit{sf.} chariot.

\noindent
\textbf{reið} \textit{} \textit{see} \textbf{rīða}.

\noindent
\textbf{reiða} \textit{wv.} 1, swing, wield, brandish.

\noindent
\textbf{reið-fara} \textit{adj.} ---`vera vel r.,' have a good passage.

\noindent
\textbf{reiði} \textit{sn.} trappings, harness.

\noindent
\textbf{reiðr} \textit{adj.} angry.

\noindent
\textbf{reka} \textit{sv.} 7, drive; carry out; perform.  `r.\ af tjǫld,' take
    down awning.

\noindent
\textbf{rękkja} \textit{wf.} bed.

\noindent
\textbf{ręnna} \textit{sv.} 3, run.

\noindent
\textbf{rētta} \textit{wv.} 1, direct; reach, stretch.  \textbf{r.\ upp}, pull
	up.

\noindent
\textbf{rēttr} \textit{adj.} right, correct; equitable, fair.

\noindent
\textbf{reyna} \textit{wv.} 1, try, test.

\noindent
\textbf{reyr-bǫnd} \textit{supl.} the wire with which the arrow-head was bound
    to the shaft.

\noindent
\textbf{rīða} \textit{sv.} 6, \textit{w.\ dat.} ride.

\noindent
\textbf{riðu} \textit{} \textit{see} \textbf{rīða}.

\noindent
\textbf{riðlask} \textit{wv.} 3, set oneself in motion.

\noindent
\textbf{rīki} \textit{sn.} power; sovereignty, reign.

\noindent
\textbf{rīkr} \textit{adj.} powerful, distinguished.

\noindent
\textbf{ripti} \textit{sn.} linen cloth.

\noindent
\textbf{rita} \textit{wv.} 3, write.

\noindent
\textbf{rōa} \textit{sv.} 1, row.

\noindent
\textbf{rœða} \textit{wv.} 1, talk about, discuss.

\noindent
\textbf{röra} \textit{} \textit{see} \textbf{rōa}.

\noindent
\textbf{rǫst} \textit{sf.} league.

\emptypage

\chapter*{S}

\noindent
\textbf{sā} \textit{prn.} that; he; such, such a one.

\noindent
\textbf{sā} \textit{} \textit{see} \textbf{sjā}.

\noindent
\textbf{sær} \textit{sm.} sea.

\noindent
\textbf{særa} \textit{wv.} 1, wound [sār].

\noindent
\textbf{sæti} \textit{sn.} seat [sitja].

\noindent
\textbf{sætt} \textit{sf.} 2, reconciliation, peace.

\noindent
\textbf{sættask} \textit{wv.} 1, be reconciled, agree.

\noindent
\textbf{saga} \textit{wf.} narrative, history, story.

\noindent
\textbf{sagða} \textit{} \textit{see} \textbf{sęgja}.

\noindent
\textbf{saka} \textit{wv.} \textit{impers.\ w.\ acc.}---`hann (acc.) sakaði
	ekki,' he was not injured.

\noindent
\textbf{sakna} \textit{wv.} 3, \textit{w.\ gen.} miss.

\noindent
\textbf{sāl} \textit{sf.} 2, soul.

\noindent
\textbf{sāld} \textit{sn.} gallon.

\noindent
\textbf{salr} \textit{sm.} 2, hall.

\noindent
\textbf{saman} \textit{adv.} together.

\noindent
\textbf{sami} \textit{weak adj.} same.

\noindent
\textbf{sam-laga} \textit{wf.} laying ships together for battle.

\noindent
\textbf{samt} \textit{adv.} together.

\noindent
\textbf{sannligr} \textit{adj.} probable; suitable, right.

\noindent
\textbf{sannr} \textit{adj.} true.

\noindent
\textbf{sār} \textit{sn.} wound.

\noindent
\textbf{sārr} \textit{adj.} wounded.

\noindent
\textbf{sat} \textit{} \textit{see} \textbf{sitja}.

\noindent
\textbf{satt} \textit{} \textit{see} \textbf{sannr}.

\noindent
\textbf{sē} \textit{} \textit{see} \textbf{vera}.

\noindent
\textbf{sē} \textit{} \textit{see} \textbf{sjā}.

\noindent
\textbf{sefask} \textit{wv.} 3, be pacified.

\noindent
\textbf{sęgja} \textit{wv.} 1b, say, relate [saga].

\noindent
\textbf{segl} \textit{sn.} sail.

\noindent
\textbf{seilask} \textit{wv.} 1, stretch \textit{intr.}

\noindent
\textbf{seinn} \textit{adj.} late, slow, tedious.  \textbf{seint}
	\textit{adv.} slowly.

\noindent
\textbf{sęlja} \textit{wv.} 1, give; sell.

\noindent
\textbf{sem} \textit{adv.} as; \textit{w.\ subj.} as if; \textit{to strengthen
	the superl.}---`sem mest,' the most possible, as much as possible.

\noindent
\textbf{sęnda} \textit{wv.} 1, send.

\noindent
\textbf{sęndi-maðr} \textit{sm.} messenger.

\noindent
\textbf{sęnn} \textit{adv.} at the same time, at once; immediately,
	forthwith.

\noindent
\textbf{sēnn} \textit{} \textit{see} \textbf{sjā}.

\noindent
\textbf{sēr} \textit{} \textit{see} \textbf{sik}.

\noindent
\textbf{sēr} \textit{} \textit{see} \textbf{sjā}.

\noindent
\textbf{sēt} \textit{} \textit{see} \textbf{sjā}.

\noindent
\textbf{set-berg} \textit{sn.} seat-shaped rock, crag [sitja].

\noindent
\textbf{sętja} \textit{wv.} 1, set, place.  \textbf{s.\ fram}, launch (a
	ship).  \textbf{sętjask}, sit down.  \textbf{sętjask upp}, sit up
	[sitja].

\noindent
\textbf{sī-byrða} \textit{wv.} 1, \textit{w.\ dat.} lay a ship alongside
	another.  \textit{neut.\ ptc.} \textbf{sībyrt}, close up to [borð].

\noindent
\textbf{sīð} \textit{adv.} late.  \textit{comp.} sīðar, later, afterwards.
	\textit{superl.} \textbf{sīðast}, latest, last.

\noindent
\textbf{sīða} \textit{wf.} side.

\noindent
\textbf{sīðan} \textit{adv.} afterwards, then; since.

\noindent
\textbf{sīðari} \textit{adj.\ comp.} later, second (in order).

\noindent
\textbf{siðr} \textit{sm.} 3, custom.

\noindent
\textbf{sīga} \textit{sv.} 6, sink.

\noindent
\textbf{sigla} \textit{wf.} mast [segl].

\noindent
\textbf{sigla} \textit{wv.} 1, sail.

\noindent
\textbf{siglu-skeið} \textit{sn.} middle of a ship.

\noindent
\textbf{sigr (-rs)} \textit{sm.} victory.

\noindent
\textbf{sigr-ōp} \textit{} shout of victory.

\noindent
\textbf{sik} \textit{prn.} oneself.

\noindent
\textbf{silfr} \textit{sn.} silver.

\noindent
\textbf{sīn} \textit{} \textit{see} \textbf{sik}.

\noindent
\textbf{sinn} \textit{sn.} time (of repetition)---`einu sinni,' once,
	for once.  `eigi optar at sinni,' not oftener than that time,
	i.e.\ only once.

\noindent
\textbf{sinn} \textit{} \textit{see} \textbf{sik}.

\noindent
\textbf{sitja} \textit{sv.} 5, sit.  \textbf{s.\ fyrir}, sit in readiness.

\noindent
\textbf{sjā} \textit{} = þessi.

\noindent
\textbf{sjā} \textit{sv.} 5, see; `s.\ fyrir einu,' look after, take care of.
    \textit{impers.} `lītt sēr þat ā, at'\ldots it will hardly be seen
    that\ldots  \textbf{sjāsk}, see one another, meet.

\noindent
\textbf{sjālfr} \textit{prn.} self.

\noindent
\textbf{sjōða} \textit{sv.} 7, boil; cook.

\noindent
\textbf{sjōn} \textit{sf.} sight.

\noindent
\textbf{sjōn-hvęrfing} \textit{sf.} ocular delusion.

\noindent
\textbf{skal} \textit{} \textit{see} \textbf{skulu}.

\noindent
\textbf{skalf} \textit{} \textit{see} \textbf{skjālfa}.

\noindent
\textbf{skāli} \textit{wm.} hall.

\noindent
\textbf{skammr} \textit{adj.} short.

\noindent
\textbf{skap} \textit{sn.} state, condition; state of mind, mood, humour.

\noindent
\textbf{skapligr} \textit{adj.} suitable, fit.

\noindent
\textbf{skapt} \textit{sn.} shaft, handle.

\noindent
\textbf{skar} \textit{} \textit{see} \textbf{skera}.

\noindent
\textbf{skarð} \textit{sn.} notch, gap; defect.

\noindent
\textbf{skaut} \textit{} \textit{see} \textbf{skjōta}.

\noindent
\textbf{skęgg} \textit{sn.} beard; beak (of a ship).

\noindent
\textbf{skeið} \textit{sn.} race-course, running-ground;
	race---`taka sk.,' start in a race.

\noindent
\textbf{skeina} \textit{wv.} 1, graze.

\noindent
\textbf{skellr} \textit{sm.} knock.

\noindent
\textbf{skęmtun} \textit{sf.} amusement, entertainment [skammr,
	\textit{literally `shortening (of time)'}].

\noindent
\textbf{skera} \textit{sv.} cut, cut up; kill (animal).

\noindent
\textbf{skiljask} \textit{wv.} 1, separate, part \textit{intr.}

\noindent
\textbf{skillingr} \textit{sm.} shilling, coin.

\noindent
\textbf{skilnaðr} \textit{sm.} separation, parting.

\noindent
\textbf{skip} \textit{sn.} ship.

\noindent
\textbf{skipa} \textit{wv.} 3, order, arrange, prepare, fit out.
	`sk.\ til um eitt,' make arrangements for.  \textbf{skipask}, take
	one's place; change, alter \textit{intr.}

\noindent
\textbf{skipa-hęrr} \textit{sm.} fleet.

\noindent
\textbf{skipan} \textit{sf.} arranging; ship's crew.

\noindent
\textbf{skips-brot} \textit{sn.} shipwreck.

\noindent
\textbf{skip-stjōrnar-maðr} \textit{sm.} (steerer), commander of a ship,
    captain.

\noindent
\textbf{skjald-borg} \textit{sf.} wall of shields, testudo.

\noindent
\textbf{skjālfa} \textit{sv.} 3, shake \textit{intr.}

\noindent
\textbf{skjōta} \textit{sv.} 7, \textit{w.\ dat.} shoot, throw, push.

\noindent
\textbf{skjōt-fœri} \textit{sn.} swiftness.

\noindent
\textbf{skjōt-leikr} \textit{sm.} swiftness.

\noindent
\textbf{skjōt-liga} \textit{adv.} swiftly, quick.

\noindent
\textbf{skjōtr} \textit{adj.} swift, quick.  \textbf{skjōtt} \textit{adv.}
	quickly.

\noindent
\textbf{skjǫldr} \textit{sm.} 3, shield.

\noindent
\textbf{skōgr} \textit{sm.} forest, wood.

\noindent
\textbf{skorta} \textit{wv.} 1, \textit{impers.\ w.\ acc.\ of pers.\ and of
	thing}, want, fail.

\noindent
\textbf{skot} \textit{sn.} shot; missile [skjōta].

\noindent
\textbf{skot-māl} \textit{sn.} shot-measure, range.

\noindent
\textbf{skotta} \textit{wv.} 3, dangle---sk.\ við drift (of ships).

\noindent
\textbf{skǫkull} \textit{sm.} shaft (of a cart).

\noindent
\textbf{skǫmm} \textit{sf.} disgrace, shame.

\noindent
\textbf{skǫr} \textit{sf.} hair of the head.

\noindent
\textbf{skreppa} \textit{wf.} bag, wallet.

\noindent
\textbf{skulfu} \textit{} \textit{see} \textbf{skjālfa}.

\noindent
\textbf{skulu} \textit{swv.} shall.

\noindent
\textbf{skuta} \textit{wf.} small ship, cutter.

\noindent
\textbf{skutill} \textit{sm.} trencher, small table.

\noindent
\textbf{skutil-sveinn} \textit{sm.} page, chamberlain.

\noindent
\textbf{skutu} \textit{} \textit{see} \textbf{skjōta}.

\noindent
\textbf{skykkr} \textit{sm.} shake---`ganga skykkjum,' shake.

\noindent
\textbf{skylda} \textit{} \textit{see} \textbf{skulu}.

\noindent
\textbf{skyldr} \textit{adj.} obliged, obligatory, bound [skulu].

\noindent
\textbf{skyn} \textit{sn.} understanding, insight---`kunna, sk.' understand.

\noindent
\textbf{skynda} \textit{wv.} 1, hasten, bring in haste.

\noindent
\textbf{skyndiliga} \textit{adv.} hastily, quickly.

\noindent
\textbf{skynsamliga} \textit{adv.} intelligently, carefully.

\noindent
\textbf{skȳt} \textit{} \textit{see} \textbf{skjōta}.

\noindent
\textbf{slā} \textit{sv.} 2, strike.  `slā eldi ī,' light a fire.

\noindent
\textbf{slæliga} \textit{adv.} sluggishly, weakly.

\noindent
\textbf{slær} \textit{adj.} blunt.

\noindent
\textbf{slær} \textit{} \textit{see} \textbf{slā}.

\noindent
\textbf{slātr} \textit{sn.} meat.

\noindent
\textbf{sleikja} \textit{wv.} lick.

\noindent
\textbf{sleit} \textit{} \textit{see} \textbf{slīta}.

\noindent
\textbf{slēttr} \textit{adj.} level, smooth; comfortable, easy.

\noindent
\textbf{slīkr} \textit{adj.} such.

\noindent
\textbf{slīta} \textit{sv.} 6, tear---\textbf{sl.\ upp}, pull up; \textit{w.\ dat.}
	break (agreement).

\noindent
\textbf{smæri} \textit{} \textit{see} \textbf{smār}.

\noindent
\textbf{smār} \textit{adj.} small, insignificant.

\noindent
\textbf{smā-skip} \textit{snpl.} small ships.

\noindent
\textbf{smā-skūta} \textit{wf.} small cutter.

\noindent
\textbf{smjūga} \textit{sv.} 7, squeeze through, slip.

\noindent
\textbf{smugu} \textit{} \textit{see} \textbf{smjūga}.

\noindent
\textbf{snarpligr} \textit{adj.} vigorous.

\noindent
\textbf{snarpr} \textit{adj.} sharp; vigorous.

\noindent
\textbf{snimma} \textit{adv.} early.

\noindent
\textbf{snöra} \textit{} \textit{see} \textbf{snūa}.

\noindent
\textbf{snūa} \textit{sv.} 1, \textit{w.\ dat.} turn \textit{or} (\textit{trans.})
	direct; twist, plait.  \textbf{snūask}, turn (\textit{intr.}).

\noindent
\textbf{soðinn} \textit{} \textit{see} \textbf{sjōða}.

\noindent
\textbf{sœkja} \textit{wv.} 1c, seek; go---`s.\ aptr,' retreat.

\noindent
\textbf{sœmð} \textit{sf.} honour [sōma].

\noindent
\textbf{sœtti} \textit{} \textit{see} \textbf{sœkja}.

\noindent
\textbf{sofa} \textit{sv.} 4, sleep.

\noindent
\textbf{sofna} \textit{wv.} 3, go to sleep.

\noindent
\textbf{sǫgur} \textit{} \textit{see} \textbf{saga}.

\noindent
\textbf{sǫk} \textit{sf.} cause---`fyrir þā s.\ at'\ldots, because.

\noindent
\textbf{sökkva} \textit{sv.} 3, sink.

\noindent
\textbf{sōl} \textit{sf.} sun.

\noindent
\textbf{sōl-skin} \textit{sn.} sunshine.

\noindent
\textbf{soltinn} \textit{adj.} hungry [\textit{ptc.\ of} `svelta,' starve].

\noindent
\textbf{sōma} \textit{wv.} 2, \textit{w.\ dat.}, be suitable, befitting.

\noindent
\textbf{sōmi} \textit{wm.} honour.

\noindent
\textbf{sonr} \textit{sm.} son.

\noindent
\textbf{sōtt} \textit{sf.} illness.

\noindent
\textbf{sōtta} \textit{} \textit{see} \textbf{sœkja}.

\noindent
\textbf{sǫx} \textit{snpl.} raised prow of a war-ship.

\noindent
\textbf{spala} \textit{} \textit{see} \textbf{spǫlr}.

\noindent
\textbf{sparask} \textit{wv.} 2, spare oneself, reserve one's energy.

\noindent
\textbf{spęnna (spęnta)} \textit{wv.} 1, \textit{w.\ dat.\ of thing}, gird,
	buckle on.

\noindent
\textbf{spjōt} \textit{sn.} spear.

\noindent
\textbf{sporðr} \textit{sm.} tail.

\noindent
\textbf{spori} \textit{sm.} spur.

\noindent
\textbf{spǫlr} \textit{sm.} rail.

\noindent
\textbf{sprakk} \textit{} \textit{see} \textbf{springa}.

\noindent
\textbf{sprętta} \textit{wv.} 1, split.

\noindent
\textbf{springa} \textit{sv.} 3, burst.

\noindent
\textbf{spurða} \textit{} \textit{see} \textbf{spyrja}.

\noindent
\textbf{spyrja} \textit{wv.} 1b, ask; hear of, learn---`sp.\ til eins,' have
	news of, hear of his arrival.  \textbf{spyrjask}, be known.

\noindent
\textbf{spyrna} \textit{wv.} 1, kick.

\noindent
\textbf{staddr} \textit{adj.} placed, staying [\textit{ptc.\ of} `stęðja,'
	place].

\noindent
\textbf{staðr} \textit{sm.} place---`ī staðinn,' instead.

\noindent
\textbf{stafaðr} \textit{adj.\ (ptc.)} striped.

\noindent
\textbf{staf-karl} \textit{sm.} (staff-man), beggar.

\noindent
\textbf{stafn} \textit{sm.} prow.

\noindent
\textbf{stafn-būi} \textit{wm.} prow-man.

\noindent
\textbf{stafn-lē} \textit{wm.} grappling-hook.

\noindent
\textbf{stafr} \textit{sm.} 2 (\textit{gen.\ sg.} stafs), staff, stick.

\noindent
\textbf{stakk} \textit{} \textit{see} \textbf{stinga}.

\noindent
\textbf{stallari} \textit{sm.} marshall.

\noindent
\textbf{standa} \textit{sv.} 2, stand.  \textbf{st.\ upp}, stand up, rise.

\noindent
\textbf{starfa} \textit{wv.} 3, work.

\noindent
\textbf{stęfna} \textit{wv.} 1, steer; take a course, go.

\noindent
\textbf{steig} \textit{} \textit{see} \textbf{stīga}.

\noindent
\textbf{steinn} \textit{sm.} stone; jewel.

\noindent
\textbf{stela} \textit{sv.} 4, \textit{w.\ dat.\ of thing and acc.\ of pers.}
	steal, rob.

\noindent
\textbf{stęndr} \textit{} \textit{see} \textbf{standa}.

\noindent
\textbf{stęrkliga} \textit{adv.} vigorously.

\noindent
\textbf{stęrkr} \textit{adj.} strong.

\noindent
\textbf{steypa} \textit{wv.} 1, \textit{w.\ dat.\ of thing}, throw; pull off.
	\textbf{steypask}, throw oneself.

\noindent
\textbf{stīga} \textit{sv.} 6, advance, walk, go.  \textbf{st.\ upp}, mount
	(horse).

\noindent
\textbf{stīgr} \textit{sm.} path, way.

\noindent
\textbf{stikill} \textit{sm.} point.

\noindent
\textbf{stilla (stilta)} \textit{wv.} 1, arrange.  \textbf{st.\ til},
	arrange, dispose.

\noindent
\textbf{stinga} \textit{sv.} 3, pierce; `st.\ stǫfnum at skipi,' run the prow
    against a ship's side.

\noindent
\textbf{stirt} \textit{adv.} harshly [\textit{neut.\ of} `stirðr,' stiff].

\noindent
\textbf{stōð} \textit{} \textit{see} \textbf{standa}.

\noindent
\textbf{stœrri} \textit{} \textit{see} \textbf{stōr}.

\noindent
\textbf{stōkkva} \textit{sv.} 3, spring, rebound, start back.

\noindent
\textbf{stolinn} \textit{} \textit{see} \textbf{stela}.

\noindent
\textbf{stōrliga} \textit{adv.} bigly, arrogantly.

\noindent
\textbf{stōr-mannligr} \textit{adj.} magnificent, aristocratic.

\noindent
\textbf{stōr-męnni} \textit{sn.} great men (collective), aristocracy.

\noindent
\textbf{stōrr} \textit{adj.} big, great.  \textbf{stōrum} \textit{adv.} greatly.

\noindent
\textbf{stōr-rāðr} \textit{adj.} (great of plans), ambitious.

\noindent
\textbf{stōr-skip} \textit{sn.} big ship.

\noindent
\textbf{stōr-virki} \textit{sn.} great deed.

\noindent
\textbf{strā} \textit{wv.} 1, strew, cover with straw.

\noindent
\textbf{strauk} \textit{} \textit{see} \textbf{strjūka}.

\noindent
\textbf{stręngr} \textit{sm.} 2, string.

\noindent
\textbf{strjūka} \textit{sv.} 7, stroke.

\noindent
\textbf{stund} \textit{sf.} period of time, time.

\noindent
\textbf{stutt} \textit{adv.} shortly, abruptly [\textit{neut.\ of} `stuttr'
	short].

\noindent
\textbf{stȳra} \textit{wv.} 1, \textit{w.\ dat.} steer.

\noindent
\textbf{stȳri} \textit{sn.} rudder.

\noindent
\textbf{stȳris-hnakki} \textit{wm.} top piece of rudder.

\noindent
\textbf{styrkr} \textit{sm.} strength; help.

\noindent
\textbf{sū} \textit{} \textit{see} \textbf{sā}.

\noindent
\textbf{suðr} \textit{adv.} southward.

\noindent
\textbf{suðr-ganga} \textit{wf.} journey south (to Rome).

\noindent
\textbf{sukku} \textit{} \textit{see} \textbf{sökkva}.

\noindent
\textbf{sumar} \textit{sn.} summer.

\noindent
\textbf{sumr} \textit{prn.} some.

\noindent
\textbf{sund} \textit{sn.} sound, channel.

\noindent
\textbf{svā} \textit{adv.} so, as; as soon as.  `ok svā,' also.

\noindent
\textbf{svaf} \textit{} \textit{see} \textbf{sofa}.

\noindent
\textbf{svara} \textit{wv.} 3, \textit{w.\ dat.\ of thing}, answer.

\noindent
\textbf{svardagi} \textit{sm.} oath.

\noindent
\textbf{sveinn} \textit{sm.} boy.

\noindent
\textbf{sveinn-stauli} \textit{wm.} small boy.

\noindent
\textbf{sveit} \textit{sf.} troop.

\noindent
\textbf{svelga} \textit{sv.} 3, swallow, gulp.

\noindent
\textbf{sverð} \textit{sn.} sword.

\noindent
\textbf{sverðs-hǫgg} \textit{sn.} swordstroke.

\noindent
\textbf{svipan} \textit{sf.} jerk; moment.

\noindent
\textbf{svipting} \textit{sf.} pull, struggle.

\noindent
\textbf{sȳnask} \textit{wv.} 1, seem [sjōn].

\noindent
\textbf{syni} \textit{} \textit{see} \textbf{sonr}.

\noindent
\textbf{systir} \textit{sf.} 3, sister.

\emptypage

\chapter*{T}

\noindent
\textbf{taka} \textit{wv.} 2, \textit{w.\ acc.} take, seize, take possession
	of; \textit{w.\ inf.} begin; \textit{w.\ dat.} receive (well, ill, etc.).
	\textbf{takask}, take place, begin.  \textbf{t.\ at}, choose.  \textbf{t.
	til}, engage in, try.  \textbf{t.\ upp}, take to, choose.

\noindent
\textbf{tala} \textit{wf.} talk, speech.

\noindent
\textbf{tala} \textit{wv.} 3, speak, talk about, discuss.

\noindent
\textbf{taliðr} \textit{} \textit{see} \textbf{tęlja}.

\noindent
\textbf{tālma} \textit{wv.} 3, hinder.

\noindent
\textbf{tār} \textit{sn.} tear.

\noindent
\textbf{taumar} \textit{smpl.} reins.

\noindent
\textbf{tękinn} \textit{} \textit{see} \textbf{taka}.

\noindent
\textbf{tęlja} \textit{wv.} 1b, count, recount; account, consider; relate,
	say [tala].

\noindent
\textbf{tęngðir} \textit{sfpl.} relationship, connection by marriage.

\noindent
\textbf{tęngja} \textit{wv.} 1, bind, fasten together.

\noindent
\textbf{tęngsl} \textit{snpl.} cable.

\noindent
\textbf{tīða} \textit{wv.} 1, \textit{impers.\ w.\ acc.} desire.

\noindent
\textbf{tīðindi} \textit{snpl.} tidings, news.

\noindent
\textbf{tīðr} \textit{adj.} usual, happening---`hvat er tītt um þik?' what
	is the matter with you?  \textbf{tītt} \textit{adv.} often, quickly---`sem
	tīðast,' as quickly as possible.

\noindent
\textbf{tiginn} \textit{adj.} of high rank.

\noindent
\textbf{tigr} \textit{sm.} ---`fjōrir tigir,' forty.

\noindent
\textbf{til} \textit{prp.} \textit{w.\ gen.} to; till; for (of use)---`alt
	er t.\ vāpna var,' everything that could be used as a missile; for
	(of object, intention)---`brjōta lęgg til męrgjar,' break a leg to
    get at the marrow; with respect to---`til vista var eigi
    gott,' they were not well off for provisions.

\noindent
\textbf{til} \textit{adv.} too (of excess).

\noindent
\textbf{til-vīsan} \textit{sf.} direction, guidance.

\noindent
\textbf{tītt} \textit{} \textit{see} \textbf{tīðr}.

\noindent
\textbf{tīvi} \textit{wm.} god.

\noindent
\textbf{tjald} \textit{sn.} tent.

\noindent
\textbf{tjūgu-skęgg} \textit{sn.} forked beard.

\noindent
\textbf{tœka} \textit{} \textit{see} \textbf{taka}.

\noindent
\textbf{tōk} \textit{} \textit{see} \textbf{taka}.

\noindent
\textbf{tǫnn} \textit{sf.} 3, tooth.

\noindent
\textbf{trani} \textit{wm.} crane.

\noindent
\textbf{trē} \textit{sn.} tree.

\noindent
\textbf{troða} \textit{sv.} tread.

\noindent
\textbf{trog} \textit{sn.} trough.

\noindent
\textbf{tros} \textit{sn.} droppings, rubbish.

\noindent
\textbf{trūa} \textit{wf.} faith---`þat veit tr.\ mīn at\ldots,' by
	my faith.

\noindent
\textbf{trūa} \textit{wv.} 2, \textit{w.\ dat.\ of pers.} believe, trust,
	rely on.

\noindent
\textbf{tūn} \textit{sn.} enclosure, dwelling.

\noindent
\textbf{tveir} \textit{num.} two.

\noindent
\textbf{tȳna} \textit{wv.} 1, \textit{w.\ dat.} lose.

\noindent
\textbf{typpa} \textit{wv.} tie in a top-knot.

\emptypage

\chapter*{Þ}

\noindent
\textbf{þā} \textit{} \textit{see} \textbf{sā}.

\noindent
\textbf{þā} \textit{} \textit{see} \textbf{þiggja}.

\noindent
\textbf{þā} \textit{adv.} then.

\noindent
\textbf{þær} \textit{} \textit{see} \textbf{sā}.

\noindent
\textbf{þakðr} \textit{} \textit{see} \textbf{þękja}.

\noindent
\textbf{þakka} \textit{wv.} 3, \textit{w.\ acc.\ of thing and dat.\ of
	pers.} thank; requite, reward.

\noindent
\textbf{þambar-skęlfir} \textit{sm.} bowstring-shaker (?).

\noindent
\textbf{þann} \textit{} \textit{see} \textbf{sā}.

\noindent
\textbf{þannig} \textit{adv.} thither; so [= þann veg].

\noindent
\textbf{þar} \textit{adv.} there; then; `þ.\ af,' \textit{etc.},
	thereof.  `þar sem,' since, because.  `þar til,' until.

\noindent
\textbf{þarf} \textit{} \textit{see} \textbf{þurfa}.

\noindent
\textbf{þat} \textit{} \textit{see} \textbf{sā}.

\noindent
\textbf{þau} \textit{} \textit{see} \textbf{sā}.

\noindent
\textbf{þegar} \textit{adv.} at once.  `þ.\ er,' as soon as.

\noindent
\textbf{þęginn} \textit{} \textit{see} \textbf{þiggja}.

\noindent
\textbf{þęgja} \textit{wv.} 1b, be silent.

\noindent
\textbf{þeim} \textit{} \textit{see} \textbf{sā}.

\noindent
\textbf{þeir, þeirra} \textit{} \textit{see} \textbf{sā}.

\noindent
\textbf{þękja} \textit{wv.} 1b, roof.

\noindent
\textbf{þękkja} \textit{wv.} 1, notice.  \textbf{þękkjask}, take pleasure
	in; accept.

\noindent
\textbf{þēr} \textit{} \textit{see} \textbf{þū}.

\noindent
\textbf{þess} \textit{} \textit{see} \textbf{sā}.

\noindent
\textbf{þess} \textit{adv.\ w.\ comp.} the, so much the.

\noindent
\textbf{þessi} \textit{prn.} this.

\noindent
\textbf{þiggja} \textit{sv.} 5, receive.

\noindent
\textbf{þik} \textit{} \textit{see} \textbf{þū}.

\noindent
\textbf{þing} \textit{sn.} meeting, parliament.

\noindent
\textbf{þinn} \textit{} \textit{see} \textbf{þū}.

\noindent
\textbf{þit} \textit{} \textit{see} \textbf{þū}.

\noindent
\textbf{þjōð} \textit{sf.} nation, race.

\noindent
\textbf{þō} \textit{adv.} though, yet.

\noindent
\textbf{þǫkð} \textit{} \textit{see} \textbf{þękja}.

\noindent
\textbf{þǫkk} \textit{sf.} thanks, gratitude [þakka].

\noindent
\textbf{þola} \textit{wv.} 2, endure, put up with.

\noindent
\textbf{þora} \textit{wv.} 2, dare.

\noindent
\textbf{þǫrf} \textit{sf.} need.

\noindent
\textbf{þorrinn} \textit{} \textit{see} \textbf{þverra}.

\noindent
\textbf{þōtt} \textit{adv.} though [= þō at].

\noindent
\textbf{þōtta} \textit{} \textit{see} \textbf{þykkja}.

\noindent
\textbf{þraut} \textit{} \textit{see} \textbf{þrjōta}.

\noindent
\textbf{þreifask} \textit{wv.} 3, grope, feel.

\noindent
\textbf{þreyta} \textit{wv.} 1, make exertions, try.

\noindent
\textbf{þriði} \textit{adj.} third.

\noindent
\textbf{þriðjungr} \textit{sm.} third.

\noindent
\textbf{þrīr} \textit{num.} three.

\noindent
\textbf{þrjōta} \textit{sv.} 7, \textit{impers.\ w.\ acc.\ of pers.} come to
	an end, fail.

\noindent
\textbf{þrūðugr} \textit{adj.} mighty.

\noindent
\textbf{þū} \textit{prn.} thou.

\noindent
\textbf{þumlungr} \textit{sm.} thumb of glove.

\noindent
\textbf{þunn-vangi} \textit{wm.} temple (of head).

\noindent
\textbf{þurðr} \textit{sm.} diminution.

\noindent
\textbf{þurfa} \textit{swv.} \textit{often impers.} require, need.

\noindent
\textbf{þurr} \textit{adj.} dry.

\noindent
\textbf{þurs} \textit{sm.} giant.

\noindent
\textbf{þverra} \textit{sv.} 3, diminish.

\noindent
\textbf{þvers} \textit{adv.} across.

\noindent
\textbf{þvī} \textit{} \textit{see} \textbf{sā}.

\noindent
\textbf{þvī} \textit{adv.} therefore; \textit{w.\ compar.} the, so much the
	more.

\noindent
\textbf{þvīlikr} \textit{adj.} such.

\noindent
\textbf{þykkja} \textit{wv.} 1c, seen, be considered.  `þykkir einum
	fyrir,' there seems to be something in the way, one hesitates.
    `myndi mēr fyrir þ.\ ī,' I should be displeased.  \textbf{þykkjask},
    think.

\noindent
\textbf{þykkr} \textit{adj.} thick, close.

\noindent
\textbf{þynna} \textit{wv.} 1, make thin.  \textbf{þynnask} get thin.

\noindent
\textbf{þyrstr} \textit{adj.\ (ptc.)} thirsty.

\emptypage

\chapter*{U}

\noindent
\textbf{ū-fœra} \textit{wf.} impassable place; fix, difficulty.

\noindent
\textbf{ū-friðr} \textit{sm.} hostility, war.

\noindent
\textbf{ū-grynni} \textit{sn.} countless number [grunnr, `bottom'].

\noindent
\textbf{ū-happ} \textit{sn.} misfortune.

\noindent
\textbf{ū-hreinn} \textit{adj.} impure.

\noindent
\textbf{um} \textit{prp.} \textit{w.\ acc.} around, about, over; \textit{of
	time} in, at; \textit{of superiority} beyond; concerning, about.

\noindent
\textbf{um-rāð} \textit{sn.} advice, help.

\noindent
\textbf{um-sjā} \textit{sf.} care.

\noindent
\textbf{una} \textit{wv.} 1, \textit{w.\ dat.} be contented---`u.\ illa,' be
	discontented.

\noindent
\textbf{undan} \textit{prp.} \textit{w.\ dat.}, \textit{adv.} away (from).

\noindent
\textbf{undarliga} \textit{adv.} strangely.

\noindent
\textbf{undarligr} \textit{adj.} strange.

\noindent
\textbf{undir} \textit{prp.} \textit{w.\ acc.\ and dat.} under.

\noindent
\textbf{undr} \textit{sn.} wonder.

\noindent
\textbf{ungr} \textit{adj.} young.

\noindent
\textbf{unninn} \textit{} \textit{see} \textbf{vinna}.

\noindent
\textbf{unz} \textit{adv.} until.

\noindent
\textbf{upp} \textit{adv.} up.

\noindent
\textbf{upp-ganga} \textit{wf.} boarding (ship).

\noindent
\textbf{upp-haf} \textit{sn.} beginning [hęfja].

\noindent
\textbf{upp-himinn} \textit{sm.} high heaven.

\noindent
\textbf{uppi} \textit{adv.} up; at an end.

\noindent
\textbf{urðu} \textit{} \textit{see} \textbf{verða}.

\noindent
\textbf{ūt} \textit{adv.} out.  \textit{comp.} \textbf{utar}, outer, outwards,
	farther away.

\noindent
\textbf{utan} \textit{adv.} outside; outwards.

\noindent
\textbf{utan-fęrð} \textit{sf.} journey abroad.

\noindent
\textbf{utar} \textit{} \textit{see} \textbf{ūt}.

\noindent
\textbf{ūti} \textit{adv.} outside, out on the sea.

\noindent
\textbf{ūt-lausn} \textit{sf.} ransom.

\noindent
\textbf{ū-vinr} \textit{sm.} enemy.

\emptypage

\chapter*{V}

\noindent
\textbf{vægð} \textit{sf.} forbearance.

\noindent
\textbf{vænn} \textit{adj.} likely, to be expected.

\noindent
\textbf{væri} \textit{} \textit{see} \textbf{vera}.

\noindent
\textbf{vætr} \textit{neut.} nothing.

\noindent
\textbf{vaka} \textit{wv.} 3, be awake, wake up.

\noindent
\textbf{vakna} \textit{wv.} 3, awake.

\noindent
\textbf{vald} \textit{sn.} power, control.

\noindent
\textbf{val-kyrja} \textit{sf.} chooser of the slain,
	war-goddess [kjōsa].

\noindent
\textbf{valr} \textit{sm.} corpses on the battle-field.

\noindent
\textbf{vān,} \textit{sf.} hope, expectation, probability.

\noindent
\textbf{vandræða-skāld} \textit{sn.} the `awkward' poet, the poet
	who is difficult to deal with.

\noindent
\textbf{vandræði} \textit{snpl.} difficulty.

\noindent
\textbf{vangi} \textit{wm.} cheek.

\noindent
\textbf{vanr} \textit{adj.} accustomed.

\noindent
\textbf{vāpn} \textit{sn.} weapon.

\noindent
\textbf{vāpna-burð} \textit{sm.} bearing weapons, shower of missiles.

\noindent
\textbf{vāpn-lauss} \textit{adj.} without weapons.

\noindent
\textbf{var} \textit{} \textit{see} \textbf{vera}.

\noindent
\textbf{vār} \textit{sn.} spring.

\noindent
\textbf{vār} \textit{} \textit{see} \textbf{ek}.

\noindent
\textbf{vara} \textit{wv.} 2, \textit{impers.\ w.\ acc.\ of pers.}---`mik varði,'
	I expected.

\noindent
\textbf{varð} \textit{} \textit{see} \textbf{verða}.

\noindent
\textbf{varði} \textit{} \textit{see} \textbf{vęrja}.

\noindent
\textbf{vargr} \textit{sm.} wolf.

\noindent
\textbf{variðr} \textit{} \textit{see} \textbf{vęrja}.

\noindent
\textbf{varla} \textit{adv.} scarcely.

\noindent
\textbf{varnaðr} \textit{sm.} goods, merchandise.

\noindent
\textbf{vāru} \textit{} \textit{see} \textbf{vera}.

\noindent
\textbf{vatn} \textit{sn.} water.

\noindent
\textbf{vaxa} \textit{sv.} 2, grow; increase.

\noindent
\textbf{veðr} \textit{sn.} weather.

\noindent
\textbf{vegr} \textit{sm.} road; way, manner; direction, side;
	\textit{in composition}, region, tract, land.

\noindent
\textbf{veik} \textit{} \textit{see} \textbf{vikja}.

\noindent
\textbf{veit} \textit{} \textit{see} \textbf{vita}.

\noindent
\textbf{veita} \textit{wv.} 1, give, grant; make (resistance, etc.).

\noindent
\textbf{vękja} \textit{wv.} 1b, wake [vaka].

\noindent
\textbf{vēl} \textit{sf.} artifice, cunning.

\noindent
\textbf{vęlja} \textit{wv.} 1b, choose.

\noindent
\textbf{vęndi} \textit{} \textit{see} \textbf{vǫndr}.

\noindent
\textbf{vēr} \textit{} \textit{see} \textbf{ek}.

\noindent
\textbf{vera} \textit{sv.} exist; remain, stay, happen; be.  `hvat
	lātum hafði verit,' what had caused the noise.  \textbf{v.\ at}, be
	occupied with.

\noindent
\textbf{verð} \textit{sn.} worth, value; price.

\noindent
\textbf{verða} \textit{sv.} 3, happen; happen to come.  `v.\ fyrir
	einum,' come in one's way, appear before one; become; come into
	being, be; `v.\ til eins,' be ready for, undertake; \textit{w.\ infin.}
	be obliged, must.  `nū er ā orðit mikit fyrir mēr,' now I have
    come into a great perplexity, difficulty.  \textbf{v.\ at}, happen.

\noindent
\textbf{verðr} \textit{adj.} worth; important.

\noindent
\textbf{ver-gjarn} \textit{adj.} desirous of a husband, loose.

\noindent
\textbf{verit} \textit{} \textit{see} \textbf{vera}.

\noindent
\textbf{vert} \textit{} \textit{see} \textbf{verðr}.

\noindent
\textbf{vęrja} \textit{wv.} 1b, defend---`v.\ baki,' defend with the back,
	turn one's back on.

\noindent
\textbf{vęrja} \textit{wv.} 1b, dress; lay out money, invest.

\noindent
\textbf{verk} \textit{sn.} work, job.

\noindent
\textbf{verr} \textit{sm.} man; husband.

\noindent
\textbf{vęrr} \textit{adv.} \textit{comp.} worse.  \textit{superl.}
	\textbf{vęrst}, worst.

\noindent
\textbf{verǫld} \textit{sf.} world.

\noindent
\textbf{vestan} \textit{adv.} from the west.  `v.\ fyrir' \textit{w.\ gen.\ or
	acc.}, west of.

\noindent
\textbf{vestr} \textit{adv.} westwards.

\noindent
\textbf{vetr} \textit{sm.} winter; year.

\noindent
\textbf{vęx} \textit{} \textit{see} \textbf{vaxa}.

\noindent
\textbf{vęxti} \textit{} \textit{see} \textbf{vǫxtr}.

\noindent
\textbf{við} \textit{prp.} \textit{w.\ acc.} by, near; towards (of place
	and time); with (of various relations).  \textit{w.\ dat.} towards, at
	(laugh at, etc.); in exchange for, for.

\noindent
\textbf{vīða} \textit{adv.} widely, far and wide, on many sides.

\noindent
\textbf{viðar-teinungr} \textit{sm.} tree-shoot, plant.

\noindent
\textbf{við-bragð} \textit{sn.} push.

\noindent
\textbf{viðr} \textit{sm.} 3, tree.

\noindent
\textbf{vīðr} \textit{adj.} wide, broad.

\noindent
\textbf{við-skipti} \textit{snpl.} dealings.

\noindent
\textbf{við-taka} \textit{wf.} reception; resistance.

\noindent
\textbf{vīgja} \textit{wv.} 1, consecrate, hallow.

\noindent
\textbf{vīgr} \textit{adj.} warlike, able-bodied.

\noindent
\textbf{vīking} \textit{sf.} piracy, piratical expedition.

\noindent
\textbf{vīkingr} \textit{sm.} pirate.

\noindent
\textbf{vīkja} \textit{sv.} 6, turn, move, go.

\noindent
\textbf{vilja} \textit{swv.} will.

\noindent
\textbf{villi-eldr} \textit{sm.} wildfire, conflagration.

\noindent
\textbf{vinātta} \textit{wf.} friendship.

\noindent
\textbf{vināttu-māl} \textit{snpl.} assurances of friendship.

\noindent
\textbf{vinna} \textit{sv.} 3, do, perform; win, conquer.  \textbf{vinnask
	til}, suffice.

\noindent
\textbf{vinr} \textit{sm.} 2, friend.

\noindent
\textbf{virðing} \textit{sf.} honour [verðr].

\noindent
\textbf{vīsa} \textit{wv.} 2, \textit{w.\ dat.} show, guide.

\noindent
\textbf{vīss} \textit{adj.} wise; certain.

\noindent
\textbf{vissa} \textit{} \textit{see} \textbf{vita}.

\noindent
\textbf{vist} \textit{sf.} 2, board and lodging; provisions.

\noindent
\textbf{vit} \textit{} \textit{see} \textbf{ek}.

\noindent
\textbf{vita} \textit{swv.} know; be turned in a certain direction.
	\textbf{v.\ fram}, see into futurity.

\noindent
\textbf{vīti} \textit{sn.} punishment, penalty.

\noindent
\textbf{vītis-horn} \textit{sn.} penalty-horn (whose contents were drunk
	as a punishment).

\noindent
\textbf{vitkask} \textit{wv.} 3, come to one's senses.

\noindent
\textbf{vitr (-ran)} \textit{adj.} wise.

\noindent
\textbf{vǫllr} \textit{sm.} 3, plain, field.

\noindent
\textbf{vǫndr} \textit{sm.} 3, twig, rod.

\noindent
\textbf{vǫrn} \textit{sf.} defence, resistance.

\noindent
\textbf{vǫxtr} \textit{sm.} 3, growth, stature.

\noindent
\textbf{vreiðr} \textit{} = reiðr.

\emptypage

\chapter*{Y}

\noindent
\textbf{yðr} \textit{} \textit{see} \textbf{þū}.

\noindent
\textbf{yfir} \textit{prp.} \textit{w.\ acc.\ and dat.} over [ofan].

\noindent
\textbf{ȳmiss} \textit{adj.} various, different.

\noindent
\textbf{ymr} \textit{sm.} rumbling noise.

\noindent
\textbf{ytri} \textit{adj.} \textit{comp.} outer.  \textit{superl.} \textbf{yztr},
	outside(st) [ūt].



\part{Proper Names}

\noindent
\textbf{Āki} \textit{sm.} 

\noindent
\textbf{Āsa-þōrr} \textit{sm.} (divine) Thor.

\noindent
\textbf{Āstrīðr} \textit{sf.} 

\noindent
\textbf{Auðun} \textit{sm.} 

\vspace{\baselineskip}

\noindent
\textbf{Baldr (-rs)} \textit{sm.} 

\noindent
\textbf{Barði} \textit{wm.} `the Ram' (name of a ship).

\noindent
\textbf{Bil-skīrnir} \textit{sm.} 

\noindent
\textbf{Brīsinga-męn} \textit{sn.} Freyja's necklace.

\noindent
\textbf{Burizleifr} \textit{sm.} 

\vspace{\baselineskip}

\noindent
\textbf{Dāins-leif} \textit{sf.} relic of Dāinn [`leif,' leaving, heritage].

\noindent
\textbf{Dana-konungr} \textit{sm.} king of the Danes.

\noindent
\textbf{Danir} \textit{smpl.} the Danes.

\noindent
\textbf{Dan-mǫrk} \textit{sf.} 3, Denmark.

\noindent
\textbf{Draupnir} \textit{sm.} 

\vspace{\baselineskip}

\noindent
\textbf{Einarr} \textit{sm.} 

\noindent
\textbf{Eindriði} \textit{wm.} 

\noindent
\textbf{Eirīkr} \textit{sm.} 

\noindent
\textbf{Erlingr} \textit{sm.} 

\vspace{\baselineskip}

\noindent
\textbf{Fęn-salr} \textit{sm.} 2.

\noindent
\textbf{Finnr} \textit{sm.} 

\noindent
\textbf{Finskr} \textit{adj.} Finnish.

\noindent
\textbf{Freyja} \textit{wf.} 

\noindent
\textbf{Freyr} \textit{sm.} 

\noindent
\textbf{Frigg} \textit{sf.} wife of Odin.

\noindent
\textbf{Fulla} \textit{wf.} Frigg's handmaid.

\vspace{\baselineskip}

\noindent
\textbf{Geira} \textit{wf.} 

\noindent
\textbf{Gimsar} \textit{fpl.} 

\noindent
\textbf{Gjallar-brū} \textit{sf.} the bridge over the river Gjǫll.

\noindent
\textbf{Gjǫll} \textit{sf.} `the Resounder,' the river of Hell.

\noindent
\textbf{Grœn-land} \textit{sn.} Greenland.

\noindent
\textbf{Gullin-bursti} \textit{wm.} `Golden-bristle.'

\noindent
\textbf{Gull-toppr} \textit{sm.} `Gold-top.'

\vspace{\baselineskip}

\noindent
\textbf{Hā-ey} \textit{sf.} `High-island.'

\noindent
\textbf{Hākon} \textit{sm.} 

\noindent
\textbf{Hall-freðr} \textit{sm.} 

\noindent
\textbf{Haraldr} \textit{sm.} 

\noindent
\textbf{Hēðinn} \textit{sm.} 

\noindent
\textbf{Heim-dallr} \textit{sm.} 

\noindent
\textbf{Hęl} \textit{sf.} the goddess of the infernal regions.

\noindent
\textbf{Hęr-mōðr} \textit{sm.} 

\noindent
\textbf{Hildr} \textit{sf.} [`hildr,' war].

\noindent
\textbf{Hjaðninga-vīg} \textit{sn.} battle of the Hjaðnings.

\noindent
\textbf{Hjarrandi} \textit{wm.} 

\noindent
\textbf{Hlōriði} \textit{wm.} the Thunderer, Thor.

\noindent
\textbf{Hǫgni} \textit{wm.} 

\noindent
\textbf{Hring-horni} \textit{wm.} `Ring-prowed.'

\noindent
\textbf{Hyrningr} \textit{sm.} 

\noindent
\textbf{Hyrrokin} \textit{sf.} 

\noindent
\textbf{Hǫðr} \textit{sm.} 

\vspace{\baselineskip}

\noindent
\textbf{Īs-land} \textit{sn.} Iceland.

\noindent
\textbf{Īsland-fęrð} \textit{sf.} journey to Iceland.

\noindent
\textbf{Īslęnzkr} \textit{adj.} Icelandic.

\vspace{\baselineskip}

\noindent
\textbf{Jōms-borg} \textit{sf.} 

\noindent
\textbf{Jōms-vīkingar} \textit{smpl.} the pirates of Jōmsborg.

\vspace{\baselineskip}

\noindent
\textbf{Kol-bjǫrn} \textit{sm.} 

\vspace{\baselineskip}

\noindent
\textbf{Laufey} \textit{sf.} 

\noindent
\textbf{Litr} \textit{sm.} 

\noindent
\textbf{Loki} \textit{wm.} 

\vspace{\baselineskip}

\noindent
\textbf{Mōð-guðr} \textit{sf.} 

\noindent
\textbf{Mœri} \textit{wf.} 

\vspace{\baselineskip}

\noindent
\textbf{Nanna} \textit{wf.} 

\noindent
\textbf{Nepr} \textit{sm.} 

\noindent
\textbf{Njarðar} \textit{} \textit{see} \textbf{Njǫrðr}.

\noindent
\textbf{Njǫrðr} \textit{sm.} 

\noindent
\textbf{Nōa-tūn} \textit{snpl.} 

\noindent
\textbf{Noregr} \textit{sm.} Norway [= norð-vegr].

\vspace{\baselineskip}

\noindent
\textbf{Ǫku-þōrr} \textit{sm.} Thor (the driver) [aka].

\noindent
\textbf{Ōlāfr} \textit{sm.} 

\noindent
\textbf{Ōðinn} \textit{sm.} Odin.

\noindent
\textbf{Orkn-eyjar} \textit{sfpl.} Orkneys.

\vspace{\baselineskip}

\noindent
\textbf{Rōma-borg} \textit{sf.} Rome.

\noindent
\textbf{Rūm-fęrli} \textit{wm.} pilgrim to Rome [fara].

\noindent
\textbf{Rǫskva} \textit{wf.} 

\vspace{\baselineskip}

\noindent
\textbf{Sax-land} \textit{sn.} Saxony, Germany.

\noindent
\textbf{Sif} \textit{sf.} 

\noindent
\textbf{Sigrīðr} \textit{sf.} 

\noindent
\textbf{Sig-valdi} \textit{wm.} 

\noindent
\textbf{Skjālgr} \textit{sm.} 

\noindent
\textbf{Skrȳmir} \textit{sm.} 

\noindent
\textbf{Sleipnir} \textit{sm.} 

\noindent
\textbf{Slīðrug-tanni} \textit{wm.} [tǫnn].

\noindent
\textbf{Sœnskr} \textit{adj.} Swedish.

\noindent
\textbf{Sveinn} \textit{sm.} 

\noindent
\textbf{Svīa-konungr} \textit{sm.} king of Sweden.

\noindent
\textbf{Svīar} \textit{smpl.} Swedes.

\noindent
\textbf{Svīa-vęldi} \textit{sn.} Sweden [vald].

\noindent
\textbf{Svī-þjōð} \textit{sf.} Sweden.

\noindent
\textbf{Svǫlðr (-rar)} \textit{sf.} island of Svolder, near Rügen.

\vspace{\baselineskip}

\noindent
\textbf{Tann-gnjōstr} \textit{sm.} [tǫnn].

\noindent
\textbf{Tann-grisnir} \textit{sm.} [tǫnn].

\noindent
\textbf{Tryggvi} \textit{wm.} `Trusty.'

\vspace{\baselineskip}

\noindent
\textbf{Þjālfi} \textit{wm.} 

\noindent
\textbf{Þōrir} \textit{sm.} 

\noindent
\textbf{Þor-kęll} \textit{sm.} 

\noindent
\textbf{Þōrr} \textit{sm.} Thor.

\noindent
\textbf{Þor-steinn} \textit{sm.} 

\noindent
\textbf{Þrūð-vangar} \textit{smpl.} `plains of strength.'

\noindent
\textbf{Þrymyr} \textit{sm.} 

\noindent
\textbf{Þǫkk} \textit{sf.} 

\vspace{\baselineskip}

\noindent
\textbf{Ūlfr} \textit{sm.} 

\noindent
\textbf{Ūtgarða-loki} \textit{wm.} 

\noindent
\textbf{Ūt-garðr} \textit{sm.} \textbf{Ūt-garðar} \textit{pl.} `outer enclosure,'
	world of the giants.

\vspace{\baselineskip}

\noindent
\textbf{Vanir} \textit{smpl.} race of Gods.

\noindent
\textbf{Vār} \textit{sf.} goddess of betrothal and marriage.

\noindent
\textbf{Vest-firzkr} \textit{adj.} of the west firths (in Iceland).

\noindent
\textbf{Vīk} \textit{sf.} `the Bay,' the Skagerak and the Christiania fjord.

\noindent
\textbf{Vinda-snękkja} \textit{wf.} Wendish ship.

\noindent
\textbf{Vind-land} \textit{sn.} Wendland.

\noindent
\textbf{Vindr} \textit{smpl.} the Wends.

\noindent
\textbf{Ving-þōrr} \textit{sm.} name of Thor.

\end{document}
